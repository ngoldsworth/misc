\recipe{Earl Grey Lavender Scones}

Shamelessly stolen from:
\texttt{\small https://flourcoveredapron.com/lavender-earl-grey-scones/}


\ingredients
\textit{Scones}
\begin{itemize}
    \item 2 cups all-purpose flour
    \item \nicefrac{1}{4} cup granulated sugar
    \item 1 tablespoon baking powder
    \item \nicefrac{1}{2} teaspoon salt
    \item 3 tea bags, Earl Grey blend
    \item \nicefrac{1}{2} cup unsalted butter, frozen
    \item \nicefrac{1}{2} cup heavy whipping cream
    \item 1 large egg
    \item 2 teaspoons pure vanilla extract
\end{itemize}

\textit{Glaze}
\begin{itemize}
    \item 1 \nicefrac{1}{2} teaspoons culinary lavender, plus more for garnishing
    \item \nicefrac{1}{2} cup milk
    \item 2 tea bags, Earl Grey blend
    \item 1 \nicefrac{1}{4} cup powdered sugar (add more as desired for a thicker glaze)
\end{itemize}

\instructions
\textit{Scones}
\begin{enumerate}
    \item Preheat oven to 400 degrees Fahrenheit and place a silicone baking mat on a large baking sheet.
    \item In a large bowl, whisk together the flour, sugar, baking powder, and salt. 
        Open the 3 tea bags and empty the ground tea leaves directly into the flour mixture, whisk to combine.
    \item Grate the frozen butter using a box grater and cut it into the flour
    mixture using a pastry cutter until incorporated and the mixture has the
    texture of fine crumbs.\footnote{Cold, grated butter will give you the
    fluffiest and flakiest scones! But if you prefer, you can cube the cold
    butter (rather than grate it) before cutting it into the dough. If you don’t
    have a pastry cutter, use a fork, or even your hands (but work quickly, so
    as not to let the butter get too warm)}
    \item In a small bowl, whisk together the cream, egg, and vanilla extract.
    Make a well in the middle of the dry ingredients, pour in the wet
    ingredients, then mix together.
    \item Once completely combined, turn the dough out onto a lightly floured
    surface and knead until it comes together. Shape the dough into a disc about
    8 inches in diameter. Slice into 8 wedges.
    \item Place the wedges about 2 inches apart on the silicone-lined baking
    sheet and bake at 400 degrees for 18-22 minutes, until scones are tall and
    golden brown. Transfer to a wire rack to cool completely.
\end{enumerate}

\textit{Glaze}
\begin{enumerate}
    \item In a small saucepan, bring the milk and lavender to a simmer. Remove from heat, then steep with the two bags of Earl Grey tea for about 7 minutes.
    \item Using a small sieve, strain the mixture and discard the lavender. Measure out 1/4 cup of the steeped milk. Pour into a small bowl and whisk together with the powdered sugar. For a thicker glaze, add more powdered sugar until desired consistency and sweetness are reached.
    \item Once scones are cool, drizzle with glaze and sprinkle with lavender
\end{enumerate}