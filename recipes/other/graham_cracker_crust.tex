\recipe{Graham Cracker Crust}

\textit{No-Bake vs Baked Graham Cracker Crust}:
\begin{itemize}
    \item \textbf{No-Bake Recipesk}: If the crust and pie will not be baked,
    prepare the pie crust as directed, then place the pan in the freezer for at
    least 15 minutes before filling it.
    \item \textbf{Baked Recipes}: If the graham cracker crust needs to be
    pre-baked, bake at 350 degrees Farenheit for 10 minutes, then proceed with your
    recipe as written.
\end{itemize}


\textit{Make Ahead and Freezing Instructions}
\begin{itemize}
    \item \textbf{Make Ahead}: Prepare the graham cracker crust up to 2 days in
    advance of filling it. Simply cover it with plastic wrap and refrigerate it
    until ready to fill.
    \item \textbf{Freeze}: Prepare the crust as instructed, then wrap the pie
    plate in plastic wrap and then in a layer of aluminum foil. Freeze for up to
    3 months; thaw in the refrigerator overnight before using.
\end{itemize}

\ingredients
\begin{itemize}
    \item 2 cups graham cracker crumbs, approximately 14 full graham crackers
    \item \nicefrac{1}{3} cup light brown sugar
    \item \nicefrac{1}{2} cup unsalted butter, melted
    \item Pinch salt
\end{itemize}

\instructions
\begin{enumerate}
    \item In a medium bowl, stir together the graham cracker crumbs, brown
    sugar, and salt, ensuring no lumps of brown sugar remain.

    \item Drizzle the melted butter over the graham cracker mixture and toss to
    combine with a fork, ensuring that the mixture is evenly moistened.

    \item Press the crust mixture evenly into the bottom and up the sides of a
    9-inch pie plate, 9-inch or 10-inch springform pan and pack tightly using
    the back of a measuring cup.
    \item If using in a no-bake recipe, place the pan in the freezer for at
    least 15 minutes while you prepare the filling and proceed with the recipe.
    \item If the graham cracker crust needs to be pre-baked, bake at 350 degrees
    for 10 minutes, then proceed with the recipe as written.
\end{enumerate}

\notes
\begin{itemize}
    \item \textbf{Equipment}: This recipe works for a standard 9-inch pie plate,
    8-inch or 9-inch square pan, or a 9-inch or 10-inch springform pan. You will
    need to double the recipe to use in a 9×13-inch pan.

    \item \textbf{Graham Crackers}: You can use any flavor or variety of graham
    cracker you like (and you can buy a box of graham cracker crumbs!), or you
    can also use crushed digestive biscuits, gingersnaps, vanilla wafers, or
    cornflakes. Whichever of these you use, just be sure to crush as finely as
    possible and measure out 2 cups of crumbs.

    \item \textbf{Brown Sugar}: You can substitute regular granulated sugar,
    coconut sugar, or muscovado.

    \item \textbf{Butter}: You can substitute vegan alternatives or coconut oil.
    For an extra oomph of flavor, go the extra step of making browned butter
    instead of simply melting the butter.

    \item \textbf{No-Bake Recipes}: If the crust and pie will not be baked,
    prepare the pie crust as directed, then place the pan in the freezer for at
    least 15 minutes before filling it.

    \item \textbf{Baked Recipes}: If the graham cracker crust needs to be
    pre-baked, bake at 350 degrees for 10 minutes, then proceed with the recipe
    as written.

    \item \textbf{To Make Ahead}: You can prepare the graham cracker crust up to
    2 days in advance of filling it. Simply cover it with plastic wrap and
    refrigerate it until ready to fill.

    \item \textbf{To Freeze}: Prepare the crust as instructed, then wrap the pie
    plate in plastic wrap and then in a layer of aluminum foil. Freeze for up to
    3 months; thaw in the refrigerator overnight before using.

\end{itemize}