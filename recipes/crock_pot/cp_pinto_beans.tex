\recipe{Pinto Beans}

The best crock pot pinto beans! No soaking, perfectly seasoned, and healthy.
Use this recipe anytime you need canned beans or mash them for refried beans.

Makes 6 cups.

\ingredients
\begin{itemize}
		\item 1 lb dry pinto beans -- 2 cups
		\item 2 teaspoons extra-virgin olive oil
		\item 1 small yellow onion - \textit{chopped into 1/4-inch dice}
		\item 1 jalapeno - \textit{cored, seeded, and finely chopped}
		\item 1.5 teaspoons kosher saltj
		\item 3 cloves garlic - \textit{minced}
		\item 2 bay leaves
		\item 1 teaspoon ground cumin
		\item 1 teaspoon dried oregano
		\item 1/8 to 1/4 teaspoon cayenne pepper - \textit{optional}
		\item 3 cups water
		\item 4 cups low-sodium chicken broth - \textit{or vegetable broth, divided}
		\item For serving: queso fresco or shredded Monterey jack cheese - \textit{diced tomatoes, diced red oion, chopped fresh cilantro, avocado (optional)}
\end{itemize}

\instructions
\begin{enumerate} 
		\item Place pinto beans into large colander. Thoroughly rinse them.
				Pick the beans over, removing ant damaged or clearly misshapen
				beans and discarding them. Transfer the rinsed beans to a
				6-quart or larger slow cooker. 

		\item Heat the oil in a medium nonstick skillet over a medium-high
				heat. Once the oil is hot, add the jalapeno, and 1/2 teaspoon
				salt. Saute for two minutes, then add the garlic, and let cook
				just until fragrant, about 30 seconds. Transfer to the slow
				cooker. Add the sauteed vegetables, bay leaves, cumin oregano,
				and remaining teaspoon of salt. Pout the broth and water over
				the top. 

		\item Cover and cook on HIGH for 8 to 10 hours, until the beans are
				tender. All slow cookers are different and can heat things
				differently, so if yours tends to run hot, check on it earlier.
				Depending upon your model, there may be some liquid still in
				the slow cooker. Discard the bay leaves. 

		\item \textsc{for regular pinto beans} (not refried): drain the liquid
				if you like, or leave the liquid in the crock pot and serve the
				beans with it, or use a slotted spoon for serving and drain the
				beans at the end prior to storing. Taste and adjust seasoning
				as desired. 

		\item \textsc{for refried beans}: reserve 1 cup of the cooking liquid,
				drain the beans, and return them to your slow cooker (if you
				don't have that much liquid in your slow cooker, you can use
				regular water instead). With a potato masher or a pastry
				cutter, mash the beans until they reach your desired
				consistency adding some of the reserved liquid as needed. (You
				can also scoop the beans into a blender in batches and puree
				them that way-- be sure to let the beans cool somewhat first so
				that they do not splatter). Taste and adjust the seasoning as
				desired. 
\end{enumerate}

\notes
\begin{itemize}
		\item Refrigerate leftover beans or freeze for up to three months. 
		\item For easy portions: let the beans cool \emph{completely}, then portion them into zip-top freezer bags labeled with the date. Seal the bags, removing as much air as possible, and squish the beans so that the bags lie flat. Freeze flat and remove from freezer as you need them. Let the beans thaw overnight in the refrigerator, then reheat gently on the stove with a splash of water or broth as needed to then them back out.
\end{itemize}
