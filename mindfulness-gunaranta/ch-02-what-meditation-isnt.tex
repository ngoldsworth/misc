\chapter{What Meditation Isn't}

Meditation is a word. You have heard this word before, or you would never have
picked up this book. The thinking process operates by association, and all sorts
of ideas are associated with the word `meditation'. Some of them are probably
accurate and others are hogwash. Some of them pertain more properly to other
systems of meditation and have nothing to do with Vipassana practice. Before we
proceed, it behooves us to blast some of the residue out of our own neuronal
circuits so that new information can pass unimpeded. Let us start with some of
the most obvious stuff.

We are not going to teach you to contemplate your navel or to chant secret
syllables. You are not conquering demons or harnessing invisible energies. There
are no colored belts given for your performance and you don't have to shave your
head or wear a turban. You don't even have to give away all your belongings and
move to a monastery. In fact, unless your life is immoral and chaotic, you can
probably get started right away and make some sort of progress. Sounds fairly
encouraging, wouldn't you say?  There are many, many books on the subject of
meditation. Most of them are written from the point of view which lies squarely
within one particular religious or philosophical tradition, and many of the
authors have not bothered to point this out. They make statements about
meditation which sound like general laws, but are actually highly specific
procedures exclusive to that particular system of practice. The result is
something of a muddle. Worse yet is the panoply of complex theories and
interpretations available, all of them at odds with one another. The result is a
real mess and an enormous jumble of conflicting opinions accompanied by a mass
of extraneous data. This book is specific. We are dealing exclusively with the
Vipassana system of meditation. We are going to teach you to watch the
functioning of your own mind in a calm and detached manner so you can gain
insight into your own behavior. The goal is awareness, an awareness so intense,
concentrated and finely tuned that you will be able to pierce the inner workings
of reality itself.

There are a number of common misconceptions about meditation. We see them crop
up again and again from new students, the same questions over and over. It is
best to deal with these things at once, because they are the sort of
preconceptions which can block your progress right from the outset. We are going
to take these misconceptions one at a time and explode them.

\paragraph*{Misconception 1: Meditation is just a relaxation technique.}

The bugaboo here is
the word `just'. Relaxation is a key component of meditation, but
Vipassana-style meditation aims at a much loftier goal. Nevertheless, the
statement is essentially true for many other systems of meditation. All
meditation procedures stress concentration of the mind, bringing the mind to
rest on one item or one area of thought. Do it strongly and thoroughly enough,
and you achieve a deep and blissful relaxation which is called Jhana. It is a
state of such supreme tranquility that it amounts to rapture. It is a form of
pleasure which lies above and beyond anything that can be experienced in the
normal state of consciousness.

Most systems stop right there. That is the goal, and when you attain that, you
simply repeat the experience for the rest of your life. Not so with Vipassana
meditation. Vipassana seeks another goal--awareness. Concentration and
relaxation are considered necessary concomitants to awareness. They are required
precursors, handy tools, and beneficial byproducts. But they are not the goal.
The goal is insight. Vipassana meditation is a profound religious practice aimed
at nothing less that the purification and transformation of your everyday life.
We will deal more thoroughly with the differences between concentration and
insight in Chapter 14.

\paragraph*{Misconception 2: Meditation means going into a trance.}

Here again the statement could be applied accurately to certain systems of meditation, but not to
Vipassana. Insight meditation is not a form of hypnosis. You are not trying to
black out your mind so as to become unconscious. You are not trying to turn
yourself into an emotionless vegetable. If anything, the reverse is true. You
will become more and more attuned to your own emotional changes. You will learn
to know yourself with ever- greater clarity and precision. In learning this
technique, certain states do occur which may appear trance-like to the observer.
But they are really quite the opposite. In hypnotic trance, the subject is
susceptible to control by another party, whereas in deep concentration the
meditator remains very much under his own control.

The similarity is superficial, and in any case the occurrence of these phenomena
is not the point of Vipassana. As we have said, the deep concentration of Jhana
is a tool or stepping stone on the route of heightened awareness. Vipassana by
definition is the cultivation of mindfulness or awareness. If you find that you
are becoming unconscious in meditation, then you aren't meditating, according to
the definition of the word as used in the Vipassana system. It is that simple.

\paragraph*{Misconception 3: Meditation is a mysterious practice which cannot be understood.}

Here again, this is almost true, but not quite. Meditation deals with levels of
consciousness which lie deeper than symbolic thought. Therefore, some of the
data about meditation just won't fit into words. That does not mean, however,
that it cannot be understood. There are deeper ways to understand things than
words. You understand how to walk. You probably can't describe the exact order
in which your nerve fibers and your muscles contract during that process. But
you can do it. Meditation needs to be understood that same way, by doing it. It
is not something that you can learn in abstract terms. It is to be experienced.
Meditation is not some mindless formula which gives automatic and predictable
results. You can never really predict exactly what will come up in any
particular session. It is an investigation and experiment and an adventure every
time. In fact, this is so true that when you do reach a feeling of
predictability and sameness in your practice, you use that as an indicator. It
means that you have gotten off the track somewhere and you are headed for
stagnation. Learning to look at each second as if it were the first and only
second in the universe is most essential in Vipassana meditation.

\paragraph*{Misconception 4: The purpose of meditation is to become a psychic superman.}

No, the purpose of meditation is to develop awareness. Learning to read minds is not
the point. Levitation is not the goal. The goal is liberation. There is a link
between psychic phenomena and meditation, but the relationship is somewhat
complex. During early stages of the meditator's career, such phenomena may or
may not arise. Some people may experience some intuitive understanding or
memories from past lives; others do not. In any case, these are not regarded as
well-developed and reliable psychic abilities. Nor should they be given undue
importance. Such phenomena are in fact fairly dangerous to new meditators in
that they are too seductive. They can be an ego trap which can lure you right
off the track. Your best advice is not to place any emphasis on these phenomena.
If they come up, that's fine. If they don't, that's fine, too. It's unlikely
that they will. There is a point in the meditator's career where he may practice
special exercises to develop psychic powers. But this occurs way down the line.
After he has gained a very deep stage of Jhana, the meditator will be far enough
advanced to work with such powers without the danger of their running out of
control or taking over his life. He will then develop them strictly for the
purpose of service to others. This state of affairs only occurs after decades of
practice. Don't worry about it. Just concentrate on developing more and more
awareness. If voices and visions pop up, just notice them and let them go. Don't
get involved.

\paragraph*{Misconception 5: Meditation is dangerous and a prudent person should avoid it.}

Everything is dangerous. Walk across the street and you may get hit by a bus.
Take a shower and you could break your neck.
Meditate and you will probably dredge up various nasty-matters from your past.
The suppressed material that has been buried there for quite some time can be
scary. It is also highly profitable. No activity is entirely without risk, but
that does not mean that we should wrap ourselves in some protective cocoon. That
is not living. That is premature death. The way to deal with danger is to know
approximately how much of it there is, where it is likely to be found and how to
deal with it when it arises. That is the purpose of this manual. Vipassana is
development of awareness. That in itself is not dangerous, but just the
opposite. Increased awareness is the safeguard against danger. Properly done,
meditation is a very gently and gradual process. Take it slow and easy, and
development of your practice will occur very naturally. Nothing should be
forced. Later, when you are under the close scrutiny and protective wisdom of a
competent teacher, you can accelerate your rate of growth by taking a period of
intensive meditation. In the beginning, though, easy does it. Work gently and
everything will be fine.

\paragraph*{Misconception 6: Meditation is for saints and holy men, not for regular people.}

You find this attitude very prevalent in Asia, where monks and holy men are
accorded an enormous amount of ritualized reverence.
This is somewhat akin to the American attitude of idealizing movie stars and
baseball heroes. Such people are stereotyped, made larger than life, and saddled
with all sort of characteristics that few human beings can ever live up to. Even
in the West, we share some of this attitude about meditation. We expect the
meditator to be some extraordinarily pious figure in whose mouth butter would
never dare to melt. A little personal contact with such people will quickly
dispel this illusion. They usually prove to be people of enormous energy and
gusto, people who live their lives with amazing vigor. It is true, of course,
that most holy men meditate, but they don't meditate because they are holy men.
That is backward. They are holy men because they meditate.
Meditation is how they got there. And they started meditating before they became
holy. This is an important point. A sizable number of students seems to feel
that a person should be completely moral before he begins meditation. It is an
unworkable strategy. Morality requires a certain degree of mental control. It's
a prerequisite. You can't follow any set of moral precepts without at least a
little self-control, and if your mind is perpetually spinning like a fruit
cylinder in a one- armed bandit, self-control is highly unlikely. So mental
culture has to come first.

There are three integral factors in Buddhist meditation ---\emph{morality},
\emph{concentration} and \emph{wisdom}. Those three factors grow together as your practice
deepens. Each one influences the other, so you cultivate the three of them
together, not one at a time. When you have the wisdom to truly understand a
situation, compassion towards all the parties involved is automatic, and
compassion means that you automatically restrain yourself from any thought, word
or deed that might harm yourself or others. Thus your behavior is automatically
moral. It is only when you don't understand things deeply that you create
problems. If you fail to see the consequences of your own action, you will
blunder. The fellow who waits to become totally moral before he begins to
meditate is waiting for a `but' that will never come. The ancient sages say that
he is like a man waiting for the ocean to become calm so that he can go take a
bath. To understand this relationship more fully, let us propose that there are
levels of morality. The lowest level is adherence to a set of rules and
regulations laid down by somebody else. It could be your favorite prophet. It
could be the state, the head man of your tribe or your father. No matter who
generates the rules, all you've got to do at this level is know the rules and
follow them. A robot can do that. Even a trained chimpanzee could do it if the
rules were simple enough and he was smacked with a stick every time he broke
one. This level requires no meditation at all. All you need are the rules and
somebody to swing the stick.

The next level of morality consists of obeying the same rules even in the
absence of somebody who will smack you. You obey because you have internalized
the rules. You smack yourself every time you break one. This level requires a
bit of mind control. If your thought pattern is chaotic, your behavior will be
chaotic, too. Mental culture reduces mental chaos.

There is a third level or morality, but it might be better termed ethics. This
level is a whole quantum layer up the scale, a real paradigm shift in
orientation. At the level of ethics, one does not follow hard and fast rules
dictated by authority. One chooses his own behavior according to the needs of
the situation. This level requires real intelligence and an ability to juggle
all the factors in every situation and arrive at a unique, creative and
appropriate response each time. Furthermore, the individual making these
decisions needs to have dug himself out of his own limited personal viewpoint.
He has to see the entire situation from an objective point of view, giving equal
weight to his own needs and those of others. In other words, he has to be free
from greed, hatred, envy and all the other selfish junk that ordinarily keeps us
from seeing the other guy's side of the issue. Only then can he choose that
precise set of actions which will be truly optimal for that situation. This
level of morality absolutely demands meditation, unless you were born a saint.
There is no other way to acquire the skill. Furthermore, the sorting process
required at this level is exhausting. If you tried to juggle all those factors
in every situation with your conscious mind, you'd wear yourself out. The
intellect just can't keep that many balls in the air at once. It is an overload.
Luckily, a deeper level of consciousness can do this sort of processing with
ease. Meditation can accomplish the sorting process for you. It is an eerie
feeling.

One day you've got a problem--say to handle Uncle Herman's latest divorce. It
looks absolutely unsolvable, and enormous muddle of `maybes' that would give
Solomon himself the willies. The next day you are washing the dishes, thinking
about something else entirely, and suddenly the solution is there. It just pops
out of the deep mind and you say, `Ah ha!' and the whole thing is solved.
This sort of intuition can only occur when you disengage the logic circuits from
the problem and give the deep mind the opportunity to cook up the solution. The
conscious mind just gets in the way. Meditation teaches you how to disentangle
yourself from the thought process. It is the mental art of stepping out of your
own way, and that's a pretty useful skill in everyday life. Meditation is
certainly not some irrelevant practice strictly for ascetics and hermits. It is
a practical skill that focuses on everyday events and has immediate application
in everybody's life. Meditation is not other- worldly.

Unfortunately, this very fact constitutes the drawback for certain students.
They enter the practice expecting instantaneous cosmic revelation, complete with
angelic choirs. What they usually get is a more efficient way to take out the
trash and better ways to deal with Uncle Herman. They are needlessly
disappointed. The trash solution comes first. The voices of archangels take a
bit longer.

\paragraph*{Misconception 7: Meditation is running away from reality.}
Incorrect. Meditation is running into reality. It does not insulate you from the pain of life. It
allows you to delve so deeply into life and all its aspects that you pierce the
pain barrier and you go beyond suffering. Vipassana is a practice done with the
specific intention of facing reality, to fully experience life just as it is and
to cope with exactly what you find. It allows you to blow aside the illusions
and to free yourself from all those polite little lies you tell yourself all the
time. What is there is there. You are who you are, and lying to yourself about
your own weaknesses and motivations only binds you tighter to the wheel of
illusion.  Vipassana meditation is not an attempt to forget yourself or to cover up your
troubles. It is learning to look at yourself exactly as you are. See what is
there, accept it fully. Only then can you change it.

\paragraph*{Misconception 8: Meditation is a great way to get high.} 
Well, yes and no.  Meditation does produce lovely blissful feelings sometimes.
But they are not the purpose, and they don't always occur. Furthermore, if you
do meditation with that purpose in mind, they are less likely to occur than if
you just meditate for the actual purpose of meditation, which is increased
awareness. Bliss results from relaxation, and relaxation results from release of
tension. Seeking bliss from meditation introduces tension into the process,
which blows the whole chain of events. It is a Catch-22. You can only have bliss
if you don't chase it. Besides, if euphoria and good feelings are what you are
after, there are easier ways to get them. They are available in taverns and from
shady characters on the street corners all across the nation.  Euphoria is not
the purpose of meditation. It will often arise, but it to be regarded as a by-
product. Still, it is a very pleasant side- effect, and it becomes more and more
frequent the longer you meditate. You won't hear any disagreement about this
from advanced practitioners.

\paragraph*{Misconception 9: Meditation is selfish.}
It certainly looks that way. There sits the meditator parked on his little
cushion. Is he out giving blood? No. Is he busy working with disaster victims?
No. But let us examine his motivation. Why is he doing this? His intention is to
purge his own mind of anger, prejudice and ill-will. He is actively engaged in
the process of getting rid of greed, tension and insensitivity. Those are the
very items which obstruct his compassion for others. Until they are gone, any
good works that he does are likely to be just an extension of his own ego and of
no real help in the long run. Harm in the name of help is one of the oldest
games. The grand inquisitor of the Spanish Inquisition spouts the loftiest of
motives. The Salem witchcraft trials were conducted for the public good.
Examine the personal lives of advanced meditators and you will often find them
engaged in humanitarian service. You will seldom find them as crusading
missionaries who are willing to sacrifice certain individuals for the sake of
some pious idea. The fact is we are more selfish than we know. The ego has a way
of turning the loftiest activities into trash if it is allowed free range.
Through meditation we become aware of ourselves exactly as we are, by waking up
to the numerous subtle ways that we manifest our own selfishness. Then we truly
begin to be genuinely selfless. Cleansing yourself of selfishness is not a
selfish activity.

\paragraph*{Misconception 10: When you meditate, you sit around thinking lofty thoughts.}
Wrong again. There are certain systems of contemplation in which this sort of
thing is done. But that is not Vipassana. Vipassana is the practice of
awareness. Awareness of whatever is there, be it supreme truth or crummy trash.
What is there is there. Of course, lofty aesthetic thoughts may arise during
your practice. They are certainly not to be avoided. Neither are they to be
sought. They are just pleasant side-effects. Vipassana is a simple practice. It
consists of experiencing your own life events directly, without preference and
without mental images pasted to them. Vipassana is seeing your life unfold from
moment to moment without biases. What comes up comes up. It is very simple.

\paragraph*{Misconception 11: A couple of weeks of meditation and all my problems will go away.} 
Sorry, meditation is not a quick cure-all. You will start seeing changes
right away, but really profound effects are years down the line. That is just
the way the universe is constructed. Nothing worthwhile is achieved overnight.
Meditation is tough in some respects. It requires a long discipline and
sometimes a painful process of practice. At each sitting you gain some results,
but those results are often very subtle. They occur deep within the mind, only
to manifest much later. and if you are sitting there constantly looking for some
huge instantaneous changes, you will miss the subtle shifts altogether. You will
get discouraged, give up and swear that no such changes will ever occur.
Patience is the key. Patience. If you learn nothing else from meditation, you
will learn patience. And that is the most valuable lesson available.
