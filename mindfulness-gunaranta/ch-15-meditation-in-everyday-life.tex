\chapter{Meditation In Everyday Life}

Every musician plays scales. When you
begin to study the piano, that's the first thing you learn, and you never stop
playing scales. The finest concert pianists in the world still play scales. It's
a basic skill that can't be allowed to get rusty.

Every baseball player practices batting. It's the first thing you learn in
Little League, and you never stop practicing. Every World Series game begins
with batting practice. Basic skills must always remain sharp.

Seated meditation is the arena in which the meditator practices his own
fundamental skills. The game the meditator is playing is the experience of his
own life, and the instrument upon which he plays is his own sensory apparatus.
Even the most seasoned meditator continues to practice seated meditation,
because it tunes and sharpens the basic mental skills he needs for his
particular game. We must never forget, however, that seated meditation itself is
not the game. It's the practice. The game in which those basic skills are to be
applied is the rest of one's experiential existence. Meditation that is not
applied to daily living is sterile and limited.

The purpose of Vipassana meditation is nothing less than the radical and
permanent transformation of your entire sensory and cognitive experience. It is
meant to revolutionize the whole of your life experience. Those periods of
seated practice are times set aside for instilling new mental habits. You learn
new ways to receive and understand sensation. You develop new methods of dealing
with conscious thought, and new modes of attending to the incessant rush of your
own emotions. These new mental behaviors must be made to carry over into the
rest of your life.

Otherwise, meditation remains dry and fruitless, a theoretical segment of your
existence that is unconnected to all the rest. Some effort to connect these two
segments is essential. A certain amount of carry-over will take place
spontaneously, but the process will be slow and unreliable. You are very likely
to be left with the feeling that you are getting nowhere and to drop the process
as unrewarding.

One of the most memorable events in your meditation career is the moment when
you first realize that you are meditation in the midst of some perfectly
ordinary activity. You are driving down the freeway or carrying out the trash
and it just turns on by itself.

This unplanned outpouring of the skills you have been so carefully fostering is
a genuine joy. It gives you a tiny window on the future. You catch a spontaneous
glimpse of what the practice really means. The possibility strikes you that this
transformation of consciousness could actually become a permanent feature of
your experience. You realize that you could actually spend the rest of your days
standing aside from the debilitating clamoring of your own obsessions, no longer
frantically hounded by your own needs and greed. You get a tiny taste of what it
is like to just stand aside and watch it all flow past. It's a magic moment.

That vision is liable to remain unfulfilled, however, unless you actively seek
to promote the carry-over process. The most important moment in meditation is
the instant you leave the cushion. When your practice session is over, you can
jump up and drop the whole thing, or you can bring those skills with you into
the rest of your activities.

It is crucial for you to understand what meditation is. It is not some special
posture, and it's not just a set of mental exercises.  Meditation is a
cultivation of mindfulness and the application of that mindfulness once
cultivated. You do not have to sit to meditate. You can meditate while washing
the dishes. You can meditate in the shower, or roller skating, or typing
letters.  Meditation is awareness, and it must be applied to each and every
activity of one's life. This isn't easy.

We specifically cultivate awareness through the seated posture in a quiet place
because that's the easiest situation in which to do so. Meditation in motion is
harder. Meditation in the midst of fast-paced noisy activity is harder still.
And meditation in the midst of intensely egoistic activities like romance or
arguments is the ultimate challenge. The beginner will have his hands full with
less stressful activities.

Yet the ultimate goal of practice remains: to build one's concentration and
awareness to a level of strength that will remain unwavering even in the midst
of the pressures of life in contemporary society. Life offers many challenges
and the serious meditator is very seldom bored.

Carrying your meditation into the events of your daily life is not a simple
process. Try it and you will see. That transition point between the end of your
meditation session and the beginning of `real life' is a long jump. It's too
long for most of us. We find our calm and concentration evaporating within
minutes, leaving us apparently no better off than before. In order to bridge
this gulf, Buddhists over the centuries have devised an array of exercises aimed
at smoothing the transition. They take that jump and break it down into little
steps. Each step can be practiced by itself.

\subsection*{1. Walking Meditation}

Our everyday existence is full of motion and activity.  Sitting utterly
motionless for hours on end is nearly the opposite of normal experience. Those
states of clarity and tranquility we foster in the midst of absolute stillness
tend to dissolve as soon as we move.

We need some transitional exercise that will teach us the skill of remaining
calm and aware in the midst of motion. Walking meditation helps us make that
transition from static repose to everyday life. It's meditation in motion, and
it is often used as an alternative to sitting. Walking is especially good for
those times when you are extremely restless. An hour of walking meditation will
often get you through that restless energy and still yield considerable
quantities of clarity. You can then go on to the seated meditation with greater
profit.

Standard Buddhist practice advocates frequent retreats to complement your daily
sitting practice. A retreat is a relatively long period of time devoted
exclusively to meditation. One or two day retreats are common for lay people.
Seasoned meditators in a monastic situation may spend months at a time doing
nothing else. Such practice is rigorous, and it makes sizable demands on both
mind and body. Unless you have been at it for several years, there is a limit to
how long you can sit and profit. Ten solid hours of the seated posture will
produce in most beginners a state of agony that far exceeds their concentration
powers. A profitable retreat must therefore be conducted with some change of
posture and some movement. The usual pattern is to intersperse blocks of sitting
with blocks of walking meditation. An hour of each with short breaks between is
common.

To do the walking meditation, you need a private place with enough space for at
least five to ten paces in a straight line. You are going to be walking back and
forth very slowly, and to the eyes of most Westerners, you'll look curious and
disconnected from everyday life. This is not the sort of exercise you want to
perform on the front lawn where you'll attract unnecessary attention.  Choose a
private place.

The physical directions are simple. Select an unobstructed area and start at one
end. Stand for a minute in an attentive position.  Your arms can be held in any
way that is comfortable, in front, in back, or at your sides. Then while
breathing in, lift the heel of one foot. While breathing out, rest that foot on
its toes. Again while breathing in, lift that foot, carry it forward and while
breathing out, bring the foot down and touch the floor.  Repeat this for the
other foot. Walk very slowly to the opposite end, stand for one minute, then
turn around very slowly, and stand there for another minute before you walk
back. Then repeat the process.  Keep you head up and you neck relaxed. Keep your
eyes open to maintain balance, but don't look at anything in particular. Walk
naturally. Maintain the slowest pace that is comfortable, and pay no attention
to your surroundings. Watch out for tensions building up in the body, and
release them as soon as you spot them.  Don't make any particular attempt to be
graceful. Don't try to look pretty. This is not an athletic exercise, or a
dance. It is an exercise in awareness. Your objective is to attain total
alertness, heightened sensitivity and a full, unblocked experience of the motion
of walking. Put all of your attention on the sensations coming from the feet and
legs. Try to register as much information as possible about each foot as it
moves. Dive into the pure sensation of walking, and notice every subtle nuance
of the movement. Feel each individual muscle as it moves. Experience every tiny
change in tactile sensation as the feet press against the floor and then lift
again.

Notice the way these apparently smooth motions are composed of complex series of
tiny jerks. Try to miss nothing. In order to heighten your sensitivity, you can
break the movement down into distinct components. Each foot goes through a lift,
a swing; and then a down tread. Each of these components has a beginning,
middle, and end. In order to tune yourself in to this series of motions, you can
start by making explicit mental notes of each stage.  Make a mental note of
``lifting, swinging, coming down, touching floor, pressing" and so on. This is a
training procedure to familiarize you with the sequence of motions and to make
sure that you don't miss any. As you become more aware of the myriad subtle
events going on, you won't have time for words. You will find yourself immersed
in a fluid, unbroken awareness of motion.  The feet will become your whole
universe. If your mind wanders, note the distraction in the usual way, then
return your attention to walking. Don't look at your feet while you are doing
all of this, and don't walk back and forth watching a mental picture of your
feet and legs. Don't think, just feel. You don't need the concept of feet and
you don't need pictures. Just register the sensations as they flow. In the
beginning, you will probably have some difficulties with balance. You are using
the leg muscles in a new way, and a learning period is natural. If frustration
arises, just note that and let it go.

The Vipassana walking technique is designed to flood your consciousness with
simple sensations, and to do it so thoroughly that all else is pushed aside.
There is no room for thought and no room for emotion. There is no time for
grasping, and none for freezing the activity into a series of concepts. There is
no need for a sense of self. There is only the sweep of tactile and kinesthetic
sensation, an endless and ever-changing flood of raw experience. We are learning
here to escape into reality, rather than from it.  Whatever insights we gain are
directly applicable to the rest of our notion-filled lives.

\subsection*{2. Postures}

The goal of our practice is to become fully aware of all facets of our
experience in an unbroken, moment-to-moment flow. Much of what we do and
experience is completely unconscious in the sense that we do it with little or
no attention. Our minds are on something else entirely. We spend most of our
time running on automatic pilot, lost in the fog of day-dreams and
preoccupations.

One of the most frequently ignored aspects of our existence is our body. The
technicolor cartoon show inside our head is so alluring that we tend to remove
all of our attention from the kinesthetic and tactile senses. That information
is pouring up the nerves and into the brain every second, but we have largely
sealed it off from consciousness. It pours into the lower levels of the mind and
it gets no further. Buddhists have developed an exercise to open the floodgates
and let this material through to consciousness. It's another way of making the
unconscious conscious.

Your body goes through all kinds of contortions in the course of a single day.
You sit and you stand. You walk and lie down.  You bend, run, crawl, and sprawl.
Meditation teachers urge you to become aware of this constantly ongoing dance.
As you go through your day, spend a few seconds every few minutes to check your
posture. Don't do it in a judgmental way. This is not an exercise to correct
your posture, or to improve you appearance. Sweep your attention down through
the body and feel how you are holding it. Make a silent mental note of `Walking'
or `Sitting' or `Lying down' or `Standing'. It all sounds absurdly simple, but
don't slight this procedure. This is a powerful exercise. If you do it
thoroughly, if you really instil this mental habit deeply, it can revolutionize
your experience. It taps you into a whole new dimension of sensation, and you
feel like a blind man whose sight has been restored.

\subsection*{3. Slow-Motion Activity}
Every action you perform is made up of separate
components. The simple action of tying your shoelaces is made up of a complex
series of subtle motions. Most of these details go unobserved. In order to
promote the overall habit of mindfulness, you can perform simple activities at
very low speed--making an effort to pay full attention to every nuance of the
act.

Sitting at a table and drinking a cup of tea is one example. There is much here
to be experienced. View your posture as you are sitting and feel the handle of
the cup between your fingers. Smell the aroma of the tea, notice the placement
of the cup, the tea, your arm, and the table. Watch the intention to raise the
arm arise within your mind, feel the arm as it raises, feel the cup against your
lips and the liquid pouring into your mouth. Taste the tea, then watch the
arising of the intention to lower your arm. The entire process is fascinating
and beautiful, if you attend to it fully, paying detached attention to every
sensation and to the flow of thought and emotion.

This same tactic can be applied to many of your daily activities. Intentionally
slowing down your thoughts, words and movements allows you to penetrate far more
deeply into them than you otherwise could. What you find there is utterly
astonishing. In the beginning, it is very difficult to keep this deliberately
slow pace during most regular activities, but skill grows with time. Profound
realizations occur during sitting meditation, but even more profound revelations
can take place when we really examine our own inner workings in the midst of
day-to-day activities. This is the laboratory where we really start to see the
mechanisms of our own emotions and the operations of our passions. Here is where
we can truly gauge the reliability of our reasoning, and glimpse the difference
between our true motives and the armor of pretense that we wear to fool
ourselves and others.

We will find a great deal of this information surprising, much of it disturbing,
but all of it useful. Bare attention brings order into the clutter that collects
in those untidy little hidden corners of the mind. As you achieve clear
comprehension in the midst of life's ordinary activities, you gain the ability
to remain rational and peaceful while you throw the penetrating light of
mindfulness into those irrational mental nooks and crannies. You start to see
the extent to which you are responsible for your own mental suffering.

You see your own miseries, fears, and tensions as self-generated. You see the
way you cause your own suffering, weakness, and limitations. And the more deeply
you understand these mental processes, the less hold they have on you.

\subsection*{4. Breath Coordination}

In seated meditation, our primary focus is the breath.  Total concentration on
the ever-changing breath brings us squarely into the present moment. The same
principle can be used in the midst of movement. You can coordinate the activity
in which you are involved with your breathing. This lends a flowing rhythm to
your movement, and it smooths out many of the abrupt transitions.  Activity
becomes easier to focus on, and mindfulness is increased. Your awareness thus
stays more easily in the present. Ideally, meditation should be a 24 hour-a-day
practice. This is a highly practical suggestion.

A state of mindfulness is a state of mental readiness. The mind is not burdened
with preoccupations or bound in worries.  Whatever comes up can be dealt with
instantly. When you are truly mindful, your nervous system has a freshness and
resiliency which fosters insight. A problem arises and you simply deal with it,
quickly, efficiently, and with a minimum of fuss. You don't stand there in a
dither, and you don't run off to a quiet corner so you can sit down and meditate
about it. You simply deal with it.  And in those rare circumstances when no
solution seems possible, you don't worry about that. You just go on to the next
thing that needs your attention. Your intuition becomes a very practical
faculty.

\subsection*{5. Stolen Moments}

The concept of wasted time does not exist for a serious meditator. Little dead
spaces during your day can be turned to profit. Every spare moment can be used
for meditation. Sitting anxiously in the dentist's office, meditate on your
anxiety. Feeling irritated while standing in a line at the bank, meditate on
irritation. Bored, twiddling you thumbs at the bus stop, meditate on boredom.
Try to stay alert and aware throughout the day. Be mindful of exactly what is
taking place right now, even if it is tedious drudgery. Take advantage of
moments when you are alone. Take advantage of activities that are largely
mechanical. Use every spare second to be mindful. Use all the moments you can.

\subsection*{6. Concentration On All Activities}
You should try to maintain mindfulness of
every activity and perception through the day, starting with the first
perception when you awake, and ending with the last thought before you fall
asleep. This is an incredibly tall goal to shoot for. Don't expect to be able to
achieve this work soon. Just take it slowly and let you abilities grow over
time. The most feasible way to go about the task is to divide your day up into
chunks. Dedicate a certain interval to mindfulness of posture, then extend this
mindfulness to other simple activities: eating, washing, dressing, and so forth.
Some time during the day, you can set aside 15 minutes or so to practice the
observation of specific types of mental states: pleasant, unpleasant, and
neutral feelings, for instance; or the hindrances, or thoughts. The specific
routine is up to you. The idea is to get practice at spotting the various items,
and to preserve your state of mindfulness as fully as you can throughout the
day.

Try to achieve a daily routine in which there is as little difference as
possible between seated meditation and the rest of your experience. Let the one
slide naturally into the other. Your body is almost never still. There is always
motion to observe. At the very least, there is breathing. Your mind never stops
chattering, except in the very deepest states of concentration. There is always
something coming up to observe. If you seriously apply your meditation, you will
never be at a loss for something worthy of your attention.

Your practice must be made to apply to your everyday living situation. That is
your laboratory. It provides the trials and challenges you need to make your
practice deep and genuine. It's the fire that purifies your practice of
deception and error, the acid test that shows you when you are getting somewhere
and when you are fooling yourself. If your meditation isn't helping you to cope
with everyday conflicts and struggles, then it is shallow. If your day-to-day
emotional reactions are not becoming clearer and easier to manage, then you are
wasting your time. And you never know how you are doing until you actually make
that test.

The practice of mindfulness is supposed to be a universal practice. You don't do
it sometimes and drop it the rest of the time. You do it all the time.
Meditation that is successful only when you are withdrawn in some soundproof
ivory tower is still undeveloped. Insight meditation is the practice of
moment-to-moment mindfulness. The meditator learns to pay bare attention to the
birth, growth, and decay of all the phenomena of the mind. He turns from none of
it, and he lets none of it escape. Thoughts and emotions, activities and
desires, the whole show. He watches it all and he watches it continuously. It
matters not whether it is lovely or horrid, beautiful or shameful. He sees the
way it is and the way it changes. No aspect of experience is excluded or
avoided. It is a very thoroughgoing procedure.

If you are moving through your daily activities and you find yourself in a state
of boredom, then meditate on your boredom. Find out how it feels, how it works,
and what it is composed of. If you are angry, meditate on the anger. Explore the
mechanics of anger. Don't run from it. If you find yourself sitting in the grip
of a dark depression, meditate on the depression. Investigate depression in a
detached and inquiring way. Don't flee from it blindly. Explore the maze and
chart its pathways. That way you will be better able to cope with the next
depression that comes along.

Meditating your way through the ups and downs of daily life is the whole point
of Vipassana. This kind of practice is extremely rigorous and demanding, but it
engenders a state of mental flexibility that is beyond comparison. A meditator
keeps his mind open every second. He is constantly investigating life,
inspecting his own experience, viewing existence in a detached and inquisitive
way. Thus he is constantly open to truth in any form, from any source, and at
any time. This is the state of mind you need for Liberation.

It is said that one may attain enlightenment at any moment if the mind is kept
in a state of meditative readiness. The tiniest, most ordinary perception can be
the stimulus: a view of the moon, the cry of a bird, the sound of the wind in
the trees. it's not so important what is perceived as the way in which you
attend to that perception. The state of open readiness is essential. It could
happen to you right now if you are ready. The tactile sensation of this book in
your fingers could be the cue. The sound of these words in your head might be
enough. You could attain enlightenment right now, if you are ready.