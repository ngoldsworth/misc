\chapter{Attitude}
Within the last century, Western science and physics have
made a startling discovery. We are part of the world we view. The very process
of our observation changes the things we observe. As an example, an electron is
an extremely tiny item. It cannot be viewed without instrumentation, and that
apparatus dictates what the observer will see. If you look at an electron in one
way, it appears to be a particle, a hard little ball that bounces around in nice
straight paths. When you view it another way, an electron appears to be a wave
form, with nothing solid about it. It glows and wiggles all over the place. An
electron is an event more than a thing. And the observer participates in that
event by the very process of his or her observation. There is no way to avoid
this interaction.

Eastern science has recognized this basic principle for a very long time. The
mind is a set of events, and the observer participates in those events every
time he or she looks inward. Meditation is participatory observation. What you
are looking at responds to the process of looking. What you are looking at is
you, and what you see depends on how you look. Thus the process of meditation is
extremely delicate, and the result depends absolutely on the state of mind of
the meditator. The following attitudes are essential to success in practice.
Most of them have been presented before. But we bring them together again here
as a series of rules for application.

\begin{enumerate}
\item \emph{Don't expect anything.} Just sit back and see what happens. Treat the whole
thing as an experiment. Take an active interest in the test itself. But don't
get distracted by your expectations about results. For that matter, don't be
anxious for any result whatsoever. Let the meditation move along at its own
speed and in its own direction. Let the meditation teach you what it wants you
to learn. Meditative awareness seeks to see reality exactly as it is. Whether
that corresponds to our expectations or not, it requires a temporary suspension
of all our preconceptions and ideas. We must store away our images, opinions and
interpretations someplace out of the way for the duration. Otherwise we will
stumble over them.

\item \emph{Don't strain:} Don't force anything or make grand exaggerated efforts.
Meditation is not aggressive. There is no violent striving. Just let your effort
be relaxed and steady.

\item \emph{Don't rush:} There is no hurry, so take you time. Settle yourself on a cushion
and sit as though you have a whole day.
Anything really valuable takes time to develop. Patience, patience, patience.

\item \emph{Don't cling to anything and don't reject anything:} Let come what comes and
accommodate yourself to that, whatever it is. If good mental images arise, that
is fine. If bad mental images arise, that is fine, too. Look on all of it as
equal and make yourself comfortable with whatever happens. Don't fight with what
you experience, just observe it all mindfully.

\item \emph{Let go}: Learn to flow with all the changes that come up. Loosen up and relax.

\item \emph{Accept everything that arises}: Accept your feelings, even the ones
you wish you did not have. Accept your experiences, even the ones you hate.
Don't condemn yourself for having human flaws and failings. Learn to see all the
phenomena in the mind as being perfectly natural and understandable. Try to
exercise a disinterested acceptance at all times and with respect to everything
you experience.

\item \emph{Be gentle with yourself}: Be kind to yourself. You may not be
perfect, but you are all you've got to work with. The process of becoming who
you will be begins first with the total acceptance of who you are.

\item \emph{Investigate yourself}: Question everything. Take nothing for granted. Don't
believe anything because it sounds wise and pious and some holy men said it. See
for yourself. That does not mean that you should be cynical, impudent or
irreverent. It means you should be empirical. Subject all statements to the
actual test of your experience and let the results be your guide to truth.
Insight meditation evolves out of an inner longing to wake up to what is real
and to gain liberating insight to the true structure of existence. The entire
practice hinges upon this desire to be awake to the truth.
Without it, the practice is superficial.

\item \emph{View all problems as challenges}: Look upon negatives that arise as
opportunities to learn and to grow. Don't run from them, condemn yourself or
bear your burden in saintly silence. You have a problem? Great. More grist for
the mill.  Rejoice, dive in and investigate.

\item \emph{Don't ponder}: You don't need to figure everything out. Discursive
thinking won't free you from the trap. In mediation, the mind is purified
naturally by mindfulness, by wordless bare attention. Habitual deliberation is
not necessary to eliminate those things that are keeping you in bondage. All
that is necessary is a clear, non-conceptual perception of what they are and how
they work. That alone is sufficient to dissolve them. Concepts and reasoning
just get in the way. Don't think. See.

\item \emph{Don't dwell upon contrasts}: Differences do exist between people, but
dwelling upon then is a dangerous process. Unless carefully handled, it leads
directly to egotism. Ordinary human thinking is full of greed, jealousy and
pride. A man seeing another man on the street may immediately think, ``He is
better looking than I am." The instant result is envy or shame.
A girl seeing another girl may think, ``I am prettier than she is." The instant
result is pride. This sort of comparison is a mental habit, and it leads
directly to ill feeling of one sort or another: greed, envy, pride, jealousy,
hatred. It is an unskillful mental state, but we do it all the time. We compare
our looks with others, our success, our accomplishments, our wealth,
possessions, or I.Q. and all these lead to the same place--estrangement,
barriers between people, and ill feeling.

\end{enumerate}

The meditator's job is to cancel this unskillful habit by examining it
thoroughly, and then replacing it with another. Rather than noticing the
differences between self and others, the meditator trains himself to notice
similarities. He centers his attention on those factors that are universal to
all life, things that will move him closer to others. Thus his comparison, if
any, leads to feelings of kinship rather than feelings of estrangement.

Breathing is a universal process. All vertebrates breathe in essentially the
same manner. All living things exchange gasses with their environment in some
way or other. This is one of the reasons that breathing is chosen as the focus
of meditation. the meditator is advised to explore the process of his own
breathing as a vehicle for realizing his own inherent connectedness with the
rest of life. This does not mean that we shut our eyes to all the differences
around us. Differences exist. It means simply that we de- emphasize contrasts
and emphasize the universal factors. The recommended procedure is as follows:
When the meditator perceives any sensory object, he is not to dwell upon it in
the ordinary egotistical way. He should rather examine the very process of
perception itself. He should watch the feelings that arise and the mental
activities that follow. He should note the changes that occur in his own
consciousness as a result. In watching all these phenomena, the meditator must
be aware of the universality of what he is seeing. That initial perception will
spark pleasant, unpleasant or neutral feelings. That is a universal phenomenon.
It occurs in the mind of others just as it does in his, and he should see that
clearly. Following these feelings various reactions may arise. He may feel
greed, lust, or jealousy. He may feel fear, worry, restlessness or boredom.
These reactions are universal. He simple notes them and then generalizes. He
should realize that these reactions are normal human responses and can arise in
anybody.

The practice of this style of comparison may feel forced and artificial at
first, but it is no less natural than what we ordinarily do. It is merely
unfamiliar. With practice, this habit pattern replaces our normal habit of
egoistic comparing and feels far more natural in the long run. We become very
understanding people as a result. we no longer get upset by the failings of
others. We progress toward harmony with all life.
