\chapter{Introduction: American Buddhism}

The subject of this book is Vipassana meditation practice. Repeat, practice.
This is a meditation manual, a nuts-and-bolts, step- by-step guide to Insight
meditation. It is meant to be practical. It is meant for use.

There are already many comprehensive books on Buddhism as a philosophy, and on
the theoretical aspects of Buddhist meditation. If you are interested in that
material we urge you to read those books. Many of them are excellent. This book
is a `How to.' It is written for those who actually want to meditate and
especially for those who want to start now. There are very few qualified
teachers of the Buddhist style of meditation in the United States of America. It
is our intention to give you the basic data you need to get off to a flying
start. Only those who follow the instructions given here can say whether we have
succeeded or failed. Only those who actually meditate regularly and diligently
can judge our effort. No book can possibly cover every problem that a meditator
may run into. You will need to meet a qualified teacher eventually. In the mean
time, however, these are the basic ground rules; a full understanding of these
pages will take you a very long way.

There are many styles of meditation. Every major religious tradition has some
sort of procedure which they call meditation, and the word is often very loosely
used. Please understand that this volume deals exclusively with the Vipassana
style of meditation as taught and practiced in South and Southeast Asian
Buddhism. It is often translated as Insight meditation, since the purpose of
this system is to give the meditator insight into the nature of reality and
accurate understanding of how everything works.

Buddhism as a whole is quite different from the theological religions with which
Westerners are most familiar. It is a direct entrance to a spiritual or divine
realm without addressing deities or other `agents'. Its flavor is intensely
clinical, much more akin to what we would call psychology than to what we would
usually call religion. It is an ever-ongoing investigation of reality, a
microscopic examination of the very process of perception. Its intention is to
pick apart the screen of lies and delusions through which we normally view the
world, and thus to reveal the face of ultimate reality. Vipassana meditation is
an ancient and elegant technique for doing just that.

Theravada Buddhism presents us with an effective system for exploring the deeper
levels of the mind, down to the very root of consciousness itself. It also
offers a considerable system of reverence and rituals in which those techniques
are contained. This beautiful tradition is the natural result of its 2,500-year
development within the highly traditional cultures of South and Southeast Asia.

In this volume, we will make every effort to separate the ornamental and the
fundamental and to present only the naked plain truth itself. Those readers who
are of a ritual bent may investigate the Theravada practice in other books, and
will find there a vast wealth of customs and ceremony, a rich tradition full of
beauty and significance. Those of a more clinical bent may use just the
techniques themselves, applying them within whichever philosophical and
emotional context they wish. The practice is the thing.

The distinction between Vipassana meditation and other styles of meditation is
crucial and needs to be fully understood.  Buddhism addresses two major types of
meditation. They are different mental skills, modes of functioning or qualities
of consciousness. In Pali, the original language of Theravada literature, they
are called `Vipassana' and `Samatha'.

'Vipassana' can be translated as `insight', a clear awareness of exactly what is
happening as it happens. `Samatha' can be translated as `concentration' or
`tranquility'. It is a state in which the mind is brought to rest, focused only
on one item and not allowed to wander. When this is done, a deep calm pervades
body and mind, a state of tranquility which must be experienced to be
understood. Most systems of meditation emphasize the Samatha component. The
meditator focuses his mind upon some items, such as prayer, a certain type of
box, a chant, a candle flame, a religious image or whatever, and excludes all
other thoughts and perceptions from his consciousness. The result is a state of
rapture which lasts until the meditator ends the session of sitting. It is
beautiful, delightful meaningful and alluring, but only temporary. Vipassana
meditation address the other component, insight.

The Vipassana meditator uses his concentration as a tool by which his awareness
can chip away at the wall of illusion which cuts him off from the living light
of reality. It is a gradual process of ever-increasing awareness and into the
inner workings of reality itself. It takes years, but one day the meditator
chisels through that wall and tumbles into the presence of light. The
transformation is complete. It's called liberation, and it's permanent.
Liberation is the goal of all buddhist systems of practice. But the routes to
attainment of the end are quite diverse.

There are an enormous number of distinct sects within Buddhism. But they divide
into two broad streams of thought -- Mahayana and Theravada. Mahayana Buddhism
prevails throughout East Asia, shaping the cultures of China, Korea, Japan,
Nepal, Tibet and Vietnam. The most widely known of the Mahayana systems is Zen,
practiced mainly in Japan, Korea, Vietnam and the United States. The Theravada
system of practice prevails in South and Southeast Asia in the countries of Sri
Lanka, Thailand, Burma, Laos and Cambodia. This book deals with Theravada
practice.

The traditional Theravada literature describes the techniques of both Samatha
(concentration and tranquility of mind) and Vipassana (insight or clear
awareness). There are forty different subjects of meditation described in the
Pali literature. They are recommended as objects of concentration and as
subjects of investigation leading to insight. But this is a basic manual, and we
limit our discussion to the most fundamental of those recommended
objects--breathing. This book is an introduction to the attainment of
mindfulness through bare attention to, and clear comprehension of, the whole
process of breathing. Using the breath as his primary focus of attention, the
meditator applies participatory observation to the intirety of his own
perceptual universe. He learns to watch changes occurring in all physical
experiences, in feelings and in perceptions. He learns to study his own mental
activities and the fluctuations in the character of consciousness itself. All of
these changes are occurring perpetually and are present in every moment of our
experiences.

Meditation is a living activity, an inherently experiential activity. It cannot
be taught as a purely scholastic subject. The living heart of the process must
come from the teacher's own personal experience. Nevertheless, there is a vast
fund of codified material on the subject which is the product of some of the
most intelligent and deeply illumined human beings ever to walk the earth. This
literature is worthy of attention. Most of the points given in this book are
drawn from the Tipitaka, which is the three-section collected work in which the
Buddah's original teachings have been preserved. The Tipitaka is comprised of
the Vinaya, the code of discipline for monks, nuns, and lay people; the Suttas,
public discourses attributed to the Buddha; and the Abhidhamma, a set of deep
psycho-philosophical teachings.

In the first century after Christ, an eminent Buddhist scholar named Upatissa
wrote the Vimuttimagga, (The Path of Freedom) in which he summarized the
Buddha's teachings on meditation. In the fifth century A.C. (after Christ,)
another great Buddhist scholar named Buddhaghosa covered the same ground in a
second scholastic thesis--the Visuddhimagga, (The Path of Purification) which is
the standard text on meditation even today. Modern meditation teachers rely on
the Tipitaka and upon their own personal experiences. It is our intention to
present you with the clearest and most concise directions for Vipassana
meditation available in the English language. But this book offers you a foot in
the door. It's up to you to take the first few steps on the road to the
discovery of who you are and what it all means. It is a journey worth taking. We
wish you success.

