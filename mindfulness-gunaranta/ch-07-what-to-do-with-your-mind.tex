\chapter{What To Do With Your Mind} The meditation we teach is called \textbf{Insight
Meditation}. As we have already said, the variety of possible objects of
meditation is nearly unlimited, and human beings have used an enormous number
down through the ages. Even within the Vipassana tradition there are variances.
There are meditation teachers who teach their students to follow the breath by
watching the rise and fall of the abdomen. Others recommend focusing attention
on the touch of the body against the cushion, or hand against hand, or the
feeling of one leg against the other. The method we are explaining here,
however, is considered the most traditional and is probably what Gotama Buddha
taught his students. The Satipatthana Sutta, the Buddha's original discourse on
mindfulness, specifically says that one must begin by focusing the attention on
the breathing and then go on to note all other physical and mental phenomena
which arise.

We sit, watching the air going in and out of our noses. At first glance, this
seems an exceedingly odd and useless procedure.  Before going on to specific
instructions, let us examine the reason behind it. The first question we might
address is why use any focus of attention at all? We are, after all, trying to
develop awareness. Why not just sit down and be aware of whatever happens to be
present in the mind? In fact there are meditations of that nature. They are
sometimes referred to as unstructured meditation and they are quite difficult.
The mind is tricky. Thought is an inherently complicated procedure. By that we
mean we become trapped, wrapped up, and stuck in the thought chain. One thought
leads to another which leads to another, and another, and another, and so on.
Fifteen minutes later we suddenly wake up and realize we spent that whole time
stuck in a daydream or sexual fantasy or a set of worries about our bills or
whatever.

There is a difference between being aware of a thought and thinking a thought.
That difference is very subtle. It is primarily a matter of feeling or texture.
A thought you are simply aware of with bare attention feels light in texture;
there is a sense of distance between that thought and the awareness viewing it.
It arises lightly like a bubble, and it passes away without necessarily giving
rise to the next thought in that chain. Normal conscious thought is much heavier
in texture. It is ponderous, commanding, and compulsive. It sucks you in and
grabs control of consciousness. By its very nature it is obsessional, and it
leads straight to the next thought in the chain, apparently with no gap between
them.

Conscious thought sets up a corresponding tension in the body, such as muscular
contraction or a quickening of the heartbeat. But you won't feel tension until
it grows to actual pain, because normal conscious thought is also greedy. It
grabs all your attention and leaves none to notice its own effect. The
difference between being aware of the thought and thinking the thought is very
real.

But it is extremely subtle and difficult to see. Concentration is one of the
tools needed to be able to see this difference.  Deep concentration has the
effect of slowing down the thought process and speeding up the awareness viewing
it. The result is the enhanced ability to examine the thought process.
Concentration is our microscope for viewing subtle internal states. We use the
focus of attention to achieve one-pointedness of mind with calm and constantly
applied attention. Without a fixed reference point you get lost, overcome by the
ceaseless waves of change flowing round and round within the mind.

We use breath as our focus. It serves as that vital reference point from which
the mind wanders and is drawn back. Distraction cannot be seen as distraction
unless there is some central focus to be distracted from. That is the frame of
reference against which we can view the incessant changes and interruptions that
go on all the time as a part of normal thinking.

Ancient Pali texts liken meditation to the process of taming a wild elephant.
The procedure in those days was to tie a newly captured animal to a post with a
good strong rope. When you do this the elephant is not happy. He screams and
tramples, and pulls against the rope for days. Finally it sinks through his
skull that he can't get away, and he settles down. At this point you can begin
to feed him and to handle him with some measure of safety. Eventually you can
dispense with the rope and post altogether, and train your elephant for various
tasks. Now you've got a tamed elephant that can be put to useful work. In this
analogy the wild elephant is your wildly active mind, the rope is mindfulness,
and the post is our object of meditation-- breathing.

The tamed elephant who emerges from this process is a well trained, concentrated
mind that can then be used for the exceedingly tough job of piercing the layers
of illusion that obscure reality. Meditation tames the mind.

The next question we need to address is: Why choose breathing as the primary
object of meditation? Why not something a bit more interesting? Answers to this
are numerous. A useful object of meditation should be one that promotes
mindfulness. It should be portable, easily available and cheap. It should also
be something that will not embroil us in those states of mind from which we are
trying to free ourselves, such as greed, anger and delusion. Breathing satisfies
all these criteria and more. Breathing is something common to every human being.
We all carry it with us wherever we go. It is always there, constantly
available, never ceasing from birth till death, and it costs nothing.

Breathing is a non-conceptual process, a thing that can be experienced directly
without a need for thought. Furthermore, it is a very living process, an aspect
of life that is in constant change. The breath moves in cycles--inhalation,
exhalation, breathing in and breathing out. Thus it is miniature model of life
itself.

The sensation of breath is subtle, yet it is quite distinct when you learn to
tune into it. It takes a bit of an effort to find it. Yet anybody can do it.
You've got to work at it, but not too hard. For all these reasons, breathing
makes an ideal object of meditation.

Breathing is normally an involuntary process, proceeding at its own pace without
a conscious will. Yet a single act of will can slow it down or speed it up. Make
it long and smooth or short and choppy. The balance between involuntary
breathing and forced manipulation of breath is quite delicate. And there are
lessons to be learned here on the nature of will and desire. Then, too, that
point at the tip of the nostril can be viewed as a sort of a window between the
inner and outer worlds. It is a nexus point and energy-transfer spot where stuff
from the outside world moves in and becomes a part of what we call `me', and
where a part of me flows forth to merge with the outside world. There are
lessons to be learned here about self- concept and how we form it.

Breath is a phenomenon common to all living things. A true experiential
understanding of the process moves you closer to other living beings. It shows
you your inherent connectedness with all of life. Finally, breathing is a
present-time process. By that we mean it is always occurring in the
here-and-now. We don't normally live in the present, of course. We spend most of
our time caught up in memories of the past or leaping ahead to the future, full
of worries and plans. The breath has none of that `other-timeness'. When we
truly observe the breath, we are automatically placed in the present. We are
pulled out of the morass of mental images and into a bare experience of the
here- and-now. In this sense, breath is a living slice of reality. A mindful
observation of such a miniature model of life itself leads to insight that are
broadly applicable to the rest of our experience.

The first step in using the breath as an object of meditation is to find it.
What you are looking for is the physical, tactile sensation of the air that
passes in and out of the nostrils. This is usually just inside the tip of the
nose. But the exact spot varies from one person to another, depending on the
shape of the nose. To find your own point, take a quick deep breath and notice
the point just inside the nose or on the upper lip where you have the most
distinct sensation of passing air. Now exhale and notice the sensation at the
same point. It is from this point that you will follow the whole passage of
breath. Once you have located your own breath point with clarity, don't deviate
from that spot. Use this single point in order to keep your attention fixed.
Without having selected such a point, you will find yourself moving in and out
of the nose, going up and down the windpipe, eternally chasing after the breath
which you can never catch because it keeps changing, moving and flowing.

If you ever sawed wood you already know the trick. As a carpenter, you don't
stand there watching the saw blade going up and down. You will get dizzy. You
fix your attention on the spot where the teeth of the blade dig into the wood.
It is the only way you can saw a straight line. As a meditator, you focus your
attention on that single spot of sensation inside the nose. From this vantage
point, you watch the entire movement of breath with clear and collected
attention. Make no attempt to control the breath.

This is not a breathing exercise of the sort done in Yoga. Focus on the natural
and spontaneous movement of the breath. Don't try to regulate it or emphasize it
in any way. Most beginners have some trouble in this area. In order to help
themselves focus on the sensation, they unconsciously accentuate their
breathing. The results is a forced and unnatural effort that actually inhibits
concentration rather than helping it. Don't increase the depth of your breath or
its sound. This latter point is especially important in group meditation. Loud
breathing can be a real annoyance to those around you. Just let the breath move
naturally, as if you were asleep. Let go and allow the process to go along at
its own rhythm.

This sounds easy, but it is trickier than you think. Do not be discouraged if
you find your own will getting in the way. Just use that as an opportunity to
observe the nature of conscious intention. Watch the delicate interrelation
between the breath, the impulse to control the breath and the impulse to cease
controlling the breath. You may find it frustrating for a while, but it is
highly profitable as a learning experience, and it is a passing phase.
Eventually, the breathing process will move along under its own steam. And you
will feel no impulse to manipulate it. At this point you will have learned a
major lesson about your own compulsive need to control the universe.

Breathing, which seems so mundane and uninteresting at first glance, is actually
an enormously complex and fascinating procedure. It is full of delicate
variations, if you look. There is inhalation and exhalation, long breath and
short breath, deep breath, shallow breath, smooth breath and ragged breath.
These categories combine with one another in subtle and intricate ways.

Observe the breath closely. Really study it. You find enormous variations and
constant cycle of repeated patterns. It is like a symphony. Don't observe just
the bare outline of the breath. There is more to see here than just an in-breath
and an out-breath.

Every breath has a beginning middle and end. Every inhalation goes through a
process of birth, growth and death and every exhalation does the same. The depth
and speed of your breathing changes according to your emotional state, the
thought that flows through your mind and the sounds you hear. Study these
phenomena. You will find them fascinating.

This does not mean, however, that you should be sitting there having little
conversations with yourself inside your head: ``There is a short ragged breath
and there is a deep long one. I wonder what's next?" No, that is not Vipassana.
That is thinking. You will find this sort of thing happening, especially in the
beginning. This too is a passing phase. Simply note the phenomenon and return
your attention toward the observation of the sensation of breath. Mental
distractions will happen again. But return your attention to your breath again,
and again, and again, and again, for as long as it takes until it does not
happen anymore.

When you first begin this procedure, expect to face some difficulties. Your mind
will wander off constantly, darting around like a drunken bumblebee and zooming
off on wild tangents. Try not to worry. The monkey-minded phenomenon is well
known. It is something that every advanced meditator has had to deal with. They
have pushed through it one way or another, and so can you.

When it happens, just not the fact that you have been thinking, day-dreaming,
worrying, or whatever. Gently, but firmly, without getting upset or judging
yourself for straying, simply return to the simple physical sensation of the
breath. Then do it again the next time, and again, an again, and again.

Somewhere in this process, you will come face-to-face with the sudden and
shocking realization that you are completely crazy.  Your mind is a shrieking,
gibbering madhouse on wheels barreling pell-mell down the hill, utterly out of
control and hopeless.

No problem. You are not crazier than you were yesterday. It has always been this
way, and you just never noticed. You are also no crazier than everybody else
around you. The only real difference is that you have confronted the situation;
they have not. So they still feel relatively comfortable. That does not mean
that they are better off. Ignorance may be bliss, but it does not lead to
liberation. So don't let this realization unsettle you. It is a milestone
actually, a sigh of real progress. The very fact that you have looked at the
problem straight in the eye means that you are on your way up and out of it.

In the wordless observation of the breath, there are two states to be avoided:
thinking and sinking. The thinking mind manifests most clearly as the
monkey-mind phenomenon we have just been discussing. The sinking mind is almost
the reverse. As a general term, sinking mind denotes any dimming of awareness.
At its best, it is sort of a mental vacuum in which there is no thought, no
observation of the breath, no awareness of anything. It is a gap, a formless
mental gray area rather like a dreamless sleep. Sinking mind is a void. Avoid
it.

Vipassana meditation is an active function. Concentration is a strong, energetic
attention to one single item. Awareness is a bright clean alertness. Samahdhi
and Sati--these are the two faculties we wish to cultivate. And sinking mind
contains neither. At its worst, it will put you to sleep. Even at its best it
will simply waste your time.

When you find you have fallen into a state of sinking mind, just note the fact
and return your attention to the sensation of breathing. Observe the tactile
sensation of the in-breath. Feel the touch sensation of the out-breath. Breathe
in, breathe out and watch what happens. When you have been doing that for some
time--perhaps weeks or months--you will begin to sense the touch as a physical
object. Simply continue the process--breathe in and breathe out. Watch what
happens. As your concentration deepens you will have less and less trouble with
monkey-mind. Your breathing will slow down and you will track it more and more
clearly, with fewer and fewer interruptions. You begin to experience a state of
great calm in which you enjoy complete freedom from those things we call psychic
irritants. No greed, lust, envy, jealousy or hatred. Agitation goes away. Fear
flees. These are beautiful, clear, blissful states of mind. They are temporary,
and they will end when meditation ends. Yet even these brief experiences will
change your life. This is not liberation, but these are stepping stones on the
path that leads in that direction. Do not, however, expect instant bliss. Even
these stepping stones take time and effort and patience.

The meditation experience is not a competition. There is a definite goal. But
there is no timetable. What you are doing is digging your way deeper and deeper
through the layers of illusion toward realization of the supreme truth of
existence. The process itself is fascinating and fulfilling. It can be enjoyed
for its own sake. There is no need to rush.

At the end of a well-done meditation session you will feel a delightful
freshness of mind. It is peaceful, buoyant, and joyous energy which you can then
apply to the problems of daily living. This in itself is reward enough. The
purpose of meditation is not to deal with problems, however, and problem-
solving ability is a fringe benefit and should be regarded as such. If you place
too much emphasis on the problem-solving aspect, you will find your attention
turning to those problems during the session sidetracking concentration. Don't
think about your problems during your practice. Push them aside very gently.

Take a break from all that worrying and planning. Let your meditation be a
complete vacation. Trust yourself, trust your own ability to deal with these
issues later, using the energy and freshness of mind that you built up during
your meditation. Trust yourself this way and it will actually occur.

Don't set goals for yourself that are too high to reach. Be gently with
yourself. You are trying to follow your own breathing continuously and without a
break. That sounds easy enough, so you will have a tendency at the outset to
push yourself to be scrupulous and exacting. This is unrealistic. Take time in
small units instead. At the beginning of an inhalation, make the resolve to
follow the breath just for the period of that one inhalation. Even this is not
so easy, but at least it can be done. Then, at the start of the exhalation,
resolve to follow the breath just for that one exhalation, all the way through.
You will still fail repeatedly, but keep at it.

Every time you stumble, start over. Take it one breath at a time. This is the
level of the game where you can actually win. Stick at it--fresh resolve with
every breath cycle, tiny units of time. Observe each breath with care and
precision, taking it one split second on top of another, with fresh resolve
piled one on top of the other. In this way, continuous and unbroken awareness
will eventually result.

Mindfulness of breathing is a present-time awareness. When you are doing it
properly, you are aware only of what is occurring in the present. You don't look
back and you don't look forward. You forget about the last breath, and you don't
anticipate the next one. When the inhalation is just beginning, you don't look
ahead to the end of that inhalation. You don't skip forward to the exhalation
which is to follow. You stay right there with what is actually taking place. The
inhalation is beginning, and that's what you pay attention to; that and nothing
else.

This meditation is a process of retraining the mind. The state you are aiming
for is one in which you are totally aware of everything that is happening in
your own perceptual universe, exactly the way it happens, exactly when it is
happening; total, unbroken awareness in the present time. This is an incredibly
high goal, and not to be reached all at once. It takes practice, so we start
small. We start by becoming totally aware of one small unit of time, just one
single inhalation. And, when you succeed, you are on your way to a whole new
experience of life.