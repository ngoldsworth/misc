 \chapter{What Meditation Is} 

Meditation is a word, and words are used in different ways by different
speakers. This may seem like a trivial point, but it is not.

It is quite important to distinguish exactly what a particular speaker means by
the words he uses. Every culture on earth, for example, has produced some sort
of mental practice which might be termed meditation. It all depends on how loose
a definition you give to that word. Everybody does it, from Africans to Eskimos.
The techniques are enormously varied, and we will make no attempt to survey
them. There are other books for that. For the purpose of this volume, we will
restrict our discussion to those practices best known to Western audiences and
most likely associated with the term meditation.

Within the Judeo-Christian tradition we find two overlapping practices called
prayer and contemplation. Prayer is a direct address to some spiritual entity.
Contemplation in a prolonged period of conscious thought about some specific
topic, usually a religious ideal or scriptural passage. From the standpoint of
mental culture, both of these activities are exercises in concentration. The
normal deluge of conscious thought is restricted, and the mind is brought to one
conscious area of operation. The results are those you find in any concentrative
practice: deep calm, a physiological slowing of the metabolism and a sense of
peace and well-being.

Out of the Hindu tradition comes Yogic meditation, which is also purely
concentrative. The traditional basic exercises consist of focusing the mind on a
single object a stone, a candle flame, a syllable or whatever, and not allowing
it to wander. Having acquired the basic skill, the Yogi proceeds to expand his
practice by taking on more complex objects of meditation chants, colorful
religious images, energy channels in the body and so forth. Still, no matter how
complex the object of meditation, the meditation itself remains purely an
exercise in concentration.

Within the Buddhist tradition, concentration is also highly valued. But a new
element is added and more highly stressed. That element is awareness. All
Buddhist meditation aims at the development of awareness, using concentration as
a tool. The Buddhist tradition is very wide, however, and there are several
diverse routes to this goal. Zen meditation uses two separate tacks. The first
is the direct plunge into awareness by sheer force of will. You sit down and you
just sit, meaning that you toss out of your mind everything except pure
awareness of sitting. This sounds very simple. It is not. A brief trial will
demonstrate just how difficult it really is. The second Zen approach used in the
Rinzai school is that of tricking the mind out of conscious thought and into
pure awareness. This is done by giving the student an unsolvable riddle which he
must solve anyway, and by placing him in a horrendous training situation. Since
he cannot flee from the pain of the situation, he must flee into a pure
experience of the moment. There is nowhere else to go. Zen is tough. It is
effective for many people, but it is really tough.

Another stratagem, Tantric Buddhism, is nearly the reverse. Conscious thought,
at least the way we usually do it, is the manifestation of ego, the you that you
usually think that you are. Conscious thought is tightly connected with
self-concept. The self-concept or ego is nothing more than a set of reactions
and mental images which are artificially pasted to the flowing process of pure
awareness. Tantra seeks to obtain pure awareness by destroying this ego image.
This is accomplished by a process of visualization. The student is given a
particular religious image to meditate upon, for example, one of the deities
from the Tantric pantheon. He does this in so thorough a fashion that he becomes
that entity. He takes off his own identity and puts on another.

This takes a while, as you might imagine, but it works. During the process, he
is able to watch the way that the ego is constructed and put in place. He comes
to recognize the arbitrary nature of all egos, including his own, and he escapes
from bondage to the ego. He is left in a state where he may have an ego if he so
chooses, either his own or whichever other he might wish, or he can do without
one. Result: pure awareness. Tantra is not exactly a game of patty cake either.

Vipassana is the oldest of Buddhist meditation practices. The method comes
directly from the Sitipatthana Sutta, a discourse attributed to Buddha himself.
Vipassana is a direct and gradual cultivation of mindfulness or awareness. It
proceeds piece by piece over a period of years. The student's attention is
carefully directed to an intense examination of certain aspects of his own
existence. The meditator is trained to notice more and more of his own flowing
life experience. Vipassana is a gentle technique.

But it also is very, very thorough. It is an ancient and codified system of
sensitivity training, a set of exercises dedicated to becoming more and more
receptive to your own life experience. It is attentive listening, total seeing
and careful testing. We learn to smell acutely, to touch fully and really pay
attention to what we feel.  We learn to listen to our own thoughts without
being caught up in them.

The object of Vipassana practice is to learn to pay attention. We think we are
doing this already, but that is an illusion. It comes from the fact that we are
paying so little attention to the ongoing surge of our own life experiences that
we might just as well be asleep. We are simply not paying enough attention to
notice that we are not paying attention. It is another Catch-22.

Through the process of mindfulness, we slowly become aware of what we really are
down below the ego image. We wake up to what life really is. It is not just a
parade of ups and downs, lollipops and smacks on the wrist. That is an illusion.
Life has a much deeper texture than that if we bother to look, and if we look in
the right way.

Vipassana is a form of mental training that will teach you to experience the
world in an entirely new way. You will learn for the first time what is truly
happening to you, around you and within you. It is a process of self discovery,
a participatory investigation in which you observe your own experiences while
participating in them, and as they occur. The practice must be approached with
this attitude.

``Never mind what I have been taught. Forget about theories and prejudgments and
stereotypes. I want to understand the true nature of life. I want to know what
this experience of being alive really is. I want to apprehend the true and
deepest qualities of life, and I don't want to just accept somebody else's
explanation. I want to see it for myself." If you pursue your meditation
practice with this attitude, you will succeed. You'll find yourself observing
things objectively, exactly as they are--flowing and changing from moment to
moment. Life then takes on an unbelievable richness which cannot be described.
It has to be experienced.

The Pali term for Insight meditation is Vipassana Bhavana. Bhavana comes from
the root `Bhu', which means to grow or to become. There fore Bhavana means to
cultivate, and the word is always used in reference to the mind. Bhavana means
mental cultivation. `Vipassana' is derived from two roots. `Passana' means
seeing or perceiving. `Vi' is a prefix with the complex set of connotations. The
basic meaning is `in a special way.' But there also is the connotation of both
`into' and `through'. The whole meaning of the word is looking into something
with clarity and precision, seeing each component as distinct and separate, and
piercing all the way through so as to perceive the most fundamental reality of
that thing. This process leads to insight into the basic reality of whatever is
being inspected. Put it all together and `Vipassana Bhavana' means the
cultivation of the mind, aimed at seeing in a special way that leads to insight
and to full understanding.

In Vipassana mediation we cultivate this special way of seeing life. We train
ourselves to see reality exactly as it is, and we call this special mode of
perception `mindfulness.' This process of mindfulness is really quite different
from what we usually do. We usually do not look into what is really there in
front of us. We see life through a screen of thoughts and concepts, and we
mistake those mental objects for the reality. We get so caught up in this
endless thought stream that reality flows by unnoticed. We spend our time
engrossed in activity, caught up in an eternal pursuit of pleasure and
gratification and an eternal flight from pain and unpleasantness. We spend all
of our energies trying to make ourselves feel better, trying to bury our fears.
We are endlessly seeking security. Meanwhile, the world of real experience flows
by untouched and untasted. In Vipassana meditation we train ourselves to ignore
the constant impulses to be more comfortable, and we dive into the reality
instead. The ironic thing is that real peace comes only when you stop chasing
it. Another Catch-22.

When you relax your driving desire for comfort, real fulfillment arises. When
you drop your hectic pursuit of gratification, the real beauty of life comes
out. When you seek to know the reality without illusion, complete with all its
pain and danger, that is when real freedom and security are yours. This is not
some doctrine we are trying to drill into you. This is an observable reality, a
thing you can and should see for yourself.

Buddhism is 2500 years old, and any thought system of that vintage has time to
develop layers and layers of doctrine and ritual.  Nevertheless, the fundamental
attitude of Buddhism is intensely empirical and anti-authoritarian. Gotama the
Buddha was a highly unorthodox individual and real anti-traditionalist. He did
not offer his teaching as a set of dogmas, but rather as a set of propositions
for each individual to investigate for himself. His invitation to one and all
was `Come and See'. One of the things he said to his followers was ``Place no
head above your own". By this he meant, don't accept somebody else's word. See
for yourself.

We want you to apply this attitude to every word you read in this manual. We are
not making statements that you would accept merely because we are authorities in
the field. Blind faith has nothing to do with this. These are experiential
realities. Learn to adjust your mode of perception according to instructions
given in the book, and you will see for yourself. That and only that provides
ground for your faith. Insight meditation is essentially a practice of
investigative personal discovery.

Having said this, we will present here a very short synopsis of some of the key
points of Buddhist philosophy. We make not attempt to be thorough, since that
has been quite nicely done in many other books. This material is essential to
understanding Vipassana, therefore, some mention must be made.

From the Buddhist point of view, we human beings live in a very peculiar
fashion. We view impermanent things as permanent, though everything is changing
all around us. The process of change is constant and eternal. As you read these
words, your body is aging. But you pay no attention to that. The book in you
hand is decaying. The print is fading and the pages are becoming brittle. The
walls around you are aging. The molecules within those walls are vibrating at an
enormous rate, and everything is shifting, going to pieces and dissolving
slowly. You pay no attention to that, either. Then one day you look around you.
Your body is wrinkled and squeaky and you hurt. The book is a yellowed, useless
lump; the building is caving in. So you pine for lost youth and you cry when the possessions
are gone. Where does this pain come from? It comes from your own inattention.
You failed to look closely at life. You failed to observe the constantly
shifting flow of the world as it went by. You set up a collection of mental
constructions, `me', `the book', `the building', and you assume that they would
endure forever. They never do. But you can tune into the constantly ongoing
change. You can learn to perceive your life as an ever- flowing movement, a
thing of great beauty like a dance or symphony. You can learn to take joy in the
perpetual passing away of all phenomena. You can learn to live with the flow of
existence rather than running perpetually against the grain. You can learn this.
It is just a matter of time and training.

Our human perceptual habits are remarkably stupid in some ways. We tune out 99%
of all the sensory stimuli we actually receive, and we solidify the remainder
into discrete mental objects. Then we react to those mental objects in
programmed habitual ways.

An example: There you are, sitting alone in the stillness of a peaceful night. A
dog barks in the distance. The perception itself is indescribably beautiful if
you bother to examine it. Up out of that sea of silence come surging waves of
sonic vibration. You start to hear the lovely complex patterns, and they are
turned into scintillating electronic stimulations within the nervous system. The
process is beautiful and fulfilling in itself. We humans tend to ignore it
totally. Instead, we solidify that perception into a mental object. We paste a
mental picture on it and we launch into a series of emotional and conceptual
reactions to it. ``There is that dog again. He is always barking at night. What a
nuisance. Every night he is a real bother. Somebody should do something. Maybe I
should call a cop. No, a dog catcher. So, I'll call the pound. No, maybe I'll
just write a real nasty letter to the guy who owns that dog. No, too much
trouble. I'll just get an ear plug." They are just perceptual and mental habits.
You learn to respond this way as a child by copying the perceptual habits of
those around you. These perceptual responses are not inherent in the structure
of the nervous system. The circuits are there. But this is not the only way that
our mental machinery can be used. That which has been learned can be unlearned.
The first step is to realize what you are doing, as you are doing it, and stand
back and quietly watch.

From the Buddhist perspective, we humans have a backward view of life. We look
at what is actually the cause of suffering and we see it as happiness. The cause
of suffering is that desire- aversion syndrome which we spoke of earlier. Up
pops a perception. It could be anything--a beautiful girl, a handsome guy, speed
boat, thug with a gun, truck bearing down on you, anything. Whatever it is, the
very next thing we do is to react to the stimulus with a feeling about it.

Take worry. We worry a lot. Worry itself is the problem. Worry is a process. It
has steps. Anxiety is not just a state of existence but a procedure. What you've
got to do is to look at the very beginning of that procedure, those initial
stages before the process has built up a head of steam. The very first link of
the worry chain is the grasping/rejecting reaction. As soon as some phenomenon
pops into the mind, we try mentally to grab onto it or push it away. That sets
the worry response in motion.

Luckily, there is a handy little tool called Vipassana meditation which you can
use to short-circuit the whole mechanism.  Vipassana meditation teaches us how
to scrutinize our own perceptual process with great precision. We learn to watch
the arising of thought and perception with a feeling of serene detachment. We
learn to view our own reactions to stimuli with calm and clarity. We begin to
see ourselves reacting without getting caught up in the reactions themselves.
The obsessive nature of thought slowly dies. We can still get married. We can
still step out of the path of the truck. But we don't need to go through hell
over either one.

This escape from the obsessive nature of thought produces a whole new view of
reality. It is a complete paradigm shift, a total change in the perceptual
mechanism. It brings with it the feeling of peace and rightness, a new zest for
living and a sense of completeness to every activity. Because of these
advantages, Buddhism views this way of looking at things as a correct view of
life and Buddhist texts call it seeing things as they really are.

Vipassana meditation is a set of training procedures which open us gradually to
this new view of reality as it truly is. Along with this new reality goes a new
view of the most central aspect of reality: `me'. A close inspection reveals
that we have done the same thing to `me' that we have done to all other
perceptions. We have taken a flowing vortex of thought, feeling and sensation
and we have solidified that into a mental construct. Then we have stuck a label
onto it, `me'. And forever after, we threat it as if it were a static and
enduring entity. We view it as a thing separate from all other things. We pinch
ourselves off from the rest of that process of eternal change which is the
universe. And than we grieve over how lonely we feel. We ignore our inherent
connectedness to all other beings and we decide that `I' have to get more for
'me'; then we marvel at how greedy and insensitive human beings are. And on it
goes. Every evil deed, every example of heartlessness in the world stems
directly from this false sense of `me' as distinct from all else that is out
there.

Explode the illusion of that one concept and your whole universe changes. Don't
expect to do this overnight, though. You spent your whole life building up that
concept, reinforcing it with every thought, word, and deed over all those years.
It is not going to evaporate instantly. But it will pass if you give it enough
time and enough attention. Vipassana meditation is a process by which it is
dissolved. Little by little, you chip away at it just by watching it.

The `I' concept is a process. It is a thing we are doing. In Vipassana we learn
to see that we are doing it, when we are doing it and how we are doing it. Then
it moves and fades away, like a cloud passing through the clear sky. We are left
in a state where we can do it or not do it, whichever seems appropriate to the
situation. The compulsiveness is gone. We have a choice.

These are all major insights, of course. Each one is a deep- reaching
understanding of one of the fundamental issues of human existence. They do not
occur quickly, nor without considerable effort. But the payoff is big. They lead
to a total transformation of your life. Every second of your existence
thereafter is changed. The meditator who pushes all the way down this track
achieves perfect mental health, a pure love for all that lives and complete
cessation of suffering. That is not small goal. But you don't have to go all the
way to reap benefits. They start right away and they pile up over the years. It
is a cumulative function. The more you sit, the more you learn about the real nature of your won
existence. The more hours you spend in meditation, the greater your ability to
calmly observe every impulse and intention, every thought and emotion just as it
arises in the mind. Your progress to liberation is measured in cushion-man
hours. And you can stop any time you've had enough. There is no stick over your
head except your own desire to see the true quality of life, to enhance your own
existence and that of others.

Vipassana meditation is inherently experiential. It is not theoretical. In the
practice of mediation you become sensitive to the actual experience of living,
to how things feel. You do not sit around developing subtle and aesthetic
thoughts about living. You live. Vipassana meditation more than anything else is
learning to live.