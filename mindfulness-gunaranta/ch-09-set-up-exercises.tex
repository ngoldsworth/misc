\chapter{Set Up Exercises} 
In Theravada Buddhist countries, it is traditional to begin each meditation
session with the recitation of a certain set of formulas.  An American audience
is likely to take one glance at these invocations and to dismiss them as
harmless rituals and nothing more.  The so-called rituals, however, have been
devised and refined by a set of pragmatic and dedicated men and women, and they
have a thoroughly practical purpose. They are therefore worthy of deeper
inspection.

The Buddha was considered contrary in his own day. He was born into an intensely
over-ritualized society, and his ideas appeared thoroughly iconoclastic to the
established hierarchy of his own era. On numerous occasions, he disavowed the
use of rituals for their own sake, and he was quite adamant about it. This does
not mean that ritual has no use. It means that ritual by itself, performed
strictly for it's own sake, will not get you out of the trap. If you believe
that mere recitation of words will save you, then you only increase your own
dependence on words and concepts. This moves you away from the wordless
perception of reality rather than toward it. Therefore, the formulae which
follow must be practiced with a clear understanding of what they are and why
they work. They are not magical incantations. They are psychological cleansing
devices which require active mental participation in order to be effective.
Mumbled words without intention are useless. Vipassana meditation is a delicate
psychological activity, and the mental set of the practitioner is crucial to its
success. The technique works best in an atmosphere of calm, benevolent
confidence. And these recitations have been designed to foster those attitudes.
Correctly used, they can act as a helpful tool on the path to liberation.

\subsection*{The Threefold Guidance }
Meditation is a tough job. It is an inherently solitary
activity. One person battles against enormously powerful forces, part of the
very structure of the mind doing the meditating. When you really get into it,
you will eventually find yourself confronted with a shocking realization. One
day you will look inside and realize the full enormity of what you are actually
up against. What you are struggling to pierce looks like a solid wall so tightly
knit that not a single ray of light shines through. You find yourself sitting
there, staring at this edifice and you say to yourself, ``That? I am supposed to
get past that? But it's impossible! That is all there is. That is the whole
world. That is what everything means, and that is what I use to define myself
and to understand everything around me, and if I take that away the whole world
will fall apart and I will die. I cannot get through that. I just can't." It is
a very scary feeling, a very lonely feeling. You feel like, ``Here I am, all
alone, trying to punch away something so huge it is beyond conception." To
counteract this feeling, it is useful to know that you are not alone. Others
have passed this way before.

They have confronted that same barrier, and they have pushed their way through
to the light. They have laid out the rules by which the job can be done, and
they have banded together into a brotherhood for mutual encouragement and
support. The \textbf{Buddha} found his way through this very same wall, and after him
came many others. He left clear instructions in the form of the \textbf{Dhamma} to guide
us along the same path. And he founded the Sangha, the brotherhood of monks to
preserve that path and to keep each other on it. You are not alone, and the
situation is not hopeless.

Meditation takes energy. You need courage to confront some pretty difficult
mental phenomena and the determination to sit through various unpleasant mental
states. Laziness just will not serve. In order to pump up your energy for the
job, repeat the following statements to yourself. Feel the intention you put
into them. Mean what you say.

``I am about to tread the very path that has been walked by the Buddha and by his
great and holy disciples. An indolent person cannot follow that path. May my
energy prevail. May I succeed." 

\subsection*{Universal Loving-Kindness}

Vipassana meditation
is an exercise in mindfulness, egoless awareness. It is a procedure in which the
ego will be eradicated by the penetrating gaze of mindfulness. The practitioner
begins this process with the ego in full command of mind and body. Then, as
mindfulness watches the ego function, it penetrates to the roots of the
mechanics of ego and extinguishes ego piece by piece. There is a full blown
Catch-22 in all this, however. Mindfulness is egoless awareness. If we start
with ego in full control, how do we put enough mindfulness there at the
beginning to get the job started? There is always some mindfulness present in
any moment. The real problem is to gather enough of it to be effective. To do
this we can use a clever tactic. We can weaken those aspects of ego which do the
most harm, so that mindfulness will have less resistance to overcome.

Greed and hatred are the prime manifestations of the ego process. To the extent
that grasping and rejecting are present in the mind, mindfulness will have a
very rough time. The results of this are easy to see. If you sit down to
meditate while you are in the grip of some strong obsessive attachment, you will
find that you will get nowhere. If you are all hung up in your latest scheme to
make more money, you probably will spend most of your meditation period doing
nothing but thinking about it. If you are in a black fury over some recent
insult, that will occupy your mind just as fully. There is only so much time in
one day, and your meditation minutes are precious. It is best not to waste them.
The Theravada tradition has developed a useful tool which will allow you to
remove these barriers from your mind at least temporarily, so that you can get
on with the job of removing their roots permanently.

You can use one idea to cancel another. You can balance a negative emotion by
instilling a positive one. Giving is the opposite of greed. Benevolence is the
opposite of hatred. Understand clearly now: This is not an attempt to liberate
yourself by autohypnosis.

You cannot condition Enlightenment. Nibbana is an unconditioned state. A
liberated person will indeed be generous and benevolent, but not because he has
been conditioned to be so. He will be so purely as a manifestation of his own
basic nature, which is no longer inhibited by ego. So this is not conditioning.
This is rather psychological medicine. If you take this medicine according to
directions, it will bring temporary relief from the symptoms of the malady from
which you are currently suffering.

Then you can get to work in earnest on the illness itself.  You start out by
banishing thoughts of self-hatred and self- condemnation. You allow good
feelings and good wishes first to flow to yourself, which is relatively easy.
Then you do the same for those people closest to you. Gradually, you work
outward from your own circle of intimates until you can direct a flow of those
same emotions to your enemies and to all living beings everywhere. Correctly
done, this can be a powerful and transformative exercise in itself.

At the beginning of each meditation session, say the following sentences to
yourself. Really feel the intention: 

\begin{enumerate}
\item May I be well, happy and peaceful. May
no harm come to me. May no difficulties come to me. May no problems come to me.
May I always meet with success.

\item May I also have patience, courage, understanding, and determination to meet
and overcome inevitable difficulties, problems, and failures in life.

\item May my parents be well, happy and peaceful. May no harm come to them. May
no difficulties come to them. May no problems come to them. May they always meet
with success. May they also have patience, courage, understanding, and
determination to meet and overcome inevitable difficulties, problems, and
failures in life.

\item May my teachers be well, happy and peaceful. May no harm come to them. May no
difficulties come to them. May no problems come to them. May they always meet
with success. May they also have patience, courage, understanding, and
determination to meet and overcome inevitable difficulties, problems, and
failures in life.

\item May my relatives be well, happy and peaceful. May no harm come to them. May
no difficulties come to them. May no problems come to them. May they always meet
with success. May they also have patience, courage, understanding, and
determination to meet and overcome inevitable difficulties, problems, and
failures in life.

\item May my friends be well, happy and peaceful. May no harm come to them. May no
difficulties come to them. May no problems come to them. May they always meet
with success. May they also have patience, courage, understanding, and
determination to meet and overcome inevitable difficulties, problems, and
failures in life.

\item May all indifferent persons be well, happy and peaceful. May no harm come to
them. May no difficulties come to them. May no problems come to them. May they
always meet with success. May they also have patience, courage, understanding,
and determination to meet and overcome inevitable difficulties, problems, and
failures in life.

\item May my enemies be well, happy and peaceful. May no harm come to them. May no
difficulties come to them. May no problems come to them. May they always meet
with success. May they also have patience, courage, understanding, and
determination to meet and overcome inevitable difficulties, problems, and
failures in life.

\item May all living beings be well, happy and peaceful. May no harm come to
them.  May no difficulties come to them.  May no problems come to them. May they
always meet with success. May they also have patience, courage, understanding,
and determination to meet and overcome inevitable difficulties, problems, and
failures in life.

\end{enumerate}

Once you have completed these recitations, lay aside all your troubles and
conflicts for the period of practice. Just drop the whole bundle. If they come
back into your meditation later, just treat them as what they are, distractions.

The practice of Universal Loving-Kindness is also recommended for bedtime and
just after arising. It is said to help you sleep well and to prevent nightmares.
It also makes it easier to get up in the morning. And it makes you more friendly
and open toward everybody, friend or foe, human or otherwise.

The most damaging psychic irritant arising in the mind particularly at the time
when the mind is quiet, is resentment. You may experience indignation
remembering some incident that caused you psychological and physical pain. This
experience can cause you uneasiness, tension, agitation and worry. You might not
be able to go on sitting and experiencing this state of mind.

Therefore, we strongly recommend that you should start your meditation with
generating Universal Loving-Kindness.  You sometimes may wonder how can we wish:
``May my enemies be well, happy and peaceful. May no harm come to them; may no
difficulty come to them; may no problems come to them; may they always meet with
success. May they also have patience, courage, understanding and determination
to meet and overcome inevitable difficulties, problems and failures in life"?
You must remember that you practice loving-kindness for the purification of your
own mind, just as you practice meditation for your own attainment of peace and
liberation from pain and suffering. As you practice loving-kindness within
yourself, you can behave in a most friendly manner without biases, prejudices,
discrimination or hate. Your noble behavior helps you to help others in a most
practical manner to reduce their pain and suffering. It is compassionate people
who can help others. Compassion is a manifestation of loving-kindness in action,
for one who does not have loving-kindness cannot help others. Noble behavior
means behaving in a most friendly and most cordial manner. Behavior includes
your thought speech and action. If this triple mode of expression of your
behavior is contradictory, your behavior cannot be noble behavior. On the other
hand, pragmatically speaking, it is much better to cultivate the noble thought,
``May all beings be happy minded" than the thought, ``I hate him". Our noble
thought will one day express itself in noble behavior and our spiteful thought
in evil behavior.

Remember that your thoughts are transformed into speech and action in order to
bring the expected result. Thought translated into action is capable of
producing tangible result. You should always speak and do things with
mindfulness of loving-kindness. While speaking of loving-kindness, if you act or
speak in a diametrically opposite way you will be reproached by the wise. As
mindfulness of loving-kindness develops, your thoughts, words and deeds should
be gently, pleasant, meaningful, truthful and beneficial to you as well as to
others. If your thoughts, words or deeds cause harm to you, to others or to
both, then you must ask yourself whether you are really mindful of
loving-kindness.

For all practical purposes, if all of your enemies are well, happy and peaceful,
they would not be your enemies. If they are free from problems, pain, suffering,
affliction, neurosis, psychosis, paranoia, fear, tension, anxiety, etc., they
would not be your enemies.

Your practical solution to your enemies is to help them to overcome their
problems, so you can live in peace and happiness. In fact, if you can, you
should fill the minds of all your enemies with loving-kindness and make all of
them realize the true meaning of peace, so you can live in peace and happiness.
The more they are in neurosis, psychosis, fear, tension, anxiety, etc., the more
trouble, pain and suffering they can bring to the world. If you could convert a
vicious and wicked person into a holy and saintly individual, you would perform
a miracle. Let us cultivate adequate wisdom and loving- kindness within
ourselves to convert evil minds to saintly minds.

When you hate somebody you think, ``Let him be ugly. Let him lie in pain. Let him
have no prosperity. Let him not be right.  Let him not be famous. Let him have
no friends Let him, after death, reappear in an unhappy state of depravation in
a bad destination in perdition." However, what actually happens is that your own
body generates such harmful chemistry that you experience pain, increased heart
beat, tension, change of facial expression, loss of appetite for food,
deprivation of sleep and appear very unpleasant to others. You go through the
same things you wish for your enemy.  Also you cannot see the truth as it is.
Your mind is like boiling water. Or you are like a patient suffering from
jaundice to whom any delicious food tastes bland. Similarly, you cannot
appreciate somebody's appearance, achievement, success, etc. As long as this
condition exists, you cannot meditate well.

Therefore we recommend very strongly that you practice loving- kindness before
you start your serious practice of meditation.  Repeat the proceeding passages
very mindfully and meaningfully. As you recite these passages, feel true
loving-kindness within yourself first and then share it with others, for you
cannot share with others what you do not have within yourself.

Remember, though, these are not magic formulas. They don't work by themselves.
If you use them as such, you will simply waste time and energy. But if you truly
participate in these statements and invest them with your own energy, they will
serve you will. Give them a try. See for yourself.
