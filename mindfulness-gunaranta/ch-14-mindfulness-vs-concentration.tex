\chapter{Mindfulness Versus Concentration} 

Vipassana meditation is something of a mental balancing act. You are going to be
cultivating two separate qualities of the mind --mindfulness and concentration.
Ideally these two work together as a team. They pull in tandem, so to speak.
Therefore it is important to cultivate them side-by-side and in a balanced
manner. If one of the factors is strengthened at the expense of the other, the
balance of the mind is lost and meditation impossible.

Concentration and mindfulness are distinctly different functions. They each have
their role to play in meditation, and the relationship between them is definite
and delicate. Concentration is often called one-pointedness of mind. It consists
of forcing the mind to remain on one static point. Please note the word FORCE.
Concentration is pretty much a forced type of activity. It can be developed by
force, by sheer unremitting willpower. And once developed, it retains some of
that forced flavor. Mindfulness, on the other hand, is a delicate function
leading to refined sensibilities. These two are partners in the job of
meditation. Mindfulness is the sensitive one. He notices things. Concentration
provides the power. He keeps the attention pinned down to one item. Ideally,
mindfulness is in this relationship. Mindfulness picks the objects of attention,
and notices when the attention has gone astray.  Concentration does the actual
work of holding the attention steady on that chosen object. If either of these
partners is weak, your meditation goes astray.

Concentration could be defined as that faculty of the mind which focuses single
mindedly on one object without interruption. It must be emphasized that true
concentration is a wholesome one-pointedness of mind. That is, the state is free
from greed, hatred and delusion. Unwholesome one-pointedness is also possible,
but it will not lead to liberation. You can be very single-minded in a state of
lust. But that gets you nowhere. Uninterrupted focus on something that you hate
does not help yo at all. In fact, such unwholesome concentration is fairly
short-lived even when it is achieved--especially when it is used to harm others.
True concentration itself is free from such contaminants. It is a state in which
the mind is gathered together and thus gains power and intensity. We might use
the analogy of a lens. Parallel waves of sunlight falling on a piece of paper
will do no more than warm the surface. But the same amount of light, when
focused through a lens, falls on a single point and the paper bursts into
flames.  Concentration is the lens. It produces the burning intensity necessary
to see into the deeper reaches of the mind. Mindfulness selects the object that
the lens will focus on and looks through the lens to see what is there.

Concentration should be regarded as a tool. Like any tool, it can be used for
good or for ill. A sharp knife can be used to create a beautiful carving or to
harm someone. It is all up to the one who uses the knife. Concentration is
similar. Properly used, it can assist you towards liberation. But it can also be
used in the service of the ego. It can operate in the framework of achievement
and competition. You can use concentration to dominate others. You can use it to
be selfish. The real problem is that concentration alone will not give you a
perspective on yourself. It won't throw light on the basic problems of
selfishness and the nature of suffering. It can be used to dig down into deep
psychological states. But even then, the forces of egotism won't be understood.
Only mindfulness can do that. If mindfulness is not there to look into the lens
and see what has been uncovered, then it is all for nothing. Only mindfulness
understands. Only mindfulness brings wisdom. Concentration has other
limitations, too.

Really deep concentration can only take place under certain specific conditions.
Buddhists go to a lot of trouble to build meditation halls and monasteries.
Their main purpose is to create a physical environment free of distractions in
which to learn this skill. No noise, no interruptions. Just as important,
however, is the creation of a distraction-free emotional environment. The
development of concentration will be blocked by the presence of certain mental
states which we call the five hindrances. They are greed for sensual pleasure,
hatred, mental lethargy, restlessness, and mental vacillation. We have examined
these mental states more fully in Chapter 12.

A monastery is a controlled environment where this sort of emotional noise is
kept to a minimum. No members of the opposite sex are allowed to live together
there. Therefore, there is less opportunity for lust. No possessions are
allowed. Therefore, no ownership squabbles and less chance for greed and
coveting. Another hurdle for concentration should also be mentioned. In really
deep concentration, you get so absorbed in the object of concentration that you
forget all about trifles. Like your body, for instance, and your identity and
everything around you. Here again the monastery is a useful convenience. It is
nice to know that there is somebody to take care of you by watching over all the
mundane matters of food and physical security. Without such assurance, one
hesitates to go as deeply into concentration as one might.  Mindfulness, on the
other hand, is free from all these drawbacks. Mindfulness is not dependent on
any such particular circumstance, physical or otherwise. It is a pure noticing
factor. Thus it is free to notice whatever comes up--lust, hatred, or noise.

Mindfulness is not limited by any condition. It exists to some extent in every
moment, in every circumstance that arises. Also, mindfulness has no fixed object
of focus. It observes change. Thus it has an unlimited number of objects of
attention. It just looks at whatever is passing through the mind and it does not
categorize. Distractions and interruptions are noticed with the same amount of
attention as the formal objects of meditation. In a state of pure mindfulness
your attention just flows along with whatever changes are taking place in the
mind. ``Shift, shift, shift. Now this, now this, and now this." You can't develop
mindfulness by force. Active teeth gritting willpower won't do you any good at
all. As a matter of fact, it will hinder progress. Mindfulness cannot be
cultivated by struggle. It grows by realizing, by letting go, by just settling
down in the moment and letting yourself get comfortable with whatever you are
experiencing. This does not mean that mindfulness happens all by itself. Far
from it. Energy is required. Effort is required. But this effort is different
from force. Mindfulness is cultivated by a gentle effort, by effortless effort.
The meditator cultivates mindfulness by constantly reminding himself in a gentle
way to maintain his awareness of whatever is happening right now. Persistence
and a light touch are the secrets. Mindfulness is cultivated by constantly
pulling oneself back to a state of awareness, gently, gently, gently.

Mindfulness can't be used in any selfish way, either. It is nonegoistic
alertness. There is no `me' in a state of pure mindfulness. So there is no self
to be selfish. On the contrary, it is mindfulness which gives you the real
perspective on yourself. It allows you to take that crucial mental step backward
from your own desires and aversions so that you can then look and say, ``Ah ha,
so that's how I really am." In a state of mindfulness, you see yourself exactly
as you are. You see your own selfish behavior. You see your own suffering. And
you see how you create that suffering. You see how you hurt others. You pierce
right through the layer of lies that you normally tell yourself and you see what
is really there. Mindfulness leads to wisdom.

Mindfulness is not trying to achieve anything. It is just looking. Therefore,
desire and aversion are not involved. Competition and struggle for achievement
have no place in the process. Mindfulness does not aim at anything. It just sees
whatever is already there.

Mindfulness is a broader and larger function than concentration. it is an
all-encompassing function. Concentration is exclusive. It settles down on one
item and ignores everything else. Mindfulness is inclusive. It stands back from
the focus of attention and watches with a broad focus, quick to notice any
change that occurs. If you have focused the mind on a stone, concentration will
see only the stone. Mindfulness stands back from this process, aware of the
stone, aware of the concentration focusing on the stone, aware of the intensity
of that focus and instantly aware of the shift of attention when concentration
is distracted. It is mindfulness which notices the distraction which has
occurred, and it is mindfulness which redirects the attention to the stone.
Mindfulness is more difficult to cultivate than concentration because it is a
deeper-reaching function. Concentration is merely focusing of the mind, rather
like a laser beam. It has the power to burn its way deep into the mind and
illuminate what is there. But it does not understand what it sees. Mindfulness
can examine the mechanics of selfishness and understand what it sees.
Mindfulness can pierce the mystery of suffering and the mechanism of discomfort.
Mindfulness can make you free.

There is, however, another Catch-22. Mindfulness does not react to what it sees.
It just sees and understands. Mindfulness is the essence of patience. Therefore,
whatever you see must be simply accepted, acknowledged and dispassionately
observed. This is not easy, but it is utterly necessary. We are ignorant. We are
selfish and greedy and boastful. We lust and we lie. These are facts.
Mindfulness means seeing these facts and being patient with ourselves, accepting
ourselves as we are. That goes against the grain.  We don't want to accept. We
want to deny it. Or change it, or justify it. But acceptance is the essence of
mindfulness. If we want to grow in mindfulness we must accept what mindfulness
finds. It may be boredom, irritation, or fear. It may be weakness, inadequacy,
or faults. Whatever it is, that is the way we are. That is what is real.

Mindfulness simply accepts whatever is there. If you want to grow in
mindfulness, patient acceptance is the only route.  Mindfulness grows only one
way: by continuous practice of mindfulness, by simply trying to be mindful, and
that means being patient. The process cannot be forced and it cannot be rushed.
It proceeds at its own pace.

Concentration and mindfulness go hand-in-hand in the job of meditation.
Mindfulness directs the power of concentration.  Mindfulness is the manager of
the operation. Concentration furnishes the power by which mindfulness can
penetrate into the deepest level of the mind. Their cooperation results in
insight and understanding. These must be cultivated together in a balanced
ratio. Just a bit more emphasis is given to mindfulness because mindfulness is
the center of meditation. The deepest levels of concentration are not really
needed to do the job of liberation. Still, a balance is essential. Too much
awareness without calm to balance it will result in a wildly over sensitized
state similar to abusing LSD. Too much concentration without a balancing ratio
of awareness will result in the `Stone Buddha' syndrome. The meditator gets so
tranquilized that he sits there like a rock. Both of these are to be avoided.

The initial stages of mental cultivation are especially delicate. Too much
emphasis on mindfulness at this point will actually retard the development of
concentration. When getting started in meditation, one of the first things you
will notice is how incredibly active the mind really is. The Theravada tradition
calls this phenomenon `monkey mind'. The Tibetan tradition likens it to a
waterfall of thought. If you emphasize the awareness function at this point,
there will be so much to be aware of that concentration will be impossible.
Don't get discouraged. This happens to everybody. And there is a simple
solution. Put most of your effort into one-pointedness at the beginning. Just
keep calling the attention from wandering over and over again. Tough it out.
Full instructions on how to do this are in Chapters 7 and 8. A couple of months
down the track and you will have developed concentration power. Then you can
start pumping you energy into mindfulness. Do not, however, go so far with
concentration that you find yourself going into a stupor.

Mindfulness still is the more important of the two components. It should be
built as soon as you comfortably can do so.  Mindfulness provides the needed
foundation for the subsequent development of deeper concentration. Most blunders
in this area of balance will correct themselves in time. Right concentration
develops naturally in the wake of strong mindfulness. The more you develop the
noticing factor, the quicker you will notice the distraction and the quicker you
will pull out of it and return to the formal object of attention. The natural
result is increased concentration. And as concentration develops, it assists the
development of mindfulness. The more concentration power you have, the less
chance there is of launching off on a long chain of analysis about the
distraction. You simply note the distraction and return your attention to where
it is supposed to be.

Thus the two factors tend to balance and support each other's growth quite
naturally. Just about the only rule you need to follow at this point is to put
your effort on concentration at the beginning, until the monkey mind phenomenon
has cooled down a bit. After that, emphasize mindfulness. If you find yourself
getting frantic, emphasize concentration. If you find yourself going into a
stupor, emphasize mindfulness. Overall, mindfulness is the one to emphasize.

Mindfulness guides your development in meditation because mindfulness has the
ability to be aware of itself. It is mindfulness which will give you a
perspective on your practice. Mindfulness will let you know how you are doing.
But don't worry too much about that. This is not a race. You are not in
competition with anybody, and there is no schedule.

One of the most difficult things to learn is that mindfulness is not dependent
on any emotional or mental state. We have certain images of meditation.
Meditation is something done in quiet caves by tranquil people who move slowly.
Those are training conditions. They are set up to foster concentration and to
learn the skill of mindfulness. Once you have learned that skill, however, you
can dispense with the training restrictions, and you should. You don't need to
move at a snail's pace to be mindful. You don't even need to be calm. You can be
mindful while solving problems in intensive calculus. You can be mindful in the
middle of a football scrimmage. You can even be mindful in the midst of a raging
fury. Mental and physical activities are no bar to mindfulness. If you find your
mind extremely active, then simply observe the nature and degree of that
activity. It is just a part of the passing show within.
