\chapter{Dealing with Distractions -- I}

At some time, every meditator encounters distractions during practice, and
methods are needed to deal with them. Some elegant stratagems have been devised
to get you back on the track more quickly than trying to push your way through
by sheer force of will.  Concentration and mindfulness go hand-in-hand. Each one
complements the other.  If either one is weak, the other will eventually be
affected. Bad days are usually characterized by poor concentration. Your mind
just keeps floating around. You need some method of reestablishing your
concentration, even in the face of mental adversity. Luckily, you have it. In
fact you can take your choice from a traditional array of practical maneuvers.

\subsection*{Maneuver 1: Time Gauging} This first technique has been covered in an earlier
chapter. A distraction has pulled you away from the breath, and you suddenly
realize that you've been day-dreaming. The trick is to pull all the way out of
whatever has captured you, to break its hold on you completely so you can go
back to the breath with full attention. You do this by gauging the length of
time that you were distracted. This is not a precise calculation. you don't need
a precise figure, just a rough estimate. You can figure it in minutes, or by
idea significance. Just say to yourself, ``Okay, I have been distracted for about
two minutes" or ``Since the dog started barking" or ``Since I started thinking
about money." When you first start practicing this technique, you will do it by
talking to yourself inside your head. Once the habit is well established, you
can drop that, and the action becomes wordless and very quick. The whole idea,
remember, is to pull out of the distraction and get back to the breath. You pull
out of the thought by making it the object of inspection just long enough to
glean from it a rough approximation of its duration. The interval itself is not
important.

Once you are free of the distraction, drop the whole thing and go back to the
breath. Do not get hung up in the estimate.

\subsection*{Maneuver 2: Deep Breaths}
When your mind is wild and agitated, you can often re-establish mindfulness with
a few quick deep breaths. Pull the air in strongly and let it out the same way.
This increases the sensation inside the nostrils and makes it easier to focus.
Make a strong act of will and apply some force to your attention. Concentration
can be forced into growth, remember, so you will probably find your full
attention settling nicely back on the breath.

\subsection*{Maneuver 3: Counting}
Counting the breaths as they pass is a highly traditional procedure. Some
schools of practice teach this activity as their primary tactic.  Vipassana uses
it as an auxiliary technique for re-establishing mindfulness and for
strengthening concentration. As we discussed in Chapter 5, you can count breaths
in a number of different ways. Remember to keep your attention on the breath.
You will probably notice a change after you have done your counting. The breath
slows down, or it becomes very light and refined.  This is a physiological
signal that concentration has become well-established.  At this point, the
breath is usually so light or so fast and gentle that you can't clearly
distinguish the inhalation from the exhalation. They seem to blend into each
other. You can then count both of them as a single cycle. Continue your counting
process, but only up to a count of five, covering the same five- breath
sequence, then start over. When counting becomes a bother, go on to the next
step. Drop the numbers and forget about the concepts of inhalation and
exhalation. Just dive right in to the pure sensation of breathing. Inhalation
blends into exhalation. One breath blends into the next in a never ending cycle
of pure, smooth flow.

\subsection*{Maneuver 4: The In-Out Method} This is an alternative to counting, and it
functions in much the manner. Just direct your attention to the breath and
mentally tag each cycle with the words ``Inhalation...exhalation" or ``In...out".
Continue the process until you no longer need these concepts, and then throw
them away.

\subsection*{Maneuver 5: Canceling One Thought With Another}
Some thoughts just won't go away.  We humans are obsessional beings. It's one of
our biggest problems. We tend to lock onto things like sexual fantasies and
worries and ambitions. We feed those though complexes over the years of time and
give them plenty of exercise by playing with them in every spare moment. Then
when we sit down to meditate, we order them to go away and leave us alone. It is
scarcely surprising that they don't obey. Persistent thoughts like these require
a direct approach, a full- scale frontal attack.

Buddhist psychology has developed a distinct system of classification. Rather
than dividing thoughts into classes like `good' or `bad', Buddhist thinkers
prefer to regard them as `skillful' versus `unskillful'. An unskillful thought
is on connected with greed, hatred, or delusion. These are the thoughts that the
mind most easily builds into obsessions. They are unskillful in the sense that
they lead you away from the goal of Liberation. Skillful thoughts, on the other
hand, are those connected with generosity, compassion, and wisdom. They are
skillful in the sense that they may be used as specific remedies for unskillful
thoughts, and thus can assist you toward Liberation.

You cannot condition Liberation. It is not a state built out of thoughts. Nor
can you condition the personal qualities which Liberation produces. Thoughts of
benevolence can produce a semblance of benevolence, but it's not the real item.
It will break down under pressure. Thoughts of compassion produce only
superficial compassion. Therefore, these skillful thoughts will not, in
themselves, free you from the trap. They are skillful only if applied as
antidotes to the poison of unskillful thoughts. Thoughts of generosity can
temporarily cancel greed. They kick it under the rug long enough for mindfulness
to do its work unhindered. Then, when mindfulness has penetrated to the roots of
the ego process, greed evaporates and true generosity arises.

This principle can be used on a day to day basis in your own meditation. If a
particular sort of obsession is troubling you, you can cancel it out by
generating its opposite. Here is an example: If you absolutely hate Charlie, and
his scowling face keeps popping into your mind, try directing a stream of love
and friendliness toward Charlie. You probably will get rid of the immediate
mental image. Then you can get on with the job of meditation.

Sometimes this tactic alone doesn't work. The obsession is simply too strong. In
this case you've got to weaken its hold on you somewhat before you can
successfully balance it out. Here is where guilt, one of man's most misbegotten
emotions, finally becomes of some use. Take a good strong look at the emotional
response you are trying to get rid of. Actually ponder it. See how it makes you
feel. Look at what it is doing to your life, your happiness, your health, and
your relationships. Try to see how it makes you appear to others. Look at the
way it is hindering your progress toward Liberation. The Pali scriptures urge
you to do this very thoroughly indeed. They advise you to work up the same sense
of disgust and humiliation that you would feel if you were forced to walk around
with the carcass of a dead and decaying animal tied around your neck. Real
loathing is what you are after. This step may end the problem all by itself. If
it doesn't, then balance out the lingering remainder of the obsession by once
again generating its opposite emotion.

Thoughts of greed cover everything connected with desire, from outright avarice
for material gain, all the way down to a subtle need to be respected as a moral
person. Thoughts of hatred run the gamut from petty peevishness to murderous
rage. Delusion covers everything from daydreaming through actual hallucinations.
Generosity cancels greed. Benevolence and compassion cancel hatred. You can find
a specific antidote for any troubling thought if you just think about it a
while.

\subsection*{Maneuver 6: Recalling Your Purpose}
There are times when things pop into your
mind, apparently at random. Words, phrases, or whole sentences jump up out of
the unconscious for no discernible reason. Objects appear. Pictures flash on and
off. This is an unsettling experience. Your mind feels like a flag flapping in a
stiff wind. It washes back and forth like waves in the ocean. At times like this
it is often enough just to remember why you are there. You can say to yourself,
``I'm not sitting here just to waste my time with these thoughts. I'm here to
focus my mind on the breath, which is universal and common to all living
beings". Sometimes your mind will settle down, even before you complete this
recitation. Other times you may have to repeat it several times before you
refocus on the breath.

These techniques can be used singly, or in combinations. Properly employed, they
constitute quite an effective arsenal for your battle against the monkey mind.

\chapter{Dealing with Distractions -- II}
 So there you are meditating
beautifully. Your body is totally immobile, and you mind is totally still. You
just glide right along following the flow of the breath, in, out, in,
out...calm, serene and concentrated. Everything is perfect. And then, all of a
sudden, something totally different pops into your mind: ``I sure wish I had an
ice cream cone." That's a distraction, obviously. That's not what you are
supposed to be doing. You notice that, and you drag yourself back to the breath,
back to the smooth flow, in, out, in...and then: ``Did I ever pay that gas bill?"
Another distraction. You notice that one, and you haul yourself back to the
breath.

In, out, in, out, in... ``That new science fiction movie is out. Maybe I can go
see it Tuesday night. No, not Tuesday, got too much to do on Wednesday.
Thursday's better..." Another distraction. You pull yourself out of that one and
back you go to the breath, except that you never quite get there because before
you do that little voice in your head goes, ``My back is killing me." And on and
on it goes, distraction after distraction, seemingly without end.

What a bother. But this is what it is all about. These distractions are actually
the whole point. The key is to learn to deal with these things. Learning to
notice them without being trapped in them. That's what we are here for. The
mental wandering is unpleasant, to be sure. But it is the normal mode of
operation of your mind. Don't think of it as the enemy. It is just the simple
reality. And if you want to change something, the first thing you have to do is
see it the way it is.

When you first sit down to concentrate on the breath, you will be struck by how
incredibly busy the mind actually is. It jumps and jibbers. It veers and bucks.
It chases itself around in constant circles. It chatters. It thinks. It
fantasizes and daydreams. Don't be upset about that. It's natural. When your
mind wanders from the subject of meditation, just observe the distraction
mindfully.

When we speak of a distraction in Insight Meditation, we are speaking of any
preoccupation that pulls the attention off the breath.  This brings up a new,
major rule for your meditation: When any mental state arises strongly enough to
distract you from the object of meditation, switch your attention to the
distraction briefly. Make the distraction a temporary object of meditation.
Please not the word temporary. It's quite important. We are not advising that
you switch horses in midstream. We do not expect you to adopt a whole new object
of meditation every three seconds. The breath will always remain your primary
focus. You switch your attention to the distraction only long enough to notice
certain specific things about it. What is it? How strong is it? and, how long
does it last? As soon as you have wordlessly answered these questions, you are
through with your examination of that distraction, and you return your attention
to the breath. Here again, please note the operant term, wordlessly. These
questions are not an invitation to more mental chatter. That would be moving you
in the wrong direction, toward more thinking. We want you to move away from
thinking, back to a direct, wordless and nonconceptual experience of the breath.
These questions are designed to free you from the distraction and give you
insight into its nature, not to get you more thoroughly stuck in it. They will
tune you in to what is distracting you and help you get rid of it --all in one
step.

Here is the problem: When a distraction, or any mental state, arises in the
mind, it blossoms forth first in the unconscious. Only a moment later does it
rise to the conscious mind. That split-second difference is quite important,
because it time enough for grasping to occur. Grasping occurs almost
instantaneously, and it takes place first in the unconscious. Thus, by the time
the grasping rises to the level of conscious recognition, we have already begun
to lock on to it. It is quite natural for us to simply continue that process,
getting more and more tightly stuck in the distraction as we continue to view
it. We are, by this time, quite definitely thinking the thought, rather than
just viewing it with bare attention. The whole sequence takes place in a flash.
This presents us with a problem. By the time we become consciously aware of a
distraction we are already, in a sense, stuck in it. Our three questions are a
clever remedy for this particular malady. In order to answer these questions, we
must ascertain the quality of the distraction. To do that, we must divorce
ourselves from it, take a mental step back from it, disengage from it, and view
it objectively. We must stop thinking the thought or feeling the feeling in
order to view it as an object of inspection. This very process is an exercise in
mindfulness, uninvolved, detached awareness. The hold of the distraction is thus
broken, and mindfulness is back in control. At this point, mindfulness makes a
smooth transition back to its primary focus and we return to the breath.

When you first begin to practice this technique, you will probably have to do it
with words. You will ask your questions in words, and get answers in words. It
won't be long, however, before you can dispense with the formality of words
altogether. Once the mental habits are in place, you simply note the
distraction, note the qualities of the distraction, and return to the breath.
It's a totally nonconceptual process, and it's very quick. The distraction
itself can be anything: a sound, a sensation, an emotion, a fantasy, anything at
all. Whatever it is, don't try to repress it. Don't try to force it out of your
mind. There's no need for that. Just observe it mindfully with bare attention.
Examine the distraction wordlessly and it will pass away by itself. You will
find your attention drifting effortlessly back to the breath. And do not condemn
yourself for having being distracted. Distractions are natural.
They come and they go.

Despite this piece of sage counsel, you're going to find yourself condemning
anyway. That's natural too. Just observe the process of condemnation as another
distraction, and then return to the breath.  Watch the sequence of events:
Breathing. Breathing. Distracting thought arises.  Frustration arising over the
distracting thought.  You condemn yourself for being distracted. You notice the
self condemnation. You return to the breathing. Breathing. Breathing.  It's
really a very natural, smooth-flowing cycle, if you do it correctly. The trick,
of course, is patience. If you can learn to observe these distractions without
getting involved, it's all very easy. You just glide through the distractions
and your attention returns to the breath quite easily. Of course, the very same
distraction may pop up a moment later. If it does, just observe that mindfully.
If you are dealing with an old, established thought pattern, this can go on
happening for quite a while, sometimes years. Don't get upset.  This too is
natural. just observe the distraction and return to the breath.  Don't fight
with these distracting thoughts. Don't strain or struggle. It's a waste. Every
bit of energy that you apply to that resistance goes into the thought complex
and makes it all the stronger. So don't try to force such thoughts out of your
mind. It's a battle you can never win. Just observe the distraction mindfully
and, it will eventually go away. It's very strange, but the more bare attention
you pay to such disturbances, the weaker they get.  Observe them long enough,
and often enough, with bare attention, and they fade away forever. Fight with
them and they gain in strength. Watch them with detachment and they wither.

Mindfulness is a function that disarms distractions, in the same way that a
munitions expert might defuse a bomb. Weak distractions are disarmed by a single
glance. Shine the light of awareness on them and they evaporate instantly, never
to return.  Deep-seated, habitual thought patterns require constant mindfulness
repeatedly applied over whatever time period it takes to break their hold.
Distractions are really paper tigers. They have no power of their own. They need
to be fed constantly, or else they die.  If you refuse to feed them by your own
fear, anger, and greed, they fade.

Mindfulness is the most important aspect of meditation. It is the primary thing
that you are trying to cultivate. So there is really no need at all to struggle
against distractions. The crucial thing is to be mindful of what is occurring,
not to control what is occurring. Remember, concentration is a tool. It is
secondary to bare attention. From the point of view of mindfulness, there is
really no such thing as a distraction. Whatever arises in the mind is viewed as
just one more opportunity to cultivate mindfulness.  Breath, remember, is an
arbitrary focus, and it is used as our primary object of attention. Distractions
are used as secondary objects of attention. They are certainly as much a part of
reality as breath. It actually makes rather little difference what the object of
mindfulness is. You can be mindful of the breath, or you can be mindful of the
distraction. You can be mindful of the fact that you mind is still, and your
concentration is strong, or you can be mindful of the fact that your
concentration is in ribbons and your mind is in an absolute shambles. It's all
mindfulness. Just maintain that mindfulness and concentration eventually will
follow.

The purpose of meditation is not to concentrate on the breath, without
interruption, forever. That by itself would be a useless goal.  The purpose of
meditation is not to achieve a perfectly still and serene mind. Although a
lovely state, it doesn't lead to liberation by itself. The purpose of meditation
is to achieve uninterrupted mindfulness. Mindfulness, and only mindfulness,
produces Enlightenment.

Distractions come in all sizes, shapes and flavors. Buddhist philosophy has
organized them into categories. One of them is the category of hindrances. They
are called hindrances because they block your development of both components of
mediation, mindfulness and concentration. A bit of caution on this term: The
word `hindrances' carries a negative connotation, and indeed these are states of
mind we want to eradicate. That does not mean, however, that they are to be
repressed, avoided or condemned.

Let's use greed as an example. We wish to avoid prolonging any state of greed
that arises, because a continuation of that state leads to bondage and sorrow.
That does not mean we try to toss the thought out of the mind when it appears.
We simply refuse to encourage it to stay. We let it come, and we let it go. When
greed is first observed with bare attention, no value judgements are made. We
simply stand back and watch it arise. The whole dynamic of greed from start to
finish is simply observed in this way.

We don't help it, or hinder it, or interfere with it in the slightest. It stays
as long as it stays. And we learn as much about it as we can while it is there.
We watch what greed does. We watch how it troubles us, and how it burdens
others. We notice how it keeps us perpetually unsatisfied, forever in a state of
unfulfilled longing. From this first-hand experience, we ascertain at a gut
level that greed is an unskillful way to run your life. There is nothing
theoretical about this realization.

All of the hindrances are dealt with in the same way, and we will look at them
here one by one.

\paragraph*{Desire}: Let us suppose you have been distracted by some nice experience in meditation. It could be pleasant fantasy or a thought
of pride. It might be a feeling of self-esteem. It might be a thought of love or even the physical sensation of bliss that comes with
the meditation experience itself. Whatever it is, what follows is the state of desire -- desire to obtain whatever you have been
thinking about or desire to prolong the experience you are having. No matter what its nature, you should handle desire in the
following manner. Notice the thought or sensation as it arises. Notice the mental state of desire which accompanies it as a separate
thing. Notice the exact extent or degree of that desire. Then notice how long it lasts and when it finally disappears. When you
have done that, return your attention to breathing.

\paragraph*{Aversion}: Suppose that you have been distracted by some negative experience. It
could be something you fear or some nagging worry. It might be guilt or
depression or pain. Whatever the actual substance of the thought or sensation,
you find yourself rejecting or repressing -- trying to avoid it, resist it or
deny it. The handling here is essentially the same. Watch the arising of the
thought or sensation. Notice the state of rejection that comes with it. Gauge
the extent or degree of that rejection. See how long it lasts and when it fades
away. Then return your attention to your breath.

\paragraph*{Lethargy}: Lethargy comes in various grades and intensities, ranging from slight
drowsiness to total torpor. We are talking about a mental state here, not a
physical one. Sleepiness or physical fatigue is something quite different and,
in the Buddhist system of classification, it would be categorized as a physical
feeling. Mental lethargy is closely related to aversion in that it is one of the
mind's clever little ways of avoiding those issues it finds unpleasant. Lethargy
is a sort of turn-off of the mental apparatus, a dulling of sensory and
cognitive acuity. It is an enforced stupidity pretending to be sleep. This can
be a tough one to deal with, because its presence is directly contrary to the
employment of mindfulness. Lethargy is nearly the reverse of mindfulness.
Nevertheless, mindfulness is the cure for this hindrance, too, and the handling
is the same. Note the state of drowsiness when it arises, and note its extent or
degree. Note when it arises, how long it lasts, and when it passes away. The
only thing special here is the importance of catching the phenomenon early. You
have got to get it right at its conception and apply liberal doses of pure
awareness right away. If you let it get a start, its growth probably will out
pace your mindfulness power. When lethargy wins, the result is the sinking mind
and/or sleep.

\paragraph*{Agitation}: States of restlessness and worry are expressions of mental agitation.
Your mind keeps darting around, refusing to settle on any one thing. You may
keep running over and over the same issues. But even here an unsettled feeling
is the predominant component. The mind refuses to settle anywhere. It jumps
around constantly. The cure for this condition is the same basic sequence.
Restlessness imparts a certain feeling to consciousness. You might call it a
flavor or texture. Whatever you call it, that unsettled feeling is there as a
definable characteristic. Look for it. Once you have spotted it, note how much
of it is present. Note when it arises. Watch how long it lasts, and see when it
fades away. Then return your attention to the breath.

\paragraph*{Doubt}: Doubt has its own distinct feeling in consciousness. The Pali tests
describe it very nicely. It's the feeling of a man stumbling through a desert
and arriving at an unmarked crossroad. Which road should he take? There is no
way to tell. So he just stands there vacillating. One of the common forms this
takes in meditation is an inner dialogue something like this: ``What am I doing
just sitting like this? Am I really getting anything out of this at all? Oh!
Sure I am. This is good for me. The book said so. No, that is crazy. This is a
waste of time. No, I won't give up. I said I was going to do this, and I am
going to do it. Or am I being just stubborn? I don't know. I just don't know."
Don't get stuck in this trap. It is just another hindrance. Another of the
mind's little smoke screens to keep you from doing the most terrible thing in
the world: actually becoming aware of what is happening. To handle doubt, simply
become aware of this mental state of wavering as an object of inspection. Don't
be trapped in it. Back out of it and look at it. See how strong it is. See when
it comes and how long it lasts. Then watch it fade away, and go back to the
breathing.

This is the general pattern you will use on any distraction that arises. By
distraction, remember we mean any mental state that arises to impede your
meditation. Some of these are quite subtle. It is useful to list some of the
possibilities. The negative states are pretty easy to spot: insecurity, fear,
anger, depression, irritation and frustration.

Craving and desire are a bit more difficult to spot because they can apply to
things we normally regard as virtuous or noble. You can experience the desire to
perfect yourself. You can feel craving for greater virtue. You can even develop
an attachment to the bliss of the meditation experience itself. It is a bit hard
to detach yourself from such altruistic feelings. In the end, though, it is just
more greed. It is a desire for gratification and a clever way of ignoring the
present-time reality.

Trickiest of all, however, are those really positive mental states that come creeping into your meditation. Happiness, peace, inner
contentment, sympathy and compassion for all beings everywhere. These mental states are so sweet and so benevolent that you
can scarcely bear to pry yourself loose from them. It makes you feel like a traitor to mankind. There is no need to feel this way.

We are not advising you to reject these states of mind or to become heartless
robots. We merely want you to see them for what they are. They are mental
states. They come and they go. They arise and they pass away. As you continue
your meditation, these states will arise more often. The trick is not to become
attached to them. Just see each one as it comes up. See what it is, how strong
it is and how long it lasts. Then watch it drift away. It is all just more of
the passing show of your own mental universe.

Just as breathing comes in stages, so do the mental states. Every breath has a
beginning, a middle and an end. Every mental states has a birth, a growth and a
decay. You should strive to see these stages clearly. This is no easy thing to
do, however. As we have already noted, every thought and sensation begins first
in the unconscious region of the mind and only later rises to consciousness.  We
generally become aware of such things only after they have arisen in the
conscious realm and stayed there for some time.  Indeed we usually become aware
of distractions only when they have released their hold on us and are already on
their way out. It is at this point that we are struck with the sudden
realization that we have been somewhere, day-dreaming, fantasizing, or whatever.
Quite obviously this is far too late in the chain of events. We may call this
phenomenon catching the lion by is tail, and it is an unskillful thing to do.
Like confronting a dangerous beast, we must approach mental states head-on.
Patiently, we will learn to recognize them as they arise from progressively
deeper levels of our conscious mind.

Since mental states arise first in the unconscious, to catch the arising of the
mental state, you've got to extend your awareness down into this unconscious
area. That is difficult, because you can't see what is going on down there, at
least not in the same way you see a conscious thought. But you can learn to get
a vague sense of movement and to operate by a sort of mental sense of touch.
This comes with practice, and the ability is another of the effects of the deep
calm of concentration. Concentration slows down the arising of these mental
states and gives you time to feel each one arising out of the unconscious even
before you see it in consciousness. Concentration helps you to extend your
awareness down into that boiling darkness where thought and sensation begin.

As your concentration deepens, you gain the ability to see thoughts and
sensations arising slowly, like separate bubbles, each distinct and with spaces
between them. They bubble up in slow motion out of the unconscious. They stay a
while in the conscious mind and then they drift away.

The application of awareness to mental states is a precision operation. This is
particularly true of feelings or sensations. It is very easy to overreach the
sensation. That is, to add something to it above and beyond what is really
there. It is equally easy to fall short of sensation, to get part of it but not
all. The ideal that you are striving for is to experience each mental state
fully, exactly the way it is, adding nothing to it and not missing any part of
it. Let us use pain in the leg as an example. What is actually there is a pure
flowing sensation. It changes constantly, never the same from one moment to the
next. It moves from one location to another, and its intensity surges up and
down. Pain is not a thing. It is an event. There should be no concepts tacked on
to it and none associated with it. A pure unobstructed awareness of this event
will experience it simply as a flowing pattern of energy and nothing more. No
thought and no rejection. Just energy.

Early on in our practice of meditation, we need to rethink our underlying
assumptions regarding conceptualization. For most of us, we have earned high
marks in school and in life for our ability to manipulate mental phenomena --
concepts -- logically. Our careers, much of our success in everyday life, our
happy relationships, we view as largely the result of our successful
manipulation of concepts. In developing mindfulness, however, we temporarily
suspend the conceptualization process and focus on the pure nature of mental
phenomena. During meditation we are seeking to experience the mind at the
pre-concept level.

But the human mind conceptualizes such occurrences as pain. You find yourself
thinking of it as `the pain'. That is a concept. It is a label, something added
to the sensation itself. You find yourself building a mental image, a picture of
the pain, seeing it as a shape. You may see a diagram of the leg with the pain
outlined in some lovely color. This is very creative and terribly entertaining,
but not what we want. Those are concepts tacked on to the living reality. Most
likely, you will probably find yourself thinking: ``I have a pain in my leg." `I'
is a concept. It is something extra added to the pure experience.

When you introduce `I' into the process, you are building a conceptual gap
between the reality and the awareness viewing that reality. Thoughts such as
`Me', `My' or `Mine' have no place in direct awareness. They are extraneous
addenda, and insidious ones at that. When you bring `me' into the picture, you
are identifying with the pain. That simply adds emphasis to it. If you leave `I'
out of the operation, pain is not painful. It is just a pure surging energy
flow. It can even be beautiful. If you find `I' insinuating itself in your
experience of pain or indeed any other sensation, then just observe that
mindfully. Pay bare attention to the phenomenon of personal identification with
the pain.

The general idea, however, is almost too simple. You want to really see each
sensation, whether it is pain, bliss or boredom. You want to experience that
thing fully in its natural and unadulterated form. There is only one way to do
this. Your timing has to be precise. Your awareness of each sensation must
coordinate exactly with the arising of that sensation. If you catch it just a
bit too late, you miss the beginning. You won't get all of it. If you hang on to
any sensation past the time when it has memory. The thing itself is gone, and by
holding onto that memory, you miss the arising of the next sensation. It is a
very delicate operation.  You've got to cruise along right here in present time,
picking things up and letting things drop with no delays whatsoever. It takes a
very light touch. Your relation to sensation should never be one of past or
future but always of the simple and immediate now.

The human mind seeks to conceptualize phenomena, and it has developed a host of
clever ways to do so. Every simple sensation will trigger a burst of conceptual
thinking if you give the mind its way. Lets us take hearing, for example. You
are sitting in meditation and somebody in the next room drops a dish. The sounds
strike your ear. Instantly you see a picture of that other room. You probably
see a person dropping a dish, too. If this a familiar environment, say your own
home, you probably will have a 3-D technicolor mind movie of who did the
dropping and which dish was dropped. This whole sequence presents itself to
consciousness instantly. It just jumps out of the unconscious so bright and
clear and compelling that it shoves everything else out of sight. What happens
to the original sensation, the pure experience of hearing? It got lost in the
shuffle, completely overwhelmed and forgotten. We miss reality. We enter a world
of fantasy.

Here is another example: You are sitting in meditation and a sound strikes your
ear. It is just an indistinct noise, sort of a muffled crunch; it could be
anything. What happens next will probably be something like this. ``What was
that? Who did that? Where did that come from? How far away was that? Is it
dangerous?". And on and on you go, getting no answers but your fantasy
projection. Conceptualization is an insidiously clever process It creeps into
you experience, and it simply takes over. When you hear a sound in meditation,
pay bare attention to the experience of hearing. That and that only. What is
really happening is so utterly simple that we can and do miss it altogether.
Sound waves are striking the ear in a certain unique pattern. Those waves are
being translated into electrical impulses within the brain and those impulses
present a sound pattern to consciousness. That is all.

No pictures. No mind movies. No concepts. No interior dialogues about the
question. Just noise. Reality is elegantly simple and unadorned. When you hear a
sound, be mindful of the process of hearing. Everything else is just added
chatter. Drop it. The same rule applies to every sensation, every emotion, every
experience you may have. Look closely at your own experience. Dig down through
the layers of mental bric-a-brac and see what is really there. You will be
amazed how simple it is, and how beautiful.

There are times when a number of sensations may arise at once. You might have a
thought of fear, a squeezing in the stomach and an aching back and an itch on
your left earlobe, all at the same time. Don't sit there in a quandary. Don't
keep switching back and forth or wondering what to pick. One of them will be
strongest. Just open yourself up and the most insistent of these phenomena will
intrude itself and demand your attention. So give it some attention just long
enough to see it fade away. Then return to your breathing. If another one
intrudes itself, let it in. When it is done, return to the breathing.

This process can be carried too far, however. Don't sit there looking for things
to be mindful of. Keep your mindfulness on the breath until something else steps
in and pulls your attention away. When you feel that happening, don't fight it.
Let you attention flow naturally over to the distraction, and keep it there
until the distraction evaporates. Then return to breathing. Don't seek out other
physical or mental phenomena. Just return to breathing. Let them come to you.
There will be times when you drift off, of course. Even after long practice you
find yourself suddenly waking up, realizing you have been off the track for some
while. Don't get discouraged. Realize that you have been off the track for such
and such a length of time and go back to the breath. There is no need for any
negative reaction at all. The very act of realizing that you have been off the
track is an active awareness. It is an exercise of pure mindfulness all by
itself.

Mindfulness grows by the exercise of mindfulness. It is like exercising a
muscle. Every time you work it, you pump it up just a little. You make it a
little stronger. The very fact that you have felt that wake-up sensation means
that you have just improved your mindfulness power. That means you win. Move
back to the breathing without regret. However, the regret is a conditioned
reflex and it may come along anyway--another mental habit. If you find yourself
getting frustrated, feeling discouraged, or condemning yourself, just observe
that with bare attention. It is just another distraction. Give it some attention
and watch it fade away, and return to the breath.

The rules we have just reviewed can and should be applied thoroughly to all of
your mental states. You are going to find this an utterly ruthless injunction.
It is the toughest job that you will ever undertake. You will find yourself
relatively willing to apply this technique to certain parts of your experience,
and you will find yourself totally unwilling to use it on the other parts.

Meditation is a bit like mental acid. It eats away slowly at whatever you put it
on. We humans are very odd beings. We like the taste of certain poisons and we
stubbornly continue to eat them even while they are killing us. Thoughts to
which we are attached are poison. You will find yourself quite eager to dig some
thoughts out by the roots while you jealously guard and cherish certain others.
That is the human condition.

Vipassana meditation is not a game. Clear awareness is more than a pleasurable
pastime. It is a road up and out of the quagmire in which we are all stuck, the
swamp of our own desires and aversions. It is relatively easy to apply awareness
to the nastier aspects of your existence. Once you have seen fear and depression
evaporate in the hot, intense beacon of awareness, you want to repeat the
process. Those are the unpleasant mental states. They hurt. You want to get rid
of those things because they bother you. It is a good deal harder to apply that
same process to mental states which you cherish, like patriotism, or parental
protectiveness or true love. But it is just as necessary. Positive attachments
hold you in the mud just as assuredly as negative attachments. You may rise
above the mud far enough to breathe a bit more easily if you practice Vipassana
meditation with diligence. Vipassana meditation is the road to Nibbana. And from
the reports of those who have toiled their way to that lofty goal, it is well
worth every effort involved.