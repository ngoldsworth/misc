\chapter{What's In It For You}

You can expect certain benefits from your meditation. The initial ones are
practical, prosaic things; the later stages are profoundly transcendent. They
run together from the simple to the sublime. We will set forth some of them
here. Your own experience is all that counts.

Those things that we called hindrances or defilements are more than just
unpleasant mental habits. They are the primary manifestations of the ego process
itself. The ego sense itself is essentially a feeling of separation -- a
perception of distance between that which we call me, and that which we call
other. This perception is held in place only if it is constantly exercised, and
the hindrances constitute that exercise.

Greed and lust are attempts to get `some of that' for me; hatred and aversion
are attempts to place greater distance between `me and that'. All the
defilements depend upon the perception of a barrier between self and other, and
all of them foster this perception every time they are exercised. Mindfulness
perceives things deeply and with great clarity. It brings our attention to the
root of the defilements and lays bare their mechanism. It sees their fruits and
their effects upon us. It cannot be fooled. Once you have clearly seen what
greed really is and what it really does to you and to others, you just naturally
cease to engage in it. When a child burns his hand on a hot oven, you don't have
to tell him to pull it back; he does it naturally, without conscious thought and
without decision. There is a reflex action built into the nervous system for
just that purpose, and it works faster than thought. By the time the child
perceives the sensation of heat and begins to cry, the hand has already been
jerked back from the source of pain.

Mindfulness works in very much the same way: it is wordless, spontaneous and
utterly efficient. Clear mindfulness inhibits the growth of hindrances;
continuous mindfulness extinguishes them. Thus, as genuine mindfulness is built
up, the walls of the ego itself are broken down, craving diminishes,
defensiveness and rigidity lessen, you become more open, accepting and flexible.
You learn to share your loving-kindness.

Traditionally, Buddhists are reluctant to talk about the ultimate nature of
human beings. But those who are willing to make descriptive statements at all
usually say that our ultimate essence or Buddha nature is pure, holy and
inherently good. The only reason that human beings appear otherwise is that
their experience of that ultimate essence has been hindered; it has been blocked
like water behind a dam. The hindrances are the bricks of which the dam is
built. As mindfulness dissolves the bricks, holes are punched in the dam and
compassion and sympathetic joy come flooding forward. As meditative mindfulness
develops, your whole experience of life changes. Your experience of being alive,
the very sensation of being conscious, becomes lucid and precise, no longer just
an unnoticed background for your preoccupations. It becomes a thing consistently
perceived.

Each passing moment stands out as itself; the moments no longer blend together
in an unnoticed blur. Nothing is glossed over or taken for granted, no
experiences labeled as merely `ordinary'. Everything looks bright and special.
You refrain from categorizing your experiences into mental pigeonholes.
Descriptions and interpretations are chucked aside and each moment of time is
allowed to speak for itself. You actually listen to what it has to say, and you
listen as if it were being heard for the very first time. When your meditation
becomes really powerful, it also becomes constant. You consistently observe with
bare attention both the breath and every mental phenomenon. You feel
increasingly stable, increasingly moored in the stark and simple experience of
moment-to- moment existence.

Once your mind is free from thought, it becomes clearly wakeful and at rest in
an utterly simple awareness. This awareness cannot be described adequately.
Words are not enough. It can only be experienced. Breath ceases to be just
breath; it is no longer limited to the static and familiar concept you once
held. You no longer see it as a succession of just inhalations and exhalations;
it is no longer some insignificant monotonous experience. Breath becomes a
living, changing process, something alive and fascinating. It is no longer
something that takes place in time; it is perceived as the present moment
itself. Time is seen as a concept, not an experienced reality.

This is simplified, rudimentary awareness which is stripped of all extraneous
detail. It is grounded in a living flow of the present, and it is marked by a
pronounced sense of reality. You know absolutely that this is real, more real
than anything you have ever experienced. Once you have gained this perception
with absolute certainty, you have a fresh vantage point, a new criterion against
which to gauge all of your experience. After this perception, you see clearly
those moments when you are participating in bare phenomena alone, and those
moments when you are disturbing phenomena with mental attitudes. You watch
yourself twisting reality with mental comments, with stale images and personal
opinions. You know what you are doing, when you are doing it.

You become increasingly sensitive to the ways in which you miss the true
reality, and you gravitate towards the simple objective perspective which does
not add to or subtract from what is. You become a very perceptive individual.
From this vantage point, all is seen with clarity. The innumerable activities of
mind and body stand out in glaring detail. You mindfully observe the incessant
rise and fall of breath; you watch an endless stream of bodily sensations and
movements; you scan a rapid succession of thoughts and feelings, and you sense
the rhythm that echoes from the steady march of time. And in the midst of all
this ceaseless movement, there is no watcher, there is only watching.

In this state of perception, nothing remains the same for two consecutive
moments. Everything is seen to be in constant transformation. All things are
born, all things grow old and die. There are no exceptions. You awaken to the
unceasing changes of your own life. You look around and see everything in flux,
everything, everything, everything. It is all rising and falling, intensifying
and diminishing, coming into existence and passing away. All of life, every bit
of it from the infinitesimal to the Indian Ocean, is in motion constantly. You
perceive the universe as a great flowing river of experience. Your most
cherished possessions are slipping away, and so is your very life. Yet this
impermanence is no reason for grief. You stand there transfixed, staring at this
incessant activity, and your response is wondrous joy. It's all moving, dancing
and full of life.

As you continue to observe these changes and you see how it all fits together,
you become aware of the intimate connectedness of all mental, sensory and
affective phenomena. You watch one thought leading to another, you see
destruction giving rise to emotional reactions and feelings giving rise to more
thoughts. Actions, thoughts, feelings, desires -- you see all of them intimately
linked together in a delicate fabric of cause and effect. You watch pleasurable
experiences arise and fall and you see that they never last; you watch pain come
uninvited and you watch yourself anxiously struggling to throw it off; you see
yourself fail. It all happens over and over while you stand back quietly and
just watch it all work.

Out of this living laboratory itself comes an inner and unassailable conclusion.
You see that your life is marked by disappointment and frustration, and you
clearly see the source. These reactions arise out of your own inability to get
what you want, your fear of losing what you have already gained and your habit
of never being satisfied with what you have. These are no longer theoretical
concepts -- you have seen these things for yourself and you know that they are
real. You perceive your own fear, your own basic insecurity in the face of life
and death. It is a profound tension that goes all the way down to the root of
thought and makes all of life a struggle. You watch yourself anxiously groping
about, fearfully grasping for something, anything, to hold onto in the midst of
all these shifting sands, and you see that there is nothing to hold onto,
nothing that doesn't change.

You see the pain of loss and grief, you watch yourself being forced to adjust to
painful developments day after day in your own ordinary existence. You witness
the tensions and conflicts inherent in the very process of everyday living, and
you see how superficial most of your concerns really are. You watch the progress
of pain, sickness, old age and death. You learn to marvel that all these
horrible things are not fearful at all. They are simply reality.

Through this intensive study of the negative aspects of your existence, you
become deeply acquainted with dukkha, the unsatisfactory nature of all
existence. You begin to perceive dukkha at all levels of our human life, from
the obvious down to the most subtle. You see the way suffering inevitably
follows in the wake of clinging, as soon as you grasp anything, pain inevitably
follows. Once you become fully acquainted with the whole dynamic of desire, you
become sensitized to it. You see where it rises, when it rises and how it
affects you. You watch it operate over and over, manifesting through every sense
channel, taking control of the mind and making consciousness its slave.

In the midst of every pleasant experience, you watch your own craving and
clinging take place. In the midst of unpleasant experiences, you watch a very
powerful resistance take hold. You do not block these phenomena, you just watch
them, you see them as the very stuff of human thought. You search for that thing
you call `me', but what you find is a physical body and how you have identified
your sense of yourself with that bag of skin and bones. You search further and
you find all manner of mental phenomena, such as emotions, thought patterns and
opinions, and see how you identify the sense of yourself with each of them.

You watch yourself becoming possessive, protective and defensive over these
pitiful things and you see how crazy that is. You rummage furiously among these
various items, constantly searching for yourself--physical matter, bodily
sensations, feelings and emotions--it all keeps whirling round and round as you
root through it, peering into every nook and cranny, endlessly hunting for `me'.

You find nothing. In all that collection of mental hardware in this endless
stream of ever-shifting experience all you can find is innumerable impersonal
processes which have been caused and conditioned by previous processes. There is
no static self to be found; it is all process. You find thoughts but no thinker,
you find emotions and desires, but nobody doing them. The house itself is empty.
There is nobody home.

Your whole view of self changes at this point. You begin to look upon yourself
as if you were a newspaper photograph. When viewed with the naked eyes, the
photograph you see is a definite image. When viewed through a magnifying glass,
it all breaks down into an intricate configuration of dots. Similarly, under the
penetrating gaze of mindfulness, the feeling of self, an `I' or `being'
anything, loses its solidity and dissolves. There comes a point in insight
meditation where the three characteristics of existence--impermanence,
unsatisfactoriness and selflessness-- come rushing home with concept-searing
force. You vividly experience the impermanence of life, the suffering nature of
human existence, and the truth of no self. You experience these things so
graphically that you suddenly awake to the utter futility of craving, grasping
and resistance. In the clarity and purity of this profound moment, our
consciousness is transformed. The entity of self evaporates. All that is left is
an infinity of interrelated non- personal phenomena which are conditioned and
ever changing. Craving is extinguished and a great burden is lifted. There
remains only an effortless flow, without a trace of resistance or tension. There
remains only peace, and blessed Nibbana, the uncreated, is realized.