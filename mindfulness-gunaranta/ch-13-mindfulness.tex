\chapter{Mindfulness (Sati)}

Mindfulness is the English translation of the Pali word Sati. Sati is an
activity. What exactly is that? There can be no precise answer, at least not in
words. Words are devised by the symbolic levels of the mind and they describe
those realities with which symbolic thinking deals. Mindfulness is pre-symbolic.

It is not shackled to logic. Nevertheless, Mindfulness can be experienced --
rather easily -- and it can be described, as long as you keep in mind that the
words are only fingers pointing at the moon. They are not the thing itself. The
actual experience lies beyond the words and above the symbols. Mindfulness could
be describes in completely different terms than will be used here and each
description could still be correct.

Mindfulness is a subtle process that you are using at this very moment. The fact
that this process lies above and beyond words does not make it unreal--quite the
reverse. Mindfulness is the reality which gives rise to words--the words that
follow are simply pale shadows of reality. So, it is important to understand
that everything that follows here is analogy. It is not going to make perfect
sense. It will always remain beyond verbal logic. But you can experience it. The
meditation technique called Vipassana (insight) that was introduced by the
Buddha about twenty-five centuries ago is a set of mental activities
specifically aimed at experiencing a state of uninterrupted Mindfulness.

When you first become aware of something, there is a fleeting instant of pure
awareness just before you conceptualize the thing, before you identify it. That
is a stage of Mindfulness. Ordinarily, this stage is very short. It is that
flashing split second just as you focus your eyes on the thing, just as you
focus your mind on the thing, just before you objectify it, clamp down on it
mentally and segregate it from the rest of existence. It takes place just before
you start thinking about it--before your mind says, ``Oh, it's a dog." That
flowing, soft-focused moment of pure awareness is Mindfulness. In that brief
flashing mind-moment you experience a thing as an un-thing. You experience a
softly flowing moment of pure experience that is interlocked with the rest of
reality, not separate from it. Mindfulness is very much like what you see with
your peripheral vision as opposed to the hard focus of normal or central vision.
Yet this moment of soft, unfocused, awareness contains a very deep sort of
knowing that is lost as soon as you focus your mind and objectify the object
into a thing. In the process of ordinary perception, the Mindfulness step is so
fleeting as to be unobservable. We have developed the habit of squandering our
attention on all the remaining steps, focusing on the perception, recognizing
the perception, labeling it, and most of all, getting involved in a long string
of symbolic thought about it. That original moment of Mindfulness is rapidly
passed over. It is the purpose of the above mentioned Vipassana (or insight)
meditation to train us to prolong that moment of awareness.

When this Mindfulness is prolonged by using proper techniques, you find that
this experience is profound and it changes your entire view of the universe.
This state of perception has to be learned, however, and it takes regular
practice. Once you learn the technique, you will find that Mindfulness has many
interesting aspects.

\subsection*{The Characteristics of Mindfulness} 
Mindfulness is mirror-thought. It reflects only what is presently happening and
in exactly the way it is happening. There are no biases.

Mindfulness is non-judgmental observation. It is that ability of the mind to
observe without criticism. With this ability, one sees things without
condemnation or judgment. One is surprised by nothing. One simply takes a
balanced interest in things exactly as they are in their natural states. One
does not decide and does not judge. One just observes. Please note that when we
say ``One does not decide and does not judge," what we mean is that the meditator
observes experiences very much like a scientist observing an object under the
microscope without any preconceived notions, only to see the object exactly as
it is. In the same way the meditator notices impermanence, unsatisfactoriness
and selflessness.

It is psychologically impossible for us to objectively observe what is going on
within us if we do not at the same time accept the occurrence of our various
states of mind. This is especially true with unpleasant states of mind. In order
to observe our own fear, we must accept the fact that we are afraid. We can't
examine our own depression without accepting it fully. The same is true for
irritation and agitation, frustration and all those other uncomfortable
emotional states. You can't examine something fully if you are busy reflecting
its existence. Whatever experience we may be having, Mindfulness just accepts
it. It is simply another of life's occurrences, just another thing to be aware
of. No pride, no shame, nothing personal at stake--what is there, is there.

Mindfulness is an impartial watchfulness. It does not take sides. It does not
get hung up in what is perceived. It just perceives.  Mindfulness does not get
infatuated with the good mental states. It does not try to sidestep the bad
mental states. There is no clinging to the pleasant, no fleeing from the
unpleasant. Mindfulness sees all experiences as equal, all thoughts as equal,
all feelings as equal. Nothing is suppressed. Nothing is repressed. Mindfulness
does not play favorites.

Mindfulness is nonconceptual awareness. Another English term for Sati is `bare
attention'. It is not thinking. It does not get involved with thought or
concepts. It does not get hung up on ideas or opinions or memories. It just
looks. Mindfulness registers experiences, but it does not compare them. It does
not label them or categorize them. It just observes everything as if it was
occurring for the first time. It is not analysis which is based on reflection
and memory. It is, rather, the direct and immediate experiencing of whatever is
happening, without the medium of thought. It comes before thought in the
perceptual process.

Mindfulness is present time awareness. It takes place in the here and now. It is
the observance of what is happening right now, in the present moment. It stays
forever in the present, surging perpetually on the crest of the ongoing wave of
passing time. If you are remembering your second-grade teacher, that is memory.
When you then become aware that you are remembering your second-grade teacher,
that is mindfulness. If you then conceptualize the process and say to yourself,
``Oh, I am remembering", that is thinking.

Mindfulness is non-egoistic alertness. It takes place without reference to self.
With Mindfulness one sees all phenomena without references to concepts like
`me', `my' or `mine'. For example, suppose there is pain in your left leg.
Ordinary consciousness would say, ``I have a pain." Using Mindfulness, one would
simply note the sensation as a sensation. One would not tack on that extra
concept `I'. Mindfulness stops one from adding anything to perception, or
subtracting anything from it. One does not enhance anything. One does not
emphasize anything. One just observes exactly what is there --without distortion.

Mindfulness is goal-less awareness. In Mindfulness, one does not strain for
results. One does not try to accomplish anything.  When one is mindful, one
experiences reality in the present moment in whatever form it takes. There is
nothing to be achieved.  There is only observation.

Mindfulness is awareness of change. It is observing the passing flow of
experience. It is watching things as they are changing. it is seeing the birth,
growth, and maturity of all phenomena. It is watching phenomena decay and die.
Mindfulness is watching things moment by moment, continuously. It is observing
all phenomena--physical, mental or emotional--whatever is presently taking place
in the mind. One just sits back and watches the show. Mindfulness is the
observance of the basic nature of each passing phenomenon. It is watching the
thing arising and passing away. It is seeing how that thing makes us feel and
how we react to it.  It is observing how it affects others. In Mindfulness, one
is an unbiased observer whose sole job is to keep track of the constantly
passing show of the universe within. Please note that last point. In
Mindfulness, one watches the universe within. The meditator who is developing
Mindfulness is not concerned with the external universe. It is there, but in
meditation, one's field of study is one's own experience, one's thoughts, one's
feelings, and one's perceptions.  In meditation, one is one's own laboratory.
The universe within has an enormous fund of information containing the
reflection of the external world and much more. An examination of this material
leads to total freedom.

Mindfulness is participatory observation. The meditator is both participant and
observer at one and the same time. If one watches one's emotions or physical
sensations, one is feeling them at that very same moment. Mindfulness is not an
intellectual awareness.

It is just awareness. The mirror-thought metaphor breaks down here. Mindfulness
is objective, but it is not cold or unfeeling. It is the wakeful experience of
life, an alert participation in the ongoing process of living.

Mindfulness is an extremely difficult concept to define in words -- not because
it is complex, but because it is too simple and open. The same problem crops up
in every area of human experience. The most basic concept is always the most
difficult to pin down. Look at a dictionary and you will see a clear example.
Long words generally have concise definitions, but for short basic words like
`the' and `is', definitions can be a page long. And in physics, the most
difficult functions to describe are the most basic-- those that deal with the
most fundamental realities of quantum mechanics. Mindfulness is a pre-symbolic
function. You can play with word symbols all day long and you will never pin it
down completely. We can never fully express what it is. However, we can say what
it does.

\subsection*{Three Fundamental Activities}
 There are three fundamental activities of
Mindfulness. We can use these activities as functional definitions of the term:
(a) Mindfulness reminds us of what we are supposed to be doing; (b) it sees
things as they really are; and (c) it sees the deep nature of all phenomena.
Let's examine these definitions in greater detail.

\emph{(a) Mindfulness reminds you of what you are supposed to be doing}. In
meditation, you put your attention on one item. When your mind wanders from this
focus, it is Mindfulness that reminds you that your mind is wandering and what
you are supposed to be doing. It is Mindfulness that brings your mind back to
the object of meditation. All of this occurs instantaneously and without
internal dialogue. Mindfulness is not thinking. Repeated practice in meditation
establishes this function as a mental habit which then carries over into the
rest of your life. A serious meditator pays bare attention to occurrences all
the time, day in, day out, whether formally sitting in meditation or not. This
is a very lofty ideal towards which those who meditate may be working for a
period of years or even decades. Our habit of getting stuck in thought is years
old, and that habit will hang on in the most tenacious manner. The only way out
is to be equally persistent in the cultivation of constant Mindfulness. When
Mindfulness is present, you will notice when you become stuck in your thought
patterns. It is that very noticing which allows you to back out of the thought
process and free yourself from it. Mindfulness then returns your attention to
its proper focus. If you are meditating at that moment, then your focus will be
the formal object of meditation. If your are not in formal meditation, it will
be just a pure application of bare attention itself, just a pure noticing of
whatever comes up without getting involved --``Ah, this comes up...and now this,
and now this... and now this".

Mindfulness is at one and the same time both bare attention itself and the
function of reminding us to pay bare attention if we have ceased to do so. Bare
attention is noticing. It re- establishes itself simply by noticing that it has
not been present. As soon as you are noticing that you have not been noticing,
then by definition you are noticing and then you are back again to paying bare
attention.

Mindfulness creates its own distinct feeling in consciousness. It has a
flavor--a light, clear, energetic flavor. Conscious thought is heavy by
comparison, ponderous and picky. But here again, these are just words. Your own
practice will show you the difference.

Then you will probably come up with your own words and the words used here will
become superfluous. Remember, practice is the thing.

\emph{(b) Mindfulness sees things as they really are.} Mindfulness adds nothing to
perception and it subtracts nothing. It distorts nothing. It is bare attention and just looks at whatever comes up. Conscious
thought pastes things over our experience, loads us down with concepts and
ideas, immerses us in a churning vortex of plans and worries, fears and
fantasies. When mindful, you don't play that game. You just notice exactly what
arises in the mind, then you notice the next thing. ``Ah, this...and this...and
now this." It is really very simple.

\emph{(c) Mindfulness sees the true nature of all phenomena.} Mindfulness and only
Mindfulness can perceive the three prime characteristics that Buddhism teaches
are the deepest truths of existence. In Pali these three are called Anicca
(impermanence), Dukkha (unsatisfactoriness), and Anatta (selflessness--the
absence of a permanent, unchanging, entity that we call Soul or Self).

These truths are not present in Buddhist teaching as dogmas demanding blind
faith. The Buddhists feel that these truths are universal and self-evident to
anyone who cares to investigate in a proper way. Mindfulness is the method of
investigation.

Mindfulness alone has the power to reveal the deepest level of reality available
to human observation. At this level of inspection, one sees the following: (i)
all conditioned things are inherently transitory; (ii) every worldly thing is, in
the end, unsatisfying; and (iii) there are really no entities that are unchanging
or permanent, only processes.

Mindfulness works like and electron microscope. That is, it operates on so fine
a level that one can actually see directly those realities which are at best
theoretical constructs to the conscious thought process. Mindfulness actually
sees the impermanent character of every perception. It sees the transitory and
passing nature of everything that is perceived. It also sees the inherently
unsatisfactory nature of all conditioned things. It sees that there is no sense
grabbing onto any of these passing shows. Peace and happiness cannot be found
that way. And finally, Mindfulness sees the inherent selflessness of all
phenomena. It sees the way that we have arbitrarily selected a certain bundle of
perceptions, chopped them off from the rest of the surging flow of experience
and then conceptualized them as separate, enduring, entities. Mindfulness
actually sees these things. It does not think about them, it sees them directly.

When it is fully developed, Mindfulness sees these three attributes of existence
directly, instantaneously, and without the intervening medium of conscious
thought. In fact, even the attributes which we just covered are inherently
unified. They don't really exist as separate items. They are purely the result
of our struggle to take this fundamentally simple process called Mindfulness and
express it in the cumbersome and inadequate thought symbols of the conscious
level. Mindfulness is a process, but it does not take place in steps. It is a
holistic process that occurs as a unit: you notice your own lack of Mindfulness;
and that noticing itself is a result of Mindfulness; and Mindfulness is bare
attention; and bare attention is noticing things exactly as they are without
distortion; and the way they are is impermanent (Anicca) , unsatisfactory
(Dukkha), and selfless (Anatta). It all takes place in the space of a few
mind-moments. This does not mean, however, that you will instantly attain
liberation (freedom from all human weaknesses) as a result of your first moment
of Mindfulness. Learning to integrate this material into your conscious life is
another whole process. And learning to prolong this state of Mindfulness is
still another. They are joyous processes, however, and they are well worth the
effort.

\subsection*{Mindfulness (Sati) and Insight (Vipassana) Meditation}
Mindfulness is the center
of Vipassana Meditation and the key to the whole process. It is both the goal of
this meditation and the means to that end. You reach Mindfulness by being ever
more mindful. One other Pali word that is translated into English as Mindfulness
is Appamada , which means non-negligence or an absence of madness. One who
attends constantly to what is really going on in one's mind achieves the state
of ultimate sanity.

The Pali term Sati also bears the connotation of remembering. It is not memory
in the sense of ideas and pictures from the past, but rather clear, direct,
wordless knowing of what is and what is not, of what is correct and what is
incorrect, of what we are doing and how we should go about it. Mindfulness
reminds the meditator to apply his attention to the proper object at the proper
time and to exert precisely the amount of energy needed to do the job. When this
energy is properly applied, the meditator stays constantly in a state of calm
and alertness. As long as this condition is maintained, those mind-states call
``hindrances" or ``psychic irritants" cannot arise--there is no greed, no hatred,
no lust or laziness. But we all are human and we do err. Most of us err
repeatedly. Despite honest effort, the meditator lets his Mindfulness slip now
and then and he finds himself stuck in some regrettable, but normal, human
failure. It is Mindfulness that notices that change. And it is Mindfulness that
reminds him to apply the energy required to pull himself out. These slips happen
over and over, but their frequency decreases with practice. Once Mindfulness has
pushed these mental defilements aside, more wholesome states of mind can take
their place. Hatred makes way for loving kindness, lust is replaced by
detachment. It is Mindfulness which notices this change, too, and which reminds
the Vipassana meditator to maintain that extra little mental sharpness needed to
keep these more desirable states of mind. Mindfulness makes possible the growth
of wisdom and compassion. Without Mindfulness they cannot develop to full
maturity.

Deeply buried in the mind, there lies a mental mechanism which accepts what the
mind perceives as beautiful and pleasant experiences and rejects those
experiences which are perceived as ugly and painful. This mechanism gives rise
to those states of mind which we are training ourselves to avoid--things like
greed, lust, hatred, aversion, and jealousy. We choose to avoid these
hindrances, not because they are evil in the normal sense of the word, but
because they are compulsive; because they take the mind over and capture the
attention completely; because they keep going round and round in tight little
circles of thought; and because they seal us off from living reality.

These hindrances cannot arise when Mindfulness is present. Mindfulness is
attention to present time reality, and therefore, directly antithetical to the
dazed state of mind which characterizes impediments. As meditators, it is only
when we let our Mindfulness slip that the deep mechanisms of our mind take over
--grasping, clinging and rejecting. Then resistance emerges and obscures our
awareness. We do not notice that the change is taking place -- we are too busy
with a thought of revenge, or greed, whatever it may be. While an untrained
person will continue in this state indefinitely, a trained meditator will soon
realize what is happening.  It is Mindfulness that notices the change. It is
Mindfulness that remembers the training received and that focuses our attention
so that the confusion fades away. And it is Mindfulness that then attempts to
maintain itself indefinitely so that the resistance cannot arise again. Thus,
Mindfulness is the specific antidote for hindrances. It is both the cure and the
preventive measure.

Fully developed Mindfulness is a state of total non-attachment and utter absence
of clinging to anything in the world. If we can maintain this state, no other
means or device is needed to keep ourselves free of obstructions, to achieve
liberation from our human weaknesses. Mindfulness is non-superficial awareness.
It sees things deeply, down below the level of concepts and opinions. This sort
of deep observation leads to total certainty, and complete absence of confusion.
It manifests itself primarily as a constant and unwavering attention which never
flags and never turns away.

This pure and unstained investigative awareness not only holds mental hindrances
at bay, it lays bare their very mechanism and destroys them. Mindfulness
neutralizes defilements in the mind. The result is a mind which remains
unstained and invulnerable, completely unaffected by the ups and downs of life.
