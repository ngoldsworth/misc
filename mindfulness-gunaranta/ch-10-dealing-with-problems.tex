\chapter{Dealing With Problems}

You are going to run into problems in your
meditation. Everybody does. Problems come in all shapes and sizes, and the only
thing you can be absolutely certain about is that you will have some. The main
trick in dealing with obstacles is to adopt the right attitude. Difficulties are
an integral part of your practice. They aren't something to be avoided. They are
something to be used. They provide invaluable opportunities for learning.

The reason we are all stuck in life's mud is that we ceaselessly run from our
problems and after our desires. Meditation provides us with a laboratory
situation in which we can examine this syndrome and devise strategies for
dealing with it. The various snags and hassles that arise during meditation are
grist for the mill. They are the material on which we work. There is no pleasure
without some degree of pain. There is no pain without some amount of pleasure.
Life is composed of joys and miseries. They go hand-in-hand. Meditation is no
exception. You will experience good times and bad times, ecstasies and
frightening times.

So don't be surprised when you hit some experience that feels like a brick wall.
Don't think you are special. Every seasoned meditator has had his own brick
walls. They come up again and again. Just expect them and be ready to cope. Your
ability to cope with trouble depends upon your attitude. If you can learn to
regard these hassles as opportunities, as chances to develop in your practice,
you'll make progress. Your ability to deal with some issue that arises in
meditation will carry over into the rest of your life and allow you to smooth
out the big issues that really bother you. If you try to avoid each piece of
nastiness that arises in meditation, you are simply reinforcing the habit that
has already made life seem so unbearable at times.

It is essential to learn to confront the less pleasant aspects of existence. Our
job as meditators is to learn to be patient with ourselves, to see ourselves in
an unbiased way, complete with all our sorrows and inadequacies. We have to
learn to be kind to ourselves. In the long run, avoiding unpleasantness is a
very unkind thing to do to yourself. Paradoxically, kindness entails confronting
unpleasantness when it arises. One popular human strategy for dealing with
difficulty is autosuggestion: when something nasty pops up, you convince
yourself it is pleasant rather than unpleasant. The Buddha's tactic is quite the
reverse.

Rather than hide it or disguise it, the Buddha's teaching urges you to examine
it to death. Buddhism advises you not to implant feelings that you don't really
have or avoid feelings that you do have. If you are miserable you are miserable;
this is the reality, that is what is happening, so confront that. Look it square
in the eye without flinching. When you are having a bad time, examine the
badness, observe it mindfully, study the phenomenon and learn its mechanics. The
way out of a trap is to study the trap itself, learn how it is built. You do
this by taking the thing apart piece by piece. The trap can't trap you if it has
been taken to pieces.

The result is freedom.  This point is essential, but it is one of the least
understood aspects of Buddhist philosophy. Those who have studied Buddhism
superficially are quick to conclude that it is a pessimistic set of teachings,
always harping on unpleasant things like suffering, always urging us to confront
the uncomfortable realities of pain, death and illness. Buddhist thinkers do not
regard themselves as pessimists--quite the opposite, actually. Pain exists in
the universe; some measure of it is unavoidable. Learning to deal with it is not
pessimism, but a very pragmatic form of optimism. How would you deal with the
death of your spouse? How would you feel if you lost your mother tomorrow? Or
your sister or your closest friend? Suppose you lost your job, your savings, and
the use of your hands, on the same day; could you face the prospect of spending
the rest of your life in a wheelchair? How are you going to cope with the pain
of terminal cancer if you contract it, and how will you deal with your own
death, when that approaches? You may escape most of these misfortunes, but you
won't escape all of them. Most of us lose friends and relatives at some time
during our lives; all of us get sick now and then; at the very least you are
going to die someday. You can suffer through things like that or you can face
them openly--the choice is yours.

Pain is inevitable, suffering is not. Pain and suffering are two different
animals. If any of these tragedies strike you in your present state of mind, you
will suffer. The habit patterns that presently control your mind will lock you
into that suffering and there will be no escape. A bit of time spent in learning
alternatives to those habit patterns is time will-invested. Most human beings
spend all their energies devising ways to increase their pleasure and decrease
their pain. Buddhism does not advise that you cease this activity altogether.
Money and security are fine. Pain should be avoided where possible. Nobody is
telling you to give away all your possessions or seek out needless pain, but
Buddhism does advise you to invest some of your time and energy in learning to
deal with unpleasantness, because some pain is unavoidable.

When you see a truck bearing down on you, by all means jump out of the way. But
spend some time in meditation, too. Learning to deal with discomfort is the only
way you'll be ready to handle the truck you didn't see.

Problems arise in your practice. Some of them will be physical, some will be
emotional, and some will be attitudinal. All of them are confrontable and each
has its own specific response. All of them are opportunities to free yourself.

\subsection*{Problem 1: Physical Pain} Nobody likes pain, yet everybody has some sometime. It
is one of life's most common experiences and is bound to arise in your
meditation in one form or another. Handling pain is a two-stage process. First,
get rid of the pain if possible or at least get rid of it as much as possible.
Then, if some pain lingers, use it as an abject of meditation.

The first step is physical handling. Maybe the pain is an illness of one sort or
another, a headache, fever, bruises or whatever. In this case, employ standard
medical treatments before you sit down to meditate: take your medicine, apply
your liniment, do whatever you ordinarily do. Then there are certain pains that
are specific to the seated posture. If you never spend much time sitting
cross-legged on the floor, there will be an adjustment period. Some discomfort
is nearly inevitable. According to where the pain is, there are specific
remedies. If the pain is in the leg or knees, check you pants. If they are tight
or made of thick material, that could be the problem. Try to change it. Check
your cushion, too. It should be about three inches in height when compressed.

If the pain is around your waist, try loosening your belt. Loosen the waistband
of your pants is that is necessary. If you experience pain in your lower back,
your posture is probably at fault. Slouching will never be comfortable, so
straighten up. Don't be tight or rigid, but do keep your spine erect. Pain in
the neck or upper back has several sources. The first is improper hand position.
Your hands should be resting comfortably in your lap. Don't pull them up to your
waist. Relax your arms and your neck muscles. Don't let your head droop forward.
Keep it up and aligned with the rest of the spine.

After you have made all these various adjustments, you may find you still have
some lingering pain. If that is the case, try step two. Make the pain your
object of meditation. Don't jump up and down and get excited. Just observe the
pain mindfully. When the pain becomes demanding, you will find it pulling your
attention off the breath. Don't fight back. Just let your attention slide easily
over onto the simple sensation. Go into the pain fully. Don't block the
experience. Explore the feeling. Get beyond your avoiding reaction and go into
the pure sensations that lie below that. You will discover that there are two
things present. The first is the simple sensation--pain itself. Second is your
resistance to that sensation. Resistance reaction is partly mental and partly
physical. The physical part consists of tensing the muscles in and around the
painful area. Relax those muscles. Take them one by one and relax each one very
thoroughly. This step alone probably diminishes the pain significantly. Then go
after the mental side of the resistance. Just as you are tensing physically, you
are also tensing psychologically. You are clamping down mentally on the
sensation of pain, trying to screen it off and reject it from consciousness. The
rejection is a wordless, ``I don't like this feeling" or ``go away" attitude. It
is very subtle. But it is there, and you can find it if you really look. Locate
it and relax that, too.

That last part is more subtle. There are really no human words to describe this
action precisely. The best way to get a handle on it is by analogy. Examine what
you did to those tight muscles and transfer that same action over to the mental
sphere; relax the mind in the same way that you relax the body. Buddhism
recognizes that the body and mind are tightly linked. This is so true that many
people will not see this as a two-step procedure. For them to relax the body is
to relax the mind and vice versa. These people will experience the entire
relaxation, mental and physical, as a single process. In any case, just let go
completely till you awareness slows down past that barrier which you yourself
erected. It was a gap, a sense of distance between self and others. It was a
borderline between `me' and `the pain'. Dissolve that barrier, and separation
vanishes. You slow down into that sea of surging sensation and you merge with
the pain. You become the pain. You watch its ebb and flow and something
surprising happens. It no longer hurts. Suffering is gone. Only the pain
remains, an experience, nothing more. The `me' who was being hurt has gone.

The result is freedom from pain.  This is an incremental process. In the
beginning, you can expect to succeed with small pains and be defeated by big
ones. Like most of our skills, it grows with practice. The more you practice,
the bigger the pain you can handle. Please understand fully.
There is no masochism being advocated here. Self-mortification is not the
point.

This is an exercise in awareness, not in sadism. If the pain becomes
excruciating, go ahead and move, but move slowly and mindfully. Observe your
movements. See how it feels to move. Watch what it does to the pain. Watch the
pain diminish. Try not to move too much though. The less you move, the easier it
is to remain fully mindful. New meditators sometimes say they have trouble
remaining mindful when pain is present. This difficulty stems from a
misunderstanding. These students are conceiving mindfulness as something
distinct from the experience of pain. It is not. Mindfulness never exists by
itself. It always has some object and one object is as good as another. Pain is
a mental state. You can be mindful of pain just as you are mindful of breathing.

The rules we covered in Chapter 4 apply to pain just as they apply to any other
mental state. You must be careful not to reach beyond the sensation and not to
fall short of it. Don't add anything to it, and don't miss any part of it. Don't
muddy the pure experience with concepts or pictures or discursive thinking. And
keep your awareness right in the present time, right with the pain, so that you
won't miss its beginning or its end. Pain not viewed in the clear light of
mindfulness gives rise to emotional reactions like fear, anxiety, or anger. If
it is properly viewed, we have no such reaction. It will be just sensation, just
simple energy. Once you have learned this technique with physical pain, you can then
generalize it in the rest of your life. You can use it on any unpleasant
sensation. What works on pain will work on anxiety or chronic depression. This
technique is one of life's most useful and generalizable skills. It is patience.

\subsection*{Problem 2: Legs Going To Sleep}
 It is very common for beginners to have their legs
fall asleep or go numb during meditation. They are simply not accustomed to the
cross-legged posture. Some people get very anxious about this. They feel they
must get up and move around. A few are completely convinced that they will get
gangrene from lack of circulation. Numbness in the leg is nothing to worry
about. it is caused by nerve-pinch, not by lack of circulation. You can't damage
the tissues of your legs by sitting. So relax. When your legs fall asleep in
meditation, just mindfully observe the phenomenon. Examine what it feels like.
It may be sort of uncomfortable, but it is not painful unless you tense up. Just
stay calm and watch it. It does not matter if your legs go numb and stay that
way for the whole period. After you have meditated for some time, that numbness
gradually will disappear. Your body simply adjusts to daily practice. Then you
can sit for very long sessions with no numbness whatever.

\subsection*{Problem 3: Odd Sensations }
People experience all manner of varied phenomena in
meditation. Some people get itches. Others feel tingling, deep relaxation, a
feeling of lightness or a floating sensation. You may feel yourself growing or
shrinking or rising up in the air. Beginners often get quite excited over such
sensations. As relaxation sets in, the nervous system simply begins to pass
sensory signals more efficiently. Large amounts of previously blocked sensory
data can pour through, giving rise to all manner of unique sensations. It does
not signify anything in particular. It is just sensation. So simply employ the
normal technique. Watch it come up and watch it pass away. Don't get involved.

\subsection*{Problem 4: Drowsiness} It is quite common to experience drowsiness during
meditation. You become very calm and relaxed. That is exactly what is supposed
to happen. Unfortunately, we ordinarily experience this lovely state only when
we are falling asleep, and we associate it with that process. So naturally, you
begin to drift off. When you find this happening, apply your mindfulness to the
state of drowsiness itself. Drowsiness has certain definite characteristics. It
does certain things to your thought process. Find out what. It has certain body
feelings associated with it. Locate those.

This inquisitive awareness is the direct opposite of drowsiness, and will
evaporate it. If it does not, then you should suspect a physical cause of your
sleepiness. Search that out and handle it. If you have just eaten large meal,
that could be the cause. It is best to eat lightly before you meditate. Or wait
an hour after a big meal. And don't overlook the obvious either. If you have
been out loading bricks all day, you are naturally going to be tired. The same
is true if you only got a few hours sleep the night before.

Take care of your body's physical needs. Then meditate. Do not give in to
sleepiness. Stay awake and mindful, for sleep and meditative concentration are
two diametrically opposite experiences. You will not gain any new insight from
sleep, but only from meditation. If you are very sleepy then take a deep breath
and hold it as long as you can. Then breathe out slowly. Take another deep
breath again, hold it as long as you can and breathe out slowly. Repeat this
exercise until your body warms up and sleepiness fades away. Then return to your
breath.

\subsection*{Problem 5: Inability To Concentrate} 
An overactive, jumping attention is something
that everybody experiences from time to time. It is generally handled by
techniques presented in the chapter on distractions. You should also be
informed, however, that there are certain external factors which contribute to
this phenomenon. And these are best handled by simple adjustments in your
schedule. Mental images are powerful entities. They can remain in the mind for
long periods. All of the storytelling arts are direct manipulation of such
material, and to the extent the writer has done his job well, the characters and
images presented will have a powerful and lingering effect on the mind. If you
have been to the best movie of the year, the meditation which follows is going
to be full of those images. If you are halfway through the scariest horror novel
you ever read, your meditation is going to be full of monsters. So switch the
order of events. Do your meditation first. Then read or go to the movies.

Another influential factor is your own emotional state. If there is some real
conflict in your life, that agitation will carry over into meditation. Try to
resolve your immediate daily conflicts before meditation when you can. Your life
will run smoother, and you won't be pondering uselessly in your practice. But
don't use this advice as a way to avoid meditation. Sometimes you can't resolve
every issue before you sit. Just go ahead and sit anyway. Use your meditation to
let go of all the egocentric attitudes that keep you trapped within your own
limited viewpoint. Your problems will resolve much more easily thereafter. And
then there are those days when it seems that the mind will never rest, but your
can't locate any apparent cause. Remember the cyclic alternation we spoke of
earlier. Meditation goes in cycles. You have good days and you have bad days.

Vipassana meditation is primarily an exercise in awareness. Emptying the mind is
not as important as being mindful of what the mind is doing. If you are frantic
and you can't do a thing to stop it, just observe. It is all you. The result
will be one more step forward in your journey of self-exploration. Above all,
don't get frustrated over the nonstop chatter of your mind. That babble is just
one more thing to be mindful of.

\subsection*{Problem 6: Boredom}
It is difficult to imagine anything more inherently boring than sitting still
for an hour with nothing to do but feel the air going in and out of your nose.
You are going to run into boredom repeatedly in your meditation. Everybody does.
Boredom is a mental state and should be treated as such. A few simple strategies
will help you to cope.

\paragraph*{Tactic A: Re-establish true mindfulness.} If the breath seems an exceedingly dull
thing to observe over and over, you may rest assured of one thing: You have
ceased to observe the process with true mindfulness. Mindfulness is never
boring. Look again. Don't assume that you know what breath is.

Don't take it for granted that you have already seen everything there is to see.
If you do, you are conceptualizing the process. You are not observing its living
reality. When you are clearly mindful of breath or indeed anything else, it is
never boring. Mindfulness looks at everything with the eyes of a child, with the
sense of wonder. Mindfulness sees every second as if it were the first and the
only second in the universe. So look again.

\paragraph*{Tactic B: Observe your mental state.} Look at your state of boredom mindfully.
What is boredom? Where is boredom? What does it feel like? What are its mental
component? Does it have any physical feeling? What does it do to your thought
process? Take a fresh look at boredom, as if you have never experienced that
state before.

\subsection*{Problem 7: Fear} States of fear sometimes arise during meditation for no
discernible reason. It is a common phenomenon, and there can be a number of
causes. You may be experiencing the effect of something repressed long ago.
Remember, thoughts arise first in the unconscious.

The emotional contents of a thought complex often leach through into your
conscious awareness long before the thought itself surfaces. If you sit through
the fear, the memory itself may bubble up where you can endure it. Or you may be
dealing directly with that fear which we all fear: `fear of the unknown'. At
some point in your meditation career, you will be struck with the seriousness of
what you are actually doing. You are tearing down the wall of illusion you have
always used to explain life to yourself and to shield yourself from the intense
flame of reality. You are about to meet ultimate truth face to face. That is
scary.

But it has to be dealt with eventually. Go ahead and dive right in.  A third
possibility: the fear that your are feeling may be self- generated. It may be
arising out of unskillful concentration. You may have set an unconscious program
to `examine what comes up.' Thus when a frightening fantasy arises,
concentration locks onto it and the fantasy feeds on the energy of your
attention and grows. The real problem here is that mindfulness is weak. If
mindfulness was strongly developed, it would notice this switch of attention as
soon as it occurred and handle the situation in the usual manner. Not matter
what the source of your fear, mindfulness is the cure. Observe the emotional
reactions that come along and know them for what they are. Stand aside from the
process and don't get involved. Treat the whole dynamic as if you were an
interested bystander. Most importantly, don't fight the situation. Don't try to
repress the memories or the feelings or the fantasies.  Just step out of the way
and let the whole mess bubble up and flow past. It can't hurt you. It is just
memory. It is only fantasy. It is nothing but fear.

When you let it run its course in the arena of conscious attention, it won't
sink back into the unconscious. It won't come back to haunt you later. It will
be gone for good.

\subsection*{Problem 8: Agitation} Restlessness is often a cover-up for some deeper experience
taking place in the unconscious. We humans are great at repressing things.
Rather than confronting some unpleasant thought we experience, we try to bury
it. We won't have to deal with the issue.

Unfortunately, we usually don't succeed, at least not fully. We hide the
thought, but the mental energy we use to cover it up sits there and boils. The
result is that sense of uneasiness which we call agitation or restlessness.
There is nothing you can put your finger on. But you don't feel at ease. You
can't relax. When this uncomfortable state arises in mediation, just observe it.
Don't let it rule you. Don't jump up and run off. And don't struggle with it and
try to make it go away. Just let it be there and watch it closely. Then the
repressed material will eventually surface and you will find out what you have
been worrying about.

The unpleasant experience that you have been trying to avoid could be almost
anything: Guilt, greed or problems. It could be a low-grade pain or subtle
sickness or approaching illness. Whatever it is, let it arise and look at it
mindfully. If you just sit still and observe your agitation, it will eventually
pass. Sitting through restlessness is a little breakthrough in your meditation
career.

It will teach you much. You will find that agitation is actually a rather
superficial mental state. It is inherently ephemeral. It comes and it goes. It
has no real grip on you at all. Here again the rest of your life will profit.

\subsection*{Problem 9 Trying Too Hard} 
Advanced meditators are generally found to be pretty
jovial men and women. They possess that most valuable of all human treasures, a
sense of humor. It is not the superficial witty repartee of the talk show host.
It is a real sense of humor. They can laugh at their own human failures. They
can chuckle at personal disasters. Beginners in meditation are often much too
serious for their own good. So laugh a little. It is important to learn to
loosen up in your session, to relax into your meditation. You need to learn to
flow with whatever happens. You can't do that if you are tensed and striving,
taking it all so very, very seriously. New meditators are often overly eager for
results. They are full of enormous and inflated expectations. They jump right in
and expect incredible results in no time flat. They push. They tense. They sweat and
strain, and it is all so terribly, terribly grim and solemn. This state of
tension is the direct antithesis of mindfulness. So naturally they achieve
little. Then they decide that this meditation is not so exciting after all. It
did not give them what they wanted. They chuck it aside. It should be pointed
out that you learn about meditation only by meditating. You learn what
meditation is all about and where it leads only through direct experience of the
thing itself. Therefore the beginner does not know where he is headed because he
has developed little sense of where his practice is leading.

The novice's expectation is inherently unrealistic and uninformed. As a newcomer
to meditation, he or she would expect all the wrong things, and those
expectations do you no good at all. They get in the way. Trying too hard leads
to rigidity and unhappiness, to guilt and self-condemnation. When you are trying
too hard, your effort becomes mechanical and that defeats mindfulness before it
even gets started. You are well-advised to drop all that. Drop your expectations
and straining. Simply meditate with a steady and balanced effort. Enjoy your
mediation and don't load yourself down with sweat and struggles. Just be
mindful. The meditation itself will take care of the future.

\subsection*{Problem 10: Discouragement} The direct upshot of pushing too hard is frustration.
You are in a state of tension. You get nowhere. You realize you are not making
the progress you expected, so you get discouraged. You feel like a failure. It
is all a very natural cycle, but a totally avoidable one. The source is striving
after unrealistic expectations. Nevertheless, it is a common enough syndrome
and, in spite of all the best advice, you may find it happening to you. There is
a solution. If you find yourself discouraged, just observe your state of mind
clearly. Don't add anything to it. Just watch it. A sense of failure is only
another ephemeral emotional reaction. If you get involved, it feeds on your
energy and grows. If you simply stand aside and watch it, it passes away.

If you are discouraged over your perceived failure in meditation, that is
especially easy to deal with. You feel you have failed in your practice. You
have failed to be mindful. Simply become mindful of that sense of failure. You
have just re-established your mindfulness with that single step. The reason for
your sense of failure is nothing but memory. There is no such thing as failure
in meditation. There are setbacks and difficulties. But there is no failure
unless you give up entirely. Even if you spend twenty solid years getting
nowhere, you can be mindful at any second you choose to do so. It is your
decision. Regretting is only one more way of being unmindful. The instant that
you realize that you have been unmindful, that realization itself is an act of
mindfulness.  So continue the process. Don't get sidetracked in an emotional reaction.

\subsection*{Problem 11 Resistance To Meditation}
There are times when you don't feel like meditating. The very idea seems
obnoxious. Missing a single practice session is scarcely important, but it very
easily becomes a habit. It is wiser to push on through the resistance. Go sit
anyway. Observe this feeling of aversion. In most cases it is a passing emotion,
a flash in the pan that will evaporate right in front of your eyes. Five minutes
after you sid down it is gone. In other cases it is due to some sour mood that
day, and it lasts longer. Still, it does pass. And it is better to get rid of it
in twenty or thirty minutes of meditation than to carry it around with you and
let it ruin the rest of your day.  Another time, resistance may be due to some
difficulty you are having with the practice itself. You may or may not know what
that difficulty is. If the problem is known, handle it by one of the techniques
given in this book. Once the problem is gone, resistance will be gone. If the
problem is unknown, then you are going to have to tough it out. Just sit through
the resistance and observe that mindfully. When it has finally run its course,
it will pass. Then the problem causing it will probably bubble up in its wake,
and you can deal with that.

If resistance to meditation is a common feature of your practice, then you
should suspect some subtle error in your basic attitude.
Meditation is not a ritual conducted in a particular posture. It is not a painful exercise, or period of enforced boredom. And it is
not some grim, solemn, obligation. Meditation is mindfulness. it is a new way of seeing and it is a form of play. Meditation is
your friend. Come to regard it as such and resistance will wash away like smoke on a summer breeze.

If you try all these possibilities and the resistance remains, then there may be
a problem. There can be certain metaphysical snags that a meditator runs into
which go far beyond the scope of this book. It is not common for new meditators
to hit these, but it can happen. Don't give up. Go get help. Seek out qualified
teachers of the Vipassana style of meditation and ask them to help you resolve
the situation. Such people exist for exactly that purpose.

\subsection*{Problem 12 Stupor or Dullness}
We have already discussed the sinking mind
phenomenon. But there is a special route to that state you should watch for.
Mental dullness can result as an unwanted byproduct of deepening concentration.
As your relaxation deepens, muscles loosen and nerve transmission changes. This
produces a very calm and light feeling in the body. you feel very still and
somewhat divorced from the body. this is a very pleasant state and at first your
concentration is quite good, nicely centered on the breath. As it continues,
however, the pleasant feeling intensify and they distract your attention from
the breath. You start to really enjoy that state and your mindfulness goes way
down. Your attention winds up scattered, drifting listlessly through vague
clouds of bliss. The result is a very unmindful state, sort of an ecstatic
stupor. The cure, of course, is mindfulness. Mindfully observe these phenomena
and they will dissipate. When blissful feelings arise accept them. There is no
need to avoid them. Don't get wrapped up in them. They are physical feelings, so
treat them as such. Observe feelings as feelings. Observe dullness as dullness.
Watch them rise and watch them pass. Don't get involved.

You will have problems in meditation. Everybody does. You can treat them as
terrible torments, or as challenges to be overcome.  If you regard them as
burdens, you suffering will only increase. If you regard them as opportunities
to learn and to grow, your spiritual prospects are unlimited.
