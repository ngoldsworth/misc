\chapter{The Practice}
Although there are many subjects of meditation, we
strongly recommend you start with focusing your total undivided attention on
your breathing to gain some degree of shallow concentration. Remember that you
are not practicing a deep absorption or pure concentration technique. You are
practicing mindfulness for which you need only a certain degree of shallow
concentration. You want to cultivate mindfulness culminating in insight and
wisdom to realize the truth as it is. You want to know the working of your
body-mind complex exactly as it is. You want to get rid of all psychological
annoyance to make your life really peaceful and happy.

The mind cannot be purified without seeing things as they really are. ``Seeing
things as they really are" is such a heavily loaded and ambiguous phrase. Many
beginning meditators wonder what we mean, for anyone who has clear eye sight can
see objects as they are.

When we use this phrase in reference to insight gained from our meditation, what
we mean is not seeing things superficially with our regular eyes, but seeing
things with wisdom as they are in themselves. Seeing with wisdom means seeing
things within the framework of our body/mind complex without prejudices or
biases springing from our greed, hatred and delusion. Ordinarily when we watch
the working of our mind/body complex, we tend to hide or ignore things which are
not pleasant to us and to hold onto things which are pleasant. This is because
our minds are generally influenced by our desires, resentment and delusion. Our
ego, self or opinions get in our way and color our judgment.

When we mindfully watch our bodily sensations, we should not confuse them with
mental formations, for bodily sensations can arise without anything to do with
the mind. For instance, we sit comfortably. After a while, there can arise some
uncomfortable feeling on our back or in our legs. Our mind immediately
experiences that discomfort and forms numerous thoughts around the feeling. At
that point, without trying to confuse the feeling with the mental formations, we
should isolate the feeling as feeling and watch it mindfully. Feeling is one of
the seven universal mental factors. The other six are contact, perception,
mental formations, concentration, life force, and awareness.

At another time, we may have a certain emotion such as, resentment, fear, or
lust. Then we should watch the emotion exactly as it is without trying to
confuse it with anything else. When we bundle our form, feeling, perceptions,
mental formations and consciousness up into one and try to watch all of them as feeling, we
get confused, as we will not be able to see the source of feeling. If we simply
dwell upon the feeling alone, ignoring other mental factors, our realization of
truth becomes very difficult. We want to gain the insight into the experience of
impermanence to over come our resentment; our deeper knowledge of unhappiness
overcomes our greed which causes our unhappiness; our realization of
selflessness overcomes ignorance arising from the notion of self. We should see
the mind and body separately first. Having comprehended them separately, we
should see their essential interconnectedness. As our insight becomes sharp, we
become more and more aware of the fact that all the aggregates are cooperating
to work together. None can exist without the other. We can see the real meaning
of the famous metaphor of the blind man who has a healthy body to walk and the
disabled person who has very good eyes to see. Neither of them alone can do much
for himself. But when the disabled person climbs on the shoulders of the blind
man, together they can travel and achieve their goals easily. Similarly, the
body alone can do nothing for itself. It is like a log unable to move or do
anything by itself except to become a subject of impermanence, decay and death.
The mind itself can do nothing without the support of the body. When we
mindfully watch both body and mind, we can see how many wonderful things they do
together.

As long as we are sitting in one place we may gain some degree of mindfulness.
Going to a retreat and spending several days or several months watching our
feelings, perceptions, countless thoughts and various states of consciousness
may make us eventually calm and peaceful. Normally we do not have that much time
to spend in one place meditating all the time. Therefore, we should find a way
to apply our mindfulness to our daily life in order for us to be able to handle
daily unforeseeable eventualities. What we face every day is unpredictable.
Things happen due to multiple causes and conditions, as we are living in a
conditional and impermanent world. Mindfulness is our emergency kit, readily
available at our service at any time. When we face a situation where we feel
indignation, if we mindfully investigate our own mind, we will discover bitter
truths in ourselves. That is we are selfish; we are egocentric; we are attached
to our ego; we hold on to our opinions; we think we are right and everybody else
is wrong; we are prejudices; we are biased; and at the bottom of all of this, we
do not really love ourselves. This discovery, though bitter, is a most rewarding
experience. And in the long run, this discovery delivers us from deeply rooted
psychological and spiritual suffering.

Mindfulness practice is the practice of one hundred percent honesty with
ourselves. When we watch our own mind and body, we notice certain things that
are unpleasant to realize. As we do not like them, we try to reject them. What
are the things we do not like? We do not like to detach ourselves from loved
ones or to live with unloved ones. We include not only people, places and
material things into our likes and dislikes, but opinions, ideas, beliefs and
decisions as well. We do not like what naturally happens to us. We do not like,
for instance, growing old, becoming sick, becoming weak or showing our age, for
we have a great desire to preserve our appearance. We do not like someone
pointing out our faults, for we take great pride in ourselves. We do not like
someone to be wiser than we are, for we are deluded about ourselves. These are
but a few examples of our personal experience of greed, hatred and ignorance.

When greed, hatred and ignorance reveal themselves in our daily lives, we use
our mindfulness to track them down and comprehend their roots. The root of each
of these mental states in within ourselves. If we do not, for instance, have the
root of hatred, nobody can make us angry, for it is the root of our anger that
reacts to somebody's actions or words or behavior. If we are mindful, we will
diligently use our wisdom to look into our own mind. If we do not have hatred in
us we will not be concerned when someone points out our shortcomings. Rather, we
will be thankful to the person who draws our attention to our faults. We have to
be extremely wise and mindful to thank the person who explicates our faults so
we will be able to tread the upward path toward improving ourselves. We all have
blind spots. The other person is our mirror for us to see our faults with
wisdom. We should consider the person who shows our shortcomings as one who
excavates a hidden treasure in us that we were unaware of. It is by knowing the
existence of our deficiencies that we can improve ourselves. Improving ourselves
is the unswerving path to the perfection which is our goal in life. Only by
overcoming weaknesses can we cultivate noble qualities hidden deep down in our
subconscious mind. Before we try to surmount our defects, we should what they
are.

If we are sick, we must find out the cause of our sickness. Only then can we get
treatment. If we pretend that we do not have sickness even though we are
suffering, we will never get treatment. Similarly, if we think that we don't
have these faults, we will never clear our spiritual path. If we are blind to
our own flaws, we need someone to point them out to us. When they point out our
faults, we should be grateful to them like the Venerable Sariputta, who said:
``Even if a seven-year-old novice monk points out my mistakes, I will accept them
with utmost respect for him." Ven. Sariputta was an Arahant who was one hundred
percent mindful and had no fault in him. But since he did not have any pride, he
was able to maintain this position. Although we are not Arahants, we should
determine to emulate his example, for our goal in life also is to attain what he
attained.

Of course the person pointing out our mistakes himself may not be totally free
from defects, but he can see our problems as we can see his faults, which he
does not notice until we point them out to him.

Both pointing out shortcomings and responding to them should be done mindfully.
If someone becomes unmindful in indicating faults and uses unkind and harsh
language, he might do more harm than good to himself as well as to the person
whose shortcomings he points out. One who speaks with resentment cannot be
mindful and is unable to express himself clearly. One who feels hurt while
listening to harsh language may lose his mindfulness and not hear what the other
person is really saying. We should speak mindfully and listen mindfully to be
benefitted by talking and listening. When we listen and talk mindfully, our
minds are free from greed, selfishness, hatred and delusion.

\subsection*{Our Goal}
As meditators, we all must have a goal, for if we do not have a goal,
we will simply be groping in the dark blindly following somebody's instructions
on meditation. There must certainly be a goal for whatever we do consciously and
willingly. It is not the Vipassana meditator's goal to become enlightened before other people
or to have more power or to make more profit than others, for mindfulness
meditators are not in competition with each other.

Our goal is to reach the perfection of all the noble and wholesome qualities
latent in our subconscious mind. This goal has five elements to it: Purification
of mind, overcoming sorrow and lamentation, overcoming pain and grief, treading
the right path leading to attainment of eternal peace, and attaining happiness
by following that path. Keeping this fivefold goal in mind, we can advance with
hope and confidence to reach the goal.

\subsection*{Practice}
Once you sit, do not change the position again until the end of the
time you determined at the beginning. Suppose you change your original position
because it is uncomfortable, and assume another position. What happens after a
while is that the new position becomes uncomfortable. Then you want another and
after a while, it too becomes uncomfortable. So you may go on shifting, moving,
changing one position to another the whole time you are on your mediation
cushion and you may not gain a deep and meaningful level of concentration.
Therefore, do not change your original position, no matter how painful it is.

To avoid changing your position, determine at the beginning of meditation how
long you are going to meditate. If you have never meditated before, sit
motionless not longer than twenty minutes. As you repeat your practice, you can
increase your sitting time.

The length of sitting depends on how much time you have for sitting meditation
practice and how long you can sit without excruciating pain.

We should not have a time schedule to attain the goal, for our attainment
depends on how we progress in our practice based on our understanding and
development of our spiritual faculties. We must work diligently and mindfully
towards the goal without setting any particular time schedule to reach it. When
we are ready, we get there. All we have to do is to prepare ourselves for that
attainment.

After sitting motionless, close your eyes. Our mind is analogous to a cup of
muddy water. The longer you keep a cup of muddy water still, the more mud
settles down and the water will be seen clearly. Similarly, if you keep quiet
without moving you body, focusing your entire undivided attention on the subject
of your meditation, your mind settles down and begins to experience the bliss of
meditation.

To prepare for this attainment, we should keep our mind in the present moment.
The present moment is changing so fast that the casual observer does not seem to
notice its existence at all. Every moment is a moment of events and no moment
passes by without noticing events taking place in that moment. Therefore, the
moment we try to pay bare attention to is the present moment. Our mind goes
through a series of events like a series of pictures passing through a
projector. Some of these pictures are coming from our past experiences and
others are our imaginations of things that we plan to do in the future.

The mind can never be focused without a mental object. Therefore we must give
our mind an object which is readily available every present moment. What is
present every moment is our breath. The mind does not have to make a great
effort to find the breath, for every moment the breath is flowing in and out
through our nostrils. As our practice of insight meditation is taking place
every waking moment, our mind finds it very easy to focus itself on the breath,
for it is more conspicuous and constant than any other object.

After sitting in the manner explained earlier and having shared your
loving-kindness with everybody, take three deep breaths. After taking three deep
breaths, breathe normally, letting your breath flow in and out freely,
effortlessly and begin focusing your attention on the rims of your nostrils.
Simply notice the feeling of breath going in and out. When one inhalation is
complete and before exhaling begins, there is a brief pause. Notice it and
notice the beginning of exhaling. When the exhalation is complete, there is
another brief pause before inhaling begins. Notice this brief pause, too. This
means that there are two brief pauses of breath--one at the end of inhaling, and
the other at the end of exhaling. The two pauses occur in such a brief moment
you may not be aware of their occurrence. But when you are mindful, you can
notice them.

Do not verbalize or conceptualize anything. Simply notice the in-coming and
out-going breath without saying, ``I breathe in", or ``I breathe out." When you
focus your attention on the breath ignore any thought, memory, sound, smell,
taste, etc., and focus your attention exclusively on the breath, nothing else.

At the beginning, both the inhalations and exhalations are short because the
body and mind are not calm and relaxed. Notice the feeling of that short
inhaling and short exhaling as they occur without saying ``short inhaling" or
``short exhaling". As you remain noticing the felling of short inhaling and short
exhaling, your body and mind become relatively calm. Then your breath becomes
long. Notice the feeling of that long breath as it is without saying ``Long
breath". Then notice the entire breathing process from the beginning to the end.
Subsequently the breath becomes subtle, and the mind and body become calmer than
before. Notice this calm and peaceful feeling of your breathing.

\subsection*{What To Do When the Mind Wanders Away?}
In spite of your concerted effort to
keep the mind on your breathing, the mind may wander away. It may go to past
experiences and suddenly you may find yourself remembering places you've
visited, people you met, friends not seen for a long time, a book you read long
ago, the taste of food you ate yesterday, and so on. As soon as you notice that
you mind is no longer on your breath, mindfully bring it back to it and anchor
it there. However, in a few moments you may be caught up again thinking how to
pay your bills, to make a telephone call to you friend, write a letter to
someone, do your laundry, buy your groceries, go to a party, plan your next vacation, and so
forth. As soon as you notice that your mind is not on your subject, bring it
back mindfully.

Following are some suggestions to help you gain the concentration necessary for
the practice of mindfulness.
\subsubsection*{Counting}
In a situation like this, counting may help. The purpose of counting is simply to focus the mind on the breath. Once you mind is
focused on the breath, give up counting. This is a device for gaining concentration. There are numerous ways of counting. Any
counting should be done mentally. Do not make any sound when you count. Following are some of the ways of counting.

\begin{enumerate}
\item While breathing in count ``one, one, one, one..." until the lungs are full of
fresh air. While breathing out count ``two, two, two, two..." until the lungs are
empty of fresh air. Then while breathing in again count ``three, three, three,
three..." until the lungs are full again and while breathing out count again
``four, four, four, four..." until the lungs are empty of fresh air. Count up to
ten and repeat as many times as necessary to keep the mind focused on the
breath.

\item The second method of counting is counting rapidly up to ten. While counting
``one, two, three, four, five, six, seven, eight, nine and ten" breathe in and
again while counting ``one, two, three, four, five, six, seven, eight, nine and
ten" breathe out. This means in one inhaling you should count up to ten and in
one exhaling you should count up to ten. Repeat this way of counting as many
times as necessary to focus the mind on the breath.

\item The third method of counting is to counting secession up to ten. At this time
count ``one, two, three, four, five" (only up to five) while inhaling and then
count ``one, two, three, four, five, six" (up to six) while exhaling. Again count
``one, two, three, four fire, six seven" (only up to seven) while inhaling. Then
count ``one, two, three, four, five, six, seven, eight" while exhaling. Count up
to nine while inhaling and count up to ten while exhaling. Repeat this way of
counting as many times as necessary to focus the mind on the breath.

\item The fourth method is to take a long breath. When the lungs are full, mentally
count ``one" and breath out completely until the lungs are empty of fresh air.
Then count mentally ``two". Take a long breath again and count ``three" and breath
completely out as before. When the lungs are empty of fresh air, count mentally
``four". Count your breath in this manner up to ten. Then count backward from ten
to one. Count again from one to ten and then ten to one.

\item The fifth method is to join inhaling and exhaling. When the lungs are empty
of fresh air, count mentally ``one". This time you should count both inhalation
and exhalation as one. Again inhale, exhale, and mentally count ``two". This way
of counting should be done only up to five and repeated from five to one. Repeat
this method until you breathing becomes refined and quiet.
\end{enumerate}

Remember that you are not supposed to continue your counting all the time. As
soon as your mind is locked at the nostrils-tip where the inhaling breath and
exhaling breath touch and begin to feel that you breathing is so refined and
quiet that you cannot notice inhalation and exhalation separately, you should
give up counting. Counting is used only to train the mind to concentrate on one
point.


\subsubsection*{Connecting} After inhaling do not wait to notice the brief pause before
exhaling but connect the inhaling and exhaling, so you can notice both inhaling
and exhaling as one continuous breath.

\subsubsection*{Fixing} After joining inhaling and exhaling, fix your mind on the point where
you feel you inhaling and exhaling breath touching. Inhale and exhale as on
single breath moving in and out touching or rubbing the rims of your nostrils.

\subsubsection*{Focus you mind like a carpenter} A carpenter draws a straight line on a board
and that he wants to cut. Then he cuts the board with his handsaw along the
straight line he drew. He does not look at the teeth of his saw as they move in
and out of the board. Rather he focuses his entire attention on the line he drew
so he can cut the board straight. Similarly keep your mind straight on the point
where you feel the breath at the rims of your nostrils.

\subsubsection*{Make you mind like a gate-keeper} A gate-keeper does not take into account any
detail of the people entering a house. All he does is notice people entering the
house and leaving the house through the gate. Similarly, when you concentrate
you should not take into account any detail of your experiences. Simply notice
the feeling of your inhaling and exhaling breath as it goes in and out right at
the rims of your nostrils.

As you continue your practice you mind and body becomes so light that you may
feel as if you are floating in the air or on water.
You may even feel that your body is springing up into the sky. When the grossness of your in-and-out breathing has ceased,
subtle in-and-out breathing arises. This very subtle breath is your objective focus of the mind. This is the sign of concentration.
This first appearance of a sign-object will be replaced by more and more subtle
sign-object. This subtlety of the sign can be compared to the sound of a bell.
When a bell is struck with a big iron rod, you hear a gross sound at first. As
the sound faces away, the sound becomes very subtle. Similarly the in-and-out
breath appears at first as a gross sign. As you keep paying bare attention to
it, this sign becomes very subtle. But the consciousness remains totally focused
on the rims of the nostrils. Other meditation objects become clearer and clearer, as the sign
develops. But the breath becomes subtler and subtler as the sign develops.
Because of this subtlety, you may not notice the presence of your breath. Don't
get disappointed thinking that you lost your breath or that nothing is happening
to your meditation practice. Don't worry. Be mindful and determined to bring
your feeling of breath back to the rims of your nostrils. This is the time you
should practice more vigorously, balancing your energy, faith, mindfulness,
concentration and wisdom.

\subsection*{Farmer's simile}
Suppose there is a farmer who uses buffaloes for plowing his
rice field. As he is tired in the middle of the day, he unfastens his buffaloes
and takes a rest under the cool shade of a tree. When he wakes up, he does not
find his animals. He does not worry, but simply walks to the water place where
all the animals gather for drinking in the hot mid-day and he finds his
buffaloes there.

Without any problem he brings them back and ties them to the yoke again and
starts plowing his field.
Similarly as you continue this exercise, your breath becomes so subtle and refined that you might not be able to notice the feeling
of breath at all. When this happens, do not worry. It has not disappeared. It is still where it was before-right at the nostril-tips.

Take a few quick breaths and you will notice the feeling of breathing again.
Continue to pay bare attention to the feeling of the touch of breath at the rims
of your nostrils.

As you keep your mind focused on the rims of your nostrils, you will be able to
notice the sign of the development of meditation.
You will feel the pleasant sensation of sign. Different meditators feel this differently. It will be like a star, or a peg made of
heartwood, or a long string, or a wreath of flowers, or a puff of smoke, or a cob-web, or a film of cloud, or a lotus flower, or the
disc of the moon or the disc of the sun.

Earlier in your practice you had inhaling and exhaling as objects of meditation.
Now you have the sign as the third object of meditation. When you focus your
mind on this third object, your mind reaches a stage of concentration sufficient
for your practice of insight meditation. This sign is strongly present at the
rims of the nostrils. Master it and gain full control of it so that whenever you
want, it should be available. Unite the mind with this sign which is available
in the present moment and let the mind flow with every succeeding moment. As you
pay bare attention to it, you will see the sign itself is changing every moment.
Keep your mind with the changing moments. Also notice that your mind can be
concentrated only on the present moment. This unity of the mind with the present
moment is called momentary concentration. As moments are incessantly passing
away one after another, the mind keeps pace with them. Changing with them,
appearing and disappearing with them without clinging to any of them. If we try
to stop the mind at one moment, we end up in frustration because the mind cannot
be held fast. It must keep up with what is happening in the new moment. As the
present moment can be found any moment, every waking moment can be made a
concentrated moment.

To unite the mind with the present moment, we must find something happening in
that moment. However, you cannot focus your mind on every changing moment
without a certain degree of concentration to keep pace with the moment. Once you
gain this degree of concentration, you can use it for focusing your attention on
anything you experience--the rising and falling of your abdomen, the rising and
falling of the chest area, the rising and falling of any feeling, or the rising
and falling of your breath or thoughts and so on.

To make any progress in insight meditation you need this kind of momentary
concentration. That is all you need for the insight meditation practice because
everything in your experience lives only for one moment. When you focus this
concentrated state of mind on the changes taking place in your mind and body,
you will notice that your breath is the physical part and the feeling of breath,
consciousness of the feeling and the consciousness of the sign are the mental
parts. As you notice them you can notice that they are changing all the time.
You may have various types of sensations, other than the feeling of breathing,
taking place in your body. Watch them all over your body. Don't try to create
any feeling which is not naturally present in any part of your body.

When thought arises notice it, too. All you should notice in all these
occurrences is the impermanent, unsatisfactory and selfless nature of all your
experiences whether mental or physical.

As your mindfulness develops, your resentment for the change, your dislike for
the unpleasant experiences, your greet for the pleasant experiences and the
notion of self hood will be replaced by the deeper insight of impermanence,
unsatisfactoriness and selflessness. This knowledge of reality in your
experience helps you to foster a more calm, peaceful and mature attitude towards
your life. You will see what you thought in the past to be permanent is changing
with such an inconceivable rapidity that even your mind cannot keep up with
these changes. Somehow you will be able to notice many of the changes. You will
see the subtlety of impermanence and the subtlety of selflessness. This insight
will show you the way to peace, happiness and give you the wisdom to handle your
daily problems in life.

When the mind is united with the breath flowing all the time, we will naturally
be able to focus the mind on the present moment.
We can notice the feeling arising from contact of breath with the rim of our nostrils. As the earth element of the air that we breathe
in and out touches the earth element of our nostrils, the mind feels the flow of air in and out. The warm feeling arises at the
nostrils or any other part of the body from the contact of the heat element generated by the breathing process. The feeling of
impermanence of breath arises when the earth element of flowing breath touches the nostrils. Although the water element is present
in the breath, the mind cannot feel it.

Also we feel the expansion and contraction of our lungs, abdomen and low
abdomen, as the fresh air is pumped in and out of the lungs. The expansion and
contraction of the abdomen, lower abdomen and chest are parts of the universal
rhythm. Everything in the universe has the same rhythm of expansion and
contraction just like our breath and body. All of them are rising and falling.

However, our primary concern is the rising and falling phenomena of the breath
and minute parts of our minds and bodies.

Along with the inhaling breath, we experience a small degree of calmness. This
little degree of tension-free calmness turns into tension if we don't breathe
out in a few moments. As we breathe out this tension is released. After
breathing out, we experience discomfort if we wait too long before having fresh
brought in again. This means that every time our lings are full we must breathe
out and every time our lungs are empty we must breathe in. As we breathe in, we
experience a small degree of calmness, and as we breathe out, we experience a
small degree of calmness. We desire calmness and relief of tension and do not
like the tension and feeling resulting from the lack of breath. We wish that the
calmness would stay longer and the tension disappear more quickly that it
normally does. But neither will the tension go away as fast as we wish not the
calmness stay as long as we wish. And again we get agitated or irritated, for we
desire the calmness to return and stay longer and the tension to go away quickly
and not to return again. Here we see how even a small degree of desire for
permanency in an impermanent situation causes pain or unhappiness.
Since there is no self-entity to control this situation, we will become more
disappointed.

However, if we watch our breathing without desiring calmness and without
resenting tension arising from the breathing in and out, but experience only the
impermanence, the unsatisfactoriness and selflessness of our breath, our mind
becomes peaceful and calm.

Also, the mind does not stay all the time with the feeling of breath. It goes to
sounds, memories, emotions, perceptions, consciousness and mental formations as
well. When we experience these states, we should forget about the feeling of
breath and immediately focus our attention on these states--one at a time, not
all of them at one time. As they fade away, we let our mind return to the breath
which is the home base the mind can return to from quick or long journey to
various states of mind and body. We must remember that all these mental
journeys are made within the mind itself.

Every time the mind returns to the breath, it comes back with a deeper insight
into impermanence, unsatisfactoriness and selflessness. The mind becomes more
insightful from the impartial and unbiased watching of these occurrences. The
mind gains insight into the fact that this body, these feelings, various states
of consciousness and numerous mental formations are to be used only for the
purpose of gaining deeper insight into the reality of this mind/body complex.