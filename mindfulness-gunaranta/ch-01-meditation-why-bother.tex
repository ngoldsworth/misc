\chapter{Meditation: Why Bother?}

Meditation is not easy. It takes time and it takes energy. It also takes grit,
determination and discipline. It requires a host of personal qualities which we
normally regard as unpleasant and which we like to avoid whenever possible. We
can sum it all up in the American word `gumption'. Meditation takes `gumption'.
It is certainly a great deal easier just to kick back and watch television. So
why bother? Why waste all that time and energy when you could be out enjoying
yourself? Why bother? Simple.

Because you are human. And just because of the simple fact that you are human,
you find yourself heir to an inherent unsatisfactoriness in life which simply
will not go away. You can suppress it from your awareness for a time. You can
distract yourself for hours on end, but it always comes back--usually when you
least expect it. All of a sudden, seemingly out of the blue, you sit up, take
stock, and realize your actual situation in life.

There you are, and you suddenly realize that you are spending your whole life
just barely getting by. You keep up a good front.  You manage to make ends meed
somehow and look OK from the outside. But those periods of desperation, those
times when you feel everything caving in on you, you keep those to yourself. You
are a mess. And you know it. But you hide it beautifully.

Meanwhile, way down under all that you just know there has got be some other way
to live, some better way to look at the world, some way to touch life more
fully. You click into it by chance now and then. You get a good job. You fall in
love. You win the game. and for a while, things are different. Life takes on a
richness and clarity that makes all the bad times and humdrum fade away. The
whole texture of your experience changes and you say to yourself, ``OK, now I've
made it; now I will be happy".

But then that fades, too, like smoke in the wind. You are left with just a
memory. That and a vague awareness that something is wrong.

But there is really another whole realm of depth and sensitivity available in
life, somehow, you are just not seeing it. You wind up feeling cut off. You feel
insulated from the sweetness of experience by some sort of sensory cotton. You
are not really touching life. You are not making it again. And then even that
vague awareness fades away, and you are back to the same old reality. The world
looks like the usual foul place, which is boring at best. It is an emotional
roller coaster, and you spend a lot of your time down at the bottom of the ramp,
yearning for the heights.

So what is wrong with you? Are you a freak? No. You are just human. And you
suffer from the same malady that infects every human being. It is a monster in
side all of us, and it has many arms: Chronic tension, lack of genuine
compassion for others, including the people closest to you, feelings being
blocked up, and emotional deadness. Many, many arms. None of us is entirely free
from it. We may deny it. We try to suppress it. We build a whole culture around
hiding from it, pretending it is not there, and distracting ourselves from it
with goals and projects and status. But it never goes away. It is a constant
undercurrent in every thought and every perception; a little wordless voice at
the back of the head saying, ``Not good enough yet. Got to have more. Got to make
it better. Got to be better." It is a monster, a monster that manifests
everywhere in subtle forms.

Go to a party. Listen to the laughter, that brittle-tongued voice that says fun
on the surface and fear underneath. Feel the tension, feel the pressure. Nobody
really relaxes. They are faking it. Go to a ball game. Watch the fan in the
stand. Watch the irrational fit of anger. Watch the uncontrolled frustration
bubbling forth from people that masquerades under the guise of enthusiasm, or
team spirit. Booing, cat-calls and unbridled egotism in the name of team
loyalty. Drunkenness, fights in the stands. These are the people trying
desperately to release tension from within. These are not people who are at
peace with themselves. Watch the news on TV. Listen to the lyrics in popular
songs. You find the same theme repeated over and over in variations. Jealousy,
suffering, discontent and stress.

Life seems to be a perpetual struggle, some enormous effort against staggering
odds. And what is our solution to all this dissatisfaction? We get stuck in the
`If only' syndrome. If only I had more money, then I would be happy. If only I
can find somebody who really loves me, if only I can lose 20 pounds, if only I
had a color TV, Jacuzzi, and curly hair, and on and on forever. So where does
all this junk come from and more important, what can we do about it? It comes
from the conditions of our own minds. It is deep, subtle and pervasive set of
mental habits, a Gordian knot which we have built up bit by bit and we can
unravel just the same way, one piece at a time. We can tune up our awareness,
dredge up each separate piece and bring it out into the light. We can make the
unconscious conscious, slowly, one piece at a time.

The essence of our experience is change. Change is incessant. Moment by moment
life flows by and it is never the same. Perpetual alteration is the essence of
the perceptual universe. A thought springs up in you head and half a second
later, it is gone. In comes another one, and that is gone too. A sound strikes
your ears and then silence. Open your eyes and the world pours in, blink and it
is gone. People come into your life and they leave again. Friends go, relatives
die. Your fortunes go up and they go down.

Sometimes you win and just as often you lose. It is incessant: change, change,
change. No two moments ever the same.  There is not a thing wrong with this. It
is the nature of the universe. But human culture has taught u some odd responses
to this endless flowing. We categorize experiences. We try to stick each
perception, every mental change in this endless flow into one of three mental
pigeon holes. It is good, or it is bad, or it is neutral. Then, according to
which box we stick it in, we perceive with a set of fixed habitual mental
responses. If a particular perception has been labeled `good', then we try to
freeze time right there. We grab onto that particular thought, we fondle it, we
hold it, we try to keep it from escaping. When that does not work, we go all-
out in an effort to repeat the experience which caused that thought. Let us call
this mental habit `grasping'.

Over on the other side of the mind lies the box labeled `bad'. When we perceive
something `bad', we try to push it away. We try to deny it, reject it, get rid
of it any way we can. We fight against our own experience. We run from pieces of
ourselves. Let us call this mental habit `rejecting'. Between these two
reactions lies the neutral box. Here we place the experiences which are neither
good nor bad. They are tepid, neutral, uninteresting and boring. We pack
experience away in the neutral box so that we can ignore it and thus return jour
attention to where the action is, namely our endless round of desire and
aversion. This category of experience gets robbed of its fair share of our
attention. Let us call this mental habit `ignoring'. The direct result of all
this lunacy is a perpetual treadmill race to nowhere, endlessly pounding after
pleasure, endlessly fleeing from pain, endlessly ignoring 90 percent of our
experience. Than wondering why life tastes so flat. In the final analysis, it's
a system that does not work.

No matter how hard you pursue pleasure and success, there are times when you
fail. No matter how fast you flee, there are times when pain catches up with
you. And in between those times, life is so boring you could scream. Our minds
are full of opinions and criticisms. We have built walls all around ourselves
and we are trapped with the prison of our own lies and dislikes. We suffer.

Suffering is big word in Buddhist thought. It is a key term and it should be
thoroughly understood. The Pali word is `dukkha', and it does not just mean the
agony of the body. It means the deep, subtle sense of unsatisfactoriness which
is a part of every mental treadmill. The essence of life is suffering, said the
Buddha. At first glance this seems exceedingly morbid and pessimistic.  It even
seems untrue. After all, there are plenty of times when we are happy. Aren't
there? No, there are not. It just seems that way.

Take any moment when you feel really fulfilled and examine it closely. Down
under the joy, you will find that subtle, all- pervasive undercurrent of
tension, that no matter how great the moment is, it is going to end. No matter
how much you just gained, you are either going to lose some of it or spend the
rest of your days guarding what you have got and scheming how to get more. And
in the end, you are going to die. In the end, you lose everything. It is all
transitory.

Sounds pretty bleak, doesn't it? Luckily it's not; not at all. It only sounds
bleak when you view it from the level of ordinary mental perspective, the very
level at which the treadmill mechanism operates. Down under that level lies
another whole perspective, a completely different way to look at the universe.
It is a level of functioning where the mind does not try to freeze time, where
we do not grasp onto our experience as it flows by, where we do not try to block
things out and ignore them. It is a level of experience beyond good and bad,
beyond pleasure and pain. It is a lovely way to perceive the world, and it is a
learnable skill. It is not easy, but is learnable.

Happiness and peace. Those are really the prime issues in human existence. That
is what all of us are seeking. This often is a bit hard to see because we cover
up those basic goals with layers of surface objectives. We want food, we want
money, we want sex, possessions and respect. We even say to ourselves that the
idea of `happiness' is too abstract: ``Look, I am practical. Just give me enough
money and I will buy all the happiness I need". Unfortunately, this is an
attitude that does not work. Examine each of these goals and you will find they
are superficial. You want food. Why? Because I am hungry. So you are hungry, so
what? Well if I eat, I won't be hungry and then I'll feel good. Ah ha! Feel
good! Now there is a real item. What we really seek is not the surface goals.
They are just means to an end. What we are really after is the feeling of relief
that comes when the drive is satisfied.  Relief, relaxation and an end to the
tension. Peace, happiness, no more yearning.

So what is this happiness? For most of us, the perfect happiness would mean
getting everything we wanted, being in control of everything, playing Caesar,
making the whole world dance a jig according to our every whim. Once again, it
does not work that way. Take a look at the people in history who have actually
held this ultimate power. These were not happy people. Most assuredly they were
not men at peace with themselves. Why? Because they were driven to control the
world totally and absolutely and they could not. They wanted to control all men
and there remained men who refused to be controlled. They could not control the
stars. They still got sick. They still had to die.

You can't ever get everything you want. It is impossible. Luckily, there is
another option. You can learn to control your mind, to step outside of this
endless cycle of desire and aversion. You can learn to not want what you want,
to recognize desires but not be controlled by them. This does not mean that you
lie down on the road and invite everybody to walk all over you . It means that
you continue to live a very normal-looking life, but live from a whole new
viewpoint. You do the things that a person must do, but you are free from that
obsessive, compulsive drivenness of your own desires. You want something, but
you don't need to chase after it. You fear something, but you don't need to
stand there quaking in your boots. This sort of mental culture is very
difficult.  It takes years. But trying to control everything is impossible, and
the difficult is preferable to the impossible.

Wait a minute, though. Peace and happiness! Isn't that what civilization is all
about? We build skyscrapers and freeways. We have paid vacations, TV sets. We
provide free hospitals and sick leaves, Social Security and welfare benefits.
All of that is aimed at providing some measure of peace and happiness. Yet the
rate of mental illness climbs steadily, and the crime rates rise faster.  The
streets are crawling with delinquents and unstable individuals. Stick you arms
outside the safety of your own door and somebody is very likely to steal your
watch! Something is not working. A happy man does not feel driven to kill. We
like to think that our society is exploiting every area of human knowledge in
order to achieve peace and happiness.

We are just beginning to realize that we have overdeveloped the material aspect
of existence at the expense of the deeper emotional and spiritual aspect, and we
are paying the price for that error. It is one thing to talk about degeneration
of moral and spiritual fiber in America today, and another thing to do something
about it. The place to start is within ourselves. Look carefully inside, truly
and objectively, and each of us will see moments when ``I am the punk" and ``I am
the crazy". We will learn to see those moments, see them clearly, cleanly and
without condemnation, and we will be on our way up and out of being so.

You can't make radical changes in the pattern of your life until you begin to
see yourself exactly as you are now. As soon as you do that, changes flow
naturally. You don't have to force or struggle or obey rules dictated to you by
some authority. You just change. It is automatic. But arriving at the initial
insight is quite a task. You've got to see who you are and how you are, without
illusion, judgement or resistance of any kind. You've got to see your own place
in society and your function as a social being.

You've got to see your duties and obligations to your fellow human beings, and
above all, your responsibility to yourself as an individual living with other
individuals. And you've got to see all of that clearly and as a unit, a single
gestalt of interrelationship.  It sounds complex, but it often occurs in a
single instant. Mental culture through meditation is without rival in helping
you achieve this sort of understanding and serene happiness.

The Dhammapada is an ancient Buddhist text which anticipated Freud by thousands
of years. It says: ``What you are now is the result of what you were. What you
will be tomorrow will be the result of what you are now. The consequences of an
evil mind will follow you like the cart follows the ox that pulls it. The
consequences of a purified mind will follow you like you own shadow. No one can
do more for you than your own purified mind-- no parent, no relative, no friend,
no one. A well-disciplined mind brings happiness".

Meditation is intended to purify the mind. It cleanses the thought process of
what can be called psychic irritants, things like greed, hatred and jealousy,
things that keep you snarled up in emotional bondage. It brings the mind to a
state of tranquility and awareness, a state of concentration and insight.

In our society, we are great believers in education. We believe that knowledge
makes a cultured person civilized. Civilization, however, polishes the person
superficially. Subject our noble and sophisticated gentleman to stresses of war
or economic collapse, and see what happens. It is one thing to obey the law
because you know the penalties and fear the consequences. It is something else
entirely to obey the law because you have cleansed yourself from the greed that
would make you steal and the hatred that would make you kill. Throw a stone into
a stream. The running water would smooth the surface, but the inner part remains
unchanged. Take that same stone and place it in the intense fires of a forge,
and the whole stone changes inside and outside. It all melts. Civilization
changes man on the outside. Meditation softens him within, through and through.

Meditation is called the Great Teacher. It is the cleansing crucible fire that
works slowly through understanding. The greater your understanding, the more
flexible and tolerant you can be. The greater your understanding, the more
compassionate you can be.  You become like a perfect parent or an ideal teacher.
You are ready to forgive and forget. You feel love towards others because you
understand them. And you understand others because you have understood yourself.
You have looked deeply inside and seen self illusion and your own human
failings. You have seen your own humanity and learned to forgive and to love.
When you have learned compassion for yourself, compassion for others is
automatic. An accomplished meditator has achieved a profound understanding of
life, and he inevitably relates to the world with a deep and uncritical love.

Meditation is a lot like cultivating a new land. To make a field out of a
forest, fist you have to clear the trees and pull out the stumps. Then you till
the soil and you fertilize it. Then you sow your seed and you harvest your
crops. To cultivate your mind, first you have to clear out the various irritants
that are in the way, pull them right out by the root so that they won't grow
back.  Then you fertilize. You pump energy and discipline in the mental soil.
Then you sow the seed and you harvest your crops of faith, morality ,
mindfulness and wisdom.

Faith and morality, by the way, have a special meaning in this context. Buddhism
does not advocate faith in the sense of believing something because it is
written in a book or attributed to a prophet or taught to you by some authority
figure. The meaning here is closer to confidence. It is knowing that something
is true because you have seen it work, because you have observed that very thing
within yourself. In the same way, morality is not a ritualistic obedience to
some exterior, imposed code of behavior.

The purpose of meditation is personal transformation. The you that goes in one
side of the meditation experience is not the same you that comes out the other
side. It changes your character by a process of sensitization, by making you
deeply aware of your own thoughts, word, and deeds. Your arrogance evaporated
and your antagonism dries up. Your mind becomes still and calm. And your life
smoothes out. Thus meditation properly performed prepares you to meet the ups
and down of existence. It reduces your tension, your fear, and your worry.
Restlessness recedes and passion moderates. Things begin to fall into place and
your life becomes a glide instead of a struggle. All of this happens through
understanding.

Meditation sharpens your concentration and your thinking power. Then, piece by
piece, your own subconscious motives and mechanics become clear to you. Your
intuition sharpens. The precision of your thought increases and gradually you
come to a direct knowledge of things as they really are, without prejudice and
without illusion. So is this reason enough to bother? Scarcely.

These are just promises on paper. There is only one way you will ever know if
meditation is worth the effort. Learn to do it right, and do it. See for
yourself.


