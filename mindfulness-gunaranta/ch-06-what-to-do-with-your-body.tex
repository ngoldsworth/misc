\chapter{What To Do With Your Body} 
The practice of meditation has been going
on for several thousand years. That is quite a bit of time for experimentation,
and the procedure has been very, very thoroughly refined. Buddhist practice has
always recognized that the mind and body are tightly linked and that each
influences the other. Thus there are certain recommended physical practices
which will greatly assist you to master your skill. And these practices should
be followed. Keep in mind, however, that these postures are practice aids. Don't
confuse the two. Meditation does not mean sitting in the lotus position. It is a
mental skill. It can be practiced anywhere you wish. But these postures will
help you learn this skill and they speed your progress and development. So use
them.

\subsection*{General Rules} 
The purpose of the various postures is threefold. First, they
provide a stable feeling in the body. This allows you to remove your attention
from such issues as balance and muscular fatigue, so that you can then center
your concentration upon the formal object of meditation. Second, they promote
physical immobility which is then reflected by an immobility of mind. This
creates a deeply settled and tranquil concentration. Third, they give you the
ability to sit for a long period of time without yielding to the meditator's
three main enemies--pain, muscular tension and falling asleep. The most
essential thing is to sit with your back straight. The spine should be erect
with the spinal vertebrae held like a stack of coins, one on top of the other.
Your head should be held in line with the rest of the spine. All of this is done
in a relaxed manner. No Stiffness. You are not a wooden soldier, and there is no
drill sergeant. There should be no muscular tension involved in keeping the back
straight. Sit light and easy. The spine should be like a firm young tree growing
out of soft ground. The rest of the body just hangs from it in a loose, relaxed
manner. This is going to require a bit of experimentation on your part. We
generally sit in tight, guarded postures when we are walking or talking and in
sprawling postures when we are relaxing. Neither of those will do. But they are
cultural habits and they can be re-learned.

Your objective is to achieve a posture in which you can sit for the entire
session without moving at all. In the beginning, you will probably feel a bit
odd to sit with the straight back. But you will get used to it. It takes
practice, and an erect posture is very important. This is what is known in
physiology as a position of arousal, and with it goes mental alertness. If you
slouch, you are inviting drowsiness. What you sit on is equally important. You
are going to need a chair or a cushion, depending on the posture you choose, and
the firmness of the seat must be chosen with some care. Too soft a seat can put
you right to sleep. Too hard can promote pain.

\subsubsection*{Clothing}
 The clothes you wear for meditation should be loose and soft. If they
restrict blood flow or put pressure on nerves, the result will be pain and/or
that tingling numbness which we normally refer to as our `legs going to sleep'.
If you are wearing a belt, loosen it.

Don't wear tight pants or pants made of thick material. Long skirts are a good
choice for women. Loose pants made of thin or elastic material are fine for
anybody. Soft, flowing robes are the traditional garb in Asia and they come in
an enormous variety of styles such as sarongs and kimonos. Take your shoes off
and if your stockings are thick and binding, take them off, too.

\subsubsection*{Traditional Postures}
When you are sitting on the floor in the traditional Asian manner, you need a cushion to elevate your spine. Choose one that is
relatively firm and at least three inches thick when compressed. Sit close to the front edge of the cushion and let your crossed legs
rest on the floor in front of you. If the floor is carpeted, that may be enough to protect your shins and ankles from pressure. If it is
not, you will probably need some sort of padding for your legs. A folded blanket will do nicely. Don't sit all the way back on the
cushion. This position causes its front edge to press into the underside of your thigh, causing nerves to pinch. The result will be
leg pain.

There are a number of ways you can fold your legs. We will list four in
ascending order of preference.
\begin{enumerate}
\item American indian style. Your right foot is tucked under the left knee and left foot is tucked under your right knee.

\item Burmese style. Both of your legs lie flat on the floor from knee to foot.
They are parallel with each other and one in front of the other.

\item Half lotus. Both knees touch the floor. One leg and foot lie flat along the
calf of the other leg.

\item Full lotus. Both knees touch the floor, and your legs are crossed at the calf. Your left foot rests on the right thigh, and
your right foot rests on the left thigh. Both soles turn upward.
\end{enumerate}

In these postures, your hands are cupped one on the other, and they rest on your
lap with the palms turned upward. The hands lie just below the navel with the
bend of each wrist pressed against the thigh. This arm position provides firm
bracing for the upper body. Don't tighten your neck muscles. Relax your arms.
Your diaphragm is held relaxed, expanded to maximum fullness. Don't let tension
build up in the stomach area. Your chin is up. Your eyes can be open or closed.
If you keep them open, fix them on the tip of your nose or in the middle
distance straight in front. You are not looking at anything. You are just
putting your eyes in some arbitrary direction where there is nothing in
particular to see, so that you can forget about vision. Don't strain. Don't
stiffen and don't be rigid. Relax; let the body be natural and supple. Let it
hang from the erect spine like a rag doll.

Half and full lotus positions are the traditional meditation postures in asia.
And the full lotus is considered the best. It is the most solid by far. Once you
are locked into this position, you can be completely immovable for a very long
period. Since it requires a considerable flexibility in the legs, not everybody
can do it. Besides, the main criterion by which you choose a posture for
yourself is not what others say about it. It is your own comfort. Choose a
position which allows you to sit the longest without pain, without moving.
Experiment with different postures. The tendons will loosen with practice. And
then you can work gradually towards the full lotus.

\subsubsection*{Using A Chair} 
Sitting on the floor may not be feasible for you because of pain
or some other reason. No problem. You can always use a chair instead. Pick one
that has a level seat, a straight back and no arms. It is best to sit in such a
way that your back does not lean against the back of the chair. The front of the
seat should not dig into the underside of your thighs. Place your legs side by
side,feet flat on the floor. As with the traditional postures, place both hands
on your lap, cupped one upon the other. Don't tighten your neck or shoulder
muscles, and relax your arms. Your eyes can be open or closed.

In all the above postures, remember your objectives. You want to achieve a state
of complete physical stillness, yet you don't want to fall asleep. Recall the
analogy of the muddy water. You want to promote a totally settled state of the
body which will engender a corresponding mental settling. There must also be a
state of physical alertness which can induce the kind of mental clarity you
seek. So experiment. Your body is a tool for creating desired mental states. Use
it judiciously.
