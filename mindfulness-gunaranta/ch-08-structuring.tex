\chapter{Structuring Your Meditation} 

Everything up to this point has been theory. Now let's dive into the actual
practice. Just how do we go about this thing called meditation.

First of all, you need to establish a formal practice schedule, a specific
period when you will do Vipassana meditation and nothing else. When you were a
baby, you did not know how to walk. Somebody went to a lot of trouble to teach
you that skill. They dragged you by the arms. They gave you lots of
encouragement. Made you put one foot in front of the other until you could do it
by yourself. Those periods of instruction constituted a formal practice in the
art of walking.

In meditation, we follow the same basic procedure. We set aside a certain time,
specifically devoted to developing this mental skill called mindfulness. We
devote these times exclusively to that activity, and we structure our
environment so there will be a minimum of distraction. This is not the easiest
skill in the world to learn. We have spent our entire life developing mental
habits that are really quite contrary to the ideal of uninterrupted mindfulness.
Extricating ourselves from those habits requires a bit of strategy. As we said
earlier, our minds are like cups of muddy water. The object of meditation is to
clarify this sludge so that we can see what is going on in there. The best way
to do that is just let it sit. Give it enough time and it will settle down. You
wind up with clear water. In meditation, we set aside a specific time for this
clarifying process. When viewed from the outside, it looks utterly useless. We
sit there apparently as productive as a stone gargoyle. Inside, however, quite a
bit is happening. The mental soup settles down, and we are left with a clarity
of mind that prepares us to cope with the upcoming events of our lives.

That does not mean that we have to do anything to force this settling. It is a
natural process that happens by itself. The very act of sitting still being
mindful causes this settling. In fact, any effort on our part to force this
settling is counterproductive. That is repression, and it does not work. Try to
force things out of the mind and you merely add energy to them You may succeed
temporarily, but in the long run you will only have made them stronger. They
will hide in the unconscious until you are not watching, then they will leap out
and leave you helpless to fight them off.

The best way to clarify the mental fluid is to just let it settle all by itself.
Don't add any energy to the situation. Just mindfully watch the mud swirl,
without any involvement in the process. Then, when it settles at last, it will
stay settled. We exert energy in meditation, but not force. Our only effort is
gently, patient mindfulness.  The meditation period is like a cross-section of
your whole day. Everything that happens to you is stored away in the mind in
some form, mental or emotional. During normal activity, you get so caught up in
the press of events that the basic issues with which you are dealing are seldom
thoroughly handled. They become buried in the unconscious, where they seethe and
foam and fester. Then you wonder where all that tension came from. All of this
material comes forth in one form or another during your meditation. You get a
chance to look at it, see it for what it is, and let it go. We set up a formal
meditation period in order to create a conducive environment for this release.
We re- establish our mindfulness at regular intervals. We withdraw from those
events which constantly stimulate the mind. We back out of all the activity that
prods the emotions. We go off to a quiet place and we sit still, and it all
comes bubbling out. Then it goes away. The net effect is like recharging a
battery. Meditation recharges your mindfulness.

\subsection*{Where To Sit}
Find yourself a quiet place, a secluded place, a place where you
will be alone. It doesn't have to be some ideal spot in the middle of a forest.
That's nearly impossible for most of us, but it should be a pace where you feel
comfortable, and where you won't be disturbed. It should also be a place where
you won't feel on display. You want all of your attention free for meditation,
not wasted on worries about how you look to others. Try to pick a spot that is
as quiet as possible. It doesn't have to be a soundproof room, but there are
certain noises that are highly distracting, and they should be avoided. Music
and talking are about the worst. The mind tends to be sucked in by these sounds
in an uncontrollable manner, and there goes your concentration.

There are certain traditional aids that you can employ to set the proper mood. A
darkened room with a candle is nice. Incense is nice. A little bell to start and
end your sessions is nice. These are paraphernalia, though. They provide
encouragement to some people, but they are by no means essential to the
practice.

You will probably find it helpful to sit in the same place each time. A special
spot reserved for meditation and nothing else is an aid for most people. You
soon come to associate that spot with the tranquility of deep concentration, and
that association helps you to reach deep states more quickly. The main thing is
to sit in a place that you feel is conductive to your own practice. That
requires a bit of experimentation. Try several spots until you find one where
you feel comfortable. You only need to find a place where you don't feel
self-conscious, and where you can meditate without undue distraction.

Many people find it helpful and supportive to sit with a group of other
meditators. The discipline of regular practice is essential, and most people
find it easier to sit regularly if they are bolstered by a commitment to a group
sitting schedule. You've given your word, and you know you are expected. Thus
the `I'm too busy' syndrome is cleverly skirted. You may be able to locate a
group of practicing meditators in your area. It doesn't matter if they practice
a different form of meditation, so long as it's one of the silent forms. On the
other hand, you also should try to be self-sufficient in your practice. Don't
rely on the presence of a group as your sole motivation to sit. Properly done,
sitting is a pleasure. Use the group as an aid, not as a crutch.

\subsection*{When To Sit}
The most important rule here is this: When it comes to sitting, the
description of Buddhism as the Middle Way applies. Don't overdo it. Don't
underdo it. This doesn't mean you just sit whenever the whim strikes you. It
means you set up a practice schedule and keep to it with a gently, patient
tenacity. Setting up a schedule acts as an encouragement. If, however, you find
that your schedule has ceased to be an encouragement and become a burden, then
something is wrong. Meditation is not a duty, nor an obligation.

Meditation is psychological activity. You will be dealing with the raw stuff of
feelings and emotions. Consequently, it is an activity which is very sensitive
to the attitude with which you approach each session. What you expect is what
you are most likely to get. Your practice will therefore go best when you are
looking forward to sitting. If you sit down expecting grinding drudgery, that is
probably what will occur. So set up a daily pattern that you can live with. Make
it reasonable. Make it fit with the rest of your life. And if it starts to feel
like you're on an uphill treadmill toward liberation, then change something.

First thing in the morning is a great time to meditate. Your mind is fresh then,
before you've gotten yourself buried in responsibilities. Morning meditation is
a fine way to start the day. It tunes you up and gets you ready to deal with
things efficiently. You cruise through the rest of the day just a bit more
lightly. Be sure you are thoroughly awake, though. You won't make much progress
if you are sitting there nodding off, so get enough sleep. Wash your face, or
shower before you begin. You may want to do a bit of exercise beforehand to get
the circulation flowing. Do whatever you need to do in order to wake up fully,
then sit down to meditate. Do not, however, let yourself get hung up in the
day's activities. It's just too easy to forget to sit. Make meditation the first
major thing you do in the morning.

The evening is another good time for practice. Your mind is full of all the
mental rubbish that you have accumulated during the day, and it is great to get
rid of the burden before you sleep. Your meditation will cleanse and rejuvenate
your mind. Re- establish your mindfulness and your sleep will be real sleep.
When you first start meditation, once a day is enough. If you feel like
meditating more, that's fine, but don't overdo it. There's a burn-out phenomenon
we often see in new meditators. They dive right into the practice fifteen hours
a day for a couple of weeks, and then the real world catches up with them. They
decide that this meditation business just takes too much time. Too many
sacrifices are required. They haven't got time for all of this. Don't fall into
that trap. Don't burn yourself out the first week. Make haste slowly. Make your
effort consistent and steady. Give yourself time to incorporate the meditation
practice into your life, and let your practice grow gradually and gently.

As your interest in meditation grows, you'll find yourself making more room in
your schedule for practice. It's a spontaneous phenomenon, and it happens pretty
much by itself--no force necessary.

Seasoned meditators manage three or four hours of practice a day. They live
ordinary lives in the day-to-day world, and they still squeeze it all in. And
they enjoy it. It comes naturally.

\subsection*{How Long To Sit}
A similar rule applies here: Sit as long as you can, but don't
overdo. Most beginners start with twenty or thirty minutes. Initially, it's
difficult to sit longer than that with profit. The posture is unfamiliar to
Westerners, and it takes a bit of time for the body to adjust. The mental skills
are equally unfamiliar, and that adjustment takes time, too.

As you grow accustomed to procedure, you can extend your meditation little by
little. We recommend that after a year or so of steady practice you should be
sitting comfortable for an hour at a time.

Here is an important point, though: Vipassana meditation is not a form of
asceticism. Self-mortification is not the goal. We are trying to cultivate
mindfulness, not pain. Some pain is inevitable, especially in the legs. We will
thoroughly cover pain, and how to handle it, in Chapter 10. There are special
techniques and attitudes which you will learn for dealing with discomfort. The
point to be made here is this: This is not a grim endurance contest. You don't
need to prove anything to anybody. So don't force yourself to sit with
excruciating pain just to be able to say that you sat for an hour. That is a
useless exercise in ego. And don't overdo it in the beginning. Know your
limitations, and don't condemn yourself for not being able to sit forever, like
a rock.

As meditation becomes more and more a part of your life, you can extend your
sessions beyond an hour. As a general rule, just determine what is a comfortable
length of time for you at this point in your life. Then sit five minutes longer
than that. There is no hard and fast rule about length of time for sitting. Even
if you have established a firm minimum, there may be days when it is physically
impossible for you to sit that long. That doesn't mean that you should just
cancel the whole idea for that day. It's crucial to sit regularly. Even ten
minutes of meditation can be very beneficial.

Incidentally, you decide on the length of your session before you meditate.
Don't do it while you are meditating. It's too easy to give in to restlessness
that way, and restlessness is one of the main items that we want to learn to
mindfully observe. So choose a realistic length of time, and then stick to it.

You can use a watch to time you sessions, but don't peek at it every two minutes
to see how you are doing. Your concentration will be completely lost, and
agitation will set in. You'll find your self hoping to get up before the session
is over. That's not meditation--that's clock watching. Don;t look at the clock
until you think the whole meditation period has passed. Actually, you don't need
to consult the clock at all, at least not every time you meditate. In general,
you should be sitting for as long as you want to sit. There is no magic length
of time. It is best, though, to set yourself a minimum length of time. If you
haven't predetermined a minimum, you'll find yourself prone to short sessions.
You'll bolt every time something unpleasant comes up or whenever you feel
restless. That's not good. These experiences are some of the most profitable a
meditator can face, but only if you sit through them. You've got to learn to
observe them calmly and clearly. Look at them mindfully. When you've done that
enough time, they lose their hold on you. You see them for what they are: just
impulses, arising and passing away, just part of the passing show. Your life
smoothes out beautifully as a consequence.

`Discipline' is a difficult word for most of us. It conjures up images of
somebody standing over you with a stick, telling you that you're wrong. But
self-discipline is different. It's the skill of seeing through the hollow
shouting of your own impulses and piercing their secret. They have no power over
you. It's all a show, a deception. Your urges scream and bluster at you; they
cajole; they coax; they threaten; but they really carry no stick at all. You
give in out of habit. You give in because you never really bother to look beyond
the threat. It is all empty back there. There is only one way to learn this
lesson, though. The words on this page won't do it. But look within and watch
the stuff coming up--restlessness, anxiety, impatience, pain-- just watch it
come up and don't get involved. Much to your surprise, it will simply go away.
It rises, it passes away. As simple as that. There is another word for
`self-discipline'. It is `Patience'.