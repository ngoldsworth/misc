\documentclass[11pt, letterpaper]{article}

% PACKAGES
\usepackage[utf8]{inputenc}
\usepackage[T1]{fontenc}
\usepackage{geometry}
\usepackage{times} % Use a Times-like font
\usepackage{epigraph} % For the opening quote

% DOCUMENT GEOMETRY
\geometry{
    letterpaper,
    top=1in,
    bottom=1in,
    left=1in,
    right=2.5in
}

% METADATA
\title{The Work of Art in the Age of Mechanical Reproduction}
\author{Walter Benjamin}
\date{1935}

\usepackage[
	 activate={true,nocompatibility},
	 final,
	 tracking=true,
	 kerning=true,
	 spacing=true,
	 factor=1100,
	 stretch=10,
	 shrink=10
]{microtype}

\microtypecontext{spacing=nonfrench}

% SECTION FORMATTING
\renewcommand{\thesection}{\Roman{section}}

\begin{document}

\maketitle

\begin{center}
    \small
    Source: UCLA School of Theater, Film and Television;\\
    Translated: by Harry Zohn;\\
    Published: by Schocken/Random House, ed. by Hannah Arendt;\\
    Transcribed: by Andy Blunden 1998; proofed and corrected Feb. 2005.
\end{center}

\vspace{2em}

% OPENING QUOTE
\setlength{\epigraphwidth}{0.8\textwidth}
\epigraph{
	``Our fine arts were developed, their types and uses were
	established, in times very different from the present, by men whose power of
	action upon things was insignificant in comparison with ours. But the amazing
	growth of our techniques, the adaptability and precision they have attained,
	the ideas and habits they are creating, make it a certainty that profound
	changes are impending in the ancient craft of the Beautiful. In all the arts
	there is a physical component which can no longer be considered or treated as
	it used to be, which cannot remain unaffected by our modern knowledge and
	power. For the last twenty years neither matter nor space nor time has been
	what it was from time immemorial. We must expect great innovations to transform
	the entire technique of the arts, thereby affecting artistic invention itself
	and perhaps even bringing about an amazing change in our very notion of
	art.''}{--- Paul Valéry, \textit{Pièces sur L’Art}, ``Le Conquete de
	l’ubiquite,'' 1931}

% PREFACE
\section*{Preface}
\addcontentsline{toc}{section}{Preface}

When Marx undertook his critique of the capitalistic mode of production, this
mode was in its infancy. Marx directed his efforts in such a way as to give
them prognostic value. He went back to the basic conditions underlying
capitalistic production and through his presentation showed what could be
expected of capitalism in the future. The result was that one could expect it
not only to exploit the proletariat with increasing intensity, but ultimately
to create conditions which would make it possible to abolish capitalism itself.

The transformation of the superstructure, which takes place far more slowly
than that of the substructure, has taken more than half a century to manifest
in all areas of culture the change in the conditions of production. Only today
can it be indicated what form this has taken. Certain prognostic requirements
should be met by these statements. However, theses about the art of the
proletariat after its assumption of power or about the art of a classless
society would have less bearing on these demands than theses about the
developmental tendencies of art under present conditions of production. Their
dialectic is no less noticeable in the superstructure than in the economy. It
would therefore be wrong to underestimate the value of such theses as a weapon.
They brush aside a number of outmoded concepts, such as creativity and genius,
eternal value and mystery – concepts whose uncontrolled (and at present almost
uncontrollable) application would lead to a processing of data in the Fascist
sense. The concepts which are introduced into the theory of art in what follows
differ from the more familiar terms in that they are completely useless for the
purposes of Fascism. They are, on the other hand, useful for the formulation of
revolutionary demands in the politics of art.

\section{}

In principle a work of art has always been reproducible. Man-made artifacts
could always be imitated by men. Replicas were made by pupils in practice of
their craft, by masters for diffusing their works, and, finally, by third
parties in the pursuit of gain. Mechanical reproduction of a work of art,
however, represents something new. Historically, it advanced intermittently and
in leaps at long intervals, but with accelerated intensity. The Greeks knew
only two procedures of technically reproducing works of art: founding and
stamping. Bronzes, terra cottas, and coins were the only art works which they
could produce in quantity. All others were unique and could not be mechanically
reproduced. With the woodcut graphic art became mechanically reproducible for
the first time, long before script became reproducible by print. The enormous
changes which printing, the mechanical reproduction of writing, has brought
about in literature are a familiar story. However, within the phenomenon which
we are here examining from the perspective of world history, print is merely a
special, though particularly important, case. During the Middle Ages engraving
and etching were added to the woodcut; at the beginning of the nineteenth
century lithography made its appearance.

With lithography the technique of reproduction reached an essentially new
stage. This much more direct process was distinguished by the tracing of the
design on a stone rather than its incision on a block of wood or its etching on
a copperplate and permitted graphic art for the first time to put its products
on the market, not only in large numbers as hitherto, but also in daily
changing forms. Lithography enabled graphic art to illustrate everyday life,
and it began to keep pace with printing. But only a few decades after its
invention, lithography was surpassed by photography. For the first time in the
process of pictorial reproduction, photography freed the hand of the most
important artistic functions which henceforth devolved only upon the eye
looking into a lens. Since the eye perceives more swiftly than the hand can
draw, the process of pictorial reproduction was accelerated so enormously that
it could keep pace with speech. A film operator shooting a scene in the studio
captures the images at the speed of an actor’s speech. Just as lithography
virtually implied the illustrated newspaper, so did photography foreshadow the
sound film. The technical reproduction of sound was tackled at the end of the
last century. These convergent endeavors made predictable a situation which
Paul Valery pointed up in this sentence: 
\begin{quote}
	“Just as water, gas, and electricity are brought into our houses from far off
	to satisfy our needs in response to a minimal effort, so we shall be supplied
	with visual or auditory images, which will appear and disappear at a simple
	movement of the hand, hardly more than a sign.” 
\end{quote}

Around 1900 technical reproduction had reached a standard that not only
permitted it to reproduce all transmitted works of art and thus to cause the
most profound change in their impact upon the public; it also had captured a
place of its own among the artistic processes. For the study of this standard
nothing is more revealing than the nature of the repercussions that these two
different manifestations – the reproduction of works of art and the art of the
film – have had on art in its traditional form.

\section{}

Even the most perfect reproduction of a work of art is lacking in one element:
its presence in time and space, its unique existence at the place where it
happens to be. This unique existence of the work of art determined the history
to which it was subject throughout the time of its existence. This includes the
changes which it may have suffered in physical condition over the years as well
as the various changes in its ownership. The traces of the first can be
revealed only by chemical or physical analyses which it is impossible to
perform on a reproduction; changes of ownership are subject to a tradition
which must be traced from the situation of the original.

The presence of the original is the prerequisite to the concept of
authenticity. Chemical analyses of the patina of a bronze can help to establish
this, as does the proof that a given manuscript of the Middle Ages stems from
an archive of the fifteenth century. The whole sphere of authenticity is
outside technical – and, of course, not only technical – reproducibility.
Confronted with its manual reproduction, which was usually branded as a
forgery, the original preserved all its authority; not so vis-à-vis technical
reproduction. The reason is twofold. First, process reproduction is more
independent of the original than manual reproduction. For example, in
photography, process reproduction can bring out those aspects of the original
that are unattainable to the naked eye yet accessible to the lens, which is
adjustable and chooses its angle at will. And photographic reproduction, with
the aid of certain processes, such as enlargement or slow motion, can capture
images which escape natural vision. Secondly, technical reproduction can put
the copy of the original into situations which would be out of reach for the
original itself. Above all, it enables the original to meet the beholder
halfway, be it in the form of a photograph or a phonograph record. The
cathedral leaves its locale to be received in the studio of a lover of art; the
choral production, performed in an auditorium or in the open air, resounds in
the drawing room.

The situations into which the product of mechanical reproduction can be brought
may not touch the actual work of art, yet the quality of its presence is always
depreciated. This holds not only for the art work but also, for instance, for a
landscape which passes in review before the spectator in a movie. In the case
of the art object, a most sensitive nucleus – namely, its authenticity – is
interfered with whereas no natural object is vulnerable on that score. The
authenticity of a thing is the essence of all that is transmissible from its
beginning, ranging from its substantive duration to its testimony to the
history which it has experienced. Since the historical testimony rests on the
authenticity, the former, too, is jeopardized by reproduction when substantive
duration ceases to matter. And what is really jeopardized when the historical
testimony is affected is the authority of the object.

One might subsume the eliminated element in the term “aura” and go on to say:
that which withers in the age of mechanical reproduction is the aura of the
work of art. This is a symptomatic process whose significance points beyond the
realm of art. One might generalize by saying: the technique of reproduction
detaches the reproduced object from the domain of tradition. By making many
reproductions it substitutes a plurality of copies for a unique existence. And
in permitting the reproduction to meet the beholder or listener in his own
particular situation, it reactivates the object reproduced. These two processes
lead to a tremendous shattering of tradition which is the obverse of the
contemporary crisis and renewal of mankind. Both processes are intimately
connected with the contemporary mass movements. Their most powerful agent is
the film. Its social significance, particularly in its most positive form, is
inconceivable without its destructive, cathartic aspect, that is, the
liquidation of the traditional value of the cultural heritage. This phenomenon
is most palpable in the great historical films. It extends to ever new
positions. In 1927 Abel Gance exclaimed enthusiastically:
\begin{quote}
	“Shakespeare, Rembrandt, Beethoven will make films\ldots all legends, all
	mythologies and all myths, all founders of religion, and the very religions\ldots
	await their exposed resurrection, and the heroes crowd each other at the gate.”
\end{quote}
Presumably without intending it, he issued an invitation to a far-reaching liquidation.

\section{}

During long periods of history, the mode of human sense perception changes with
humanity’s entire mode of existence. The manner in which human sense perception
is organized, the medium in which it is accomplished, is determined not only by
nature but by historical circumstances as well. The fifth century, with its
great shifts of population, saw the birth of the late Roman art industry and
the Vienna Genesis, and there developed not only an art different from that of
antiquity but also a new kind of perception. The scholars of the Viennese
school, Riegl and Wickhoff, who resisted the weight of classical tradition
under which these later art forms had been buried, were the first to draw
conclusions from them concerning the organization of perception at the time.
However far-reaching their insight, these scholars limited themselves to
showing the significant, formal hallmark which characterized perception in late
Roman times. They did not attempt – and, perhaps, saw no way – to show the
social transformations expressed by these changes of perception. The conditions
for an analogous insight are more favorable in the present. And if changes in
the medium of contemporary perception can be comprehended as decay of the aura,
it is possible to show its social causes.

The concept of aura which was proposed above with reference to historical
objects may usefully be illustrated with reference to the aura of natural ones.
We define the aura of the latter as the unique phenomenon of a distance,
however close it may be. If, while resting on a summer afternoon, you follow
with your eyes a mountain range on the horizon or a branch which casts its
shadow over you, you experience the aura of those mountains, of that branch.
This image makes it easy to comprehend the social bases of the contemporary
decay of the aura. It rests on two circumstances, both of which are related to
the increasing significance of the masses in contemporary life. Namely, the
desire of contemporary masses to bring things “closer” spatially and humanly,
which is just as ardent as their bent toward overcoming the uniqueness of every
reality by accepting its reproduction. Every day the urge grows stronger to get
hold of an object at very close range by way of its likeness, its reproduction.
Unmistakably, reproduction as offered by picture magazines and newsreels
differs from the image seen by the unarmed eye. Uniqueness and permanence are
as closely linked in the latter as are transitoriness and reproducibility in
the former. To pry an object from its shell, to destroy its aura, is the mark
of a perception whose “sense of the universal equality of things” has increased
to such a degree that it extracts it even from a unique object by means of
reproduction. Thus is manifested in the field of perception what in the
theoretical sphere is noticeable in the increasing importance of statistics.
The adjustment of reality to the masses and of the masses to reality is a
process of unlimited scope, as much for thinking as for perception.

\section{}

The uniqueness of a work of art is inseparable from its being imbedded in the
fabric of tradition. This tradition itself is thoroughly alive and extremely
changeable. An ancient statue of Venus, for example, stood in a different
traditional context with the Greeks, who made it an object of veneration, than
with the clerics of the Middle Ages, who viewed it as an ominous idol. Both of
them, however, were equally confronted with its uniqueness, that is, its aura.
Originally the contextual integration of art in tradition found its expression
in the cult. We know that the earliest art works originated in the service of a
ritual – first the magical, then the religious kind. It is significant that the
existence of the work of art with reference to its aura is never entirely
separated from its ritual function. In other words, the unique value of the
“authentic” work of art has its basis in ritual, the location of its original
use value. This ritualistic basis, however remote, is still recognizable as
secularized ritual even in the most profane forms of the cult of beauty. The
secular cult of beauty, developed during the Renaissance and prevailing for
three centuries, clearly showed that ritualistic basis in its decline and the
first deep crisis which befell it. With the advent of the first truly
revolutionary means of reproduction, photography, simultaneously with the rise
of socialism, art sensed the approaching crisis which has become evident a
century later. At the time, art reacted with the doctrine of \textit{l’art pour
l’art}, that is, with a theology of art. This gave rise to what might be called
a negative theology in the form of the idea of “pure” art, which not only
denied any social function of art but also any categorizing by subject matter.
(In poetry, Mallarme was the first to take this position.)

An analysis of art in the age of mechanical reproduction must do justice to
these relationships, for they lead us to an all-important insight: for the
first time in world history, mechanical reproduction emancipates the work of
art from its parasitical dependence on ritual. To an ever greater degree the
work of art reproduced becomes the work of art designed for reproducibility.
From a photographic negative, for example, one can make any number of prints;
to ask for the “authentic” print makes no sense. But the instant the criterion
of authenticity ceases to be applicable to artistic production, the total
function of art is reversed. Instead of being based on ritual, it begins to be
based on another practice – politics.

\section{}

Works of art are received and valued on different planes. Two polar types stand
out; with one, the accent is on the cult value; with the other, on the
exhibition value of the work. Artistic production begins with ceremonial
objects destined to serve in a cult. One may assume that what mattered was
their existence, not their being on view. The elk portrayed by the man of the
Stone Age on the walls of his cave was an instrument of magic. He did expose it
to his fellow men, but in the main it was meant for the spirits. Today the cult
value would seem to demand that the work of art remain hidden. Certain statues
of gods are accessible only to the priest in the cella; certain Madonnas remain
covered nearly all year round; certain sculptures on medieval cathedrals are
invisible to the spectator on ground level. With the emancipation of the
various art practices from ritual go increasing opportunities for the
exhibition of their products. It is easier to exhibit a portrait bust that can
be sent here and there than to exhibit the statue of a divinity that has its
fixed place in the interior of a temple. The same holds for the painting as
against the mosaic or fresco that preceded it. And even though the public
presentability of a mass originally may have been just as great as that of a
symphony, the latter originated at the moment when its public presentability
promised to surpass that of the mass.

With the different methods of technical reproduction of a work of art, its
fitness for exhibition increased to such an extent that the quantitative shift
between its two poles turned into a qualitative transformation of its nature.
This is comparable to the situation of the work of art in prehistoric times
when, by the absolute emphasis on its cult value, it was, first and foremost,
an instrument of magic. Only later did it come to be recognized as a work of
art. In the same way today, by the absolute emphasis on its exhibition value
the work of art becomes a creation with entirely new functions, among which the
one we are conscious of, the artistic function, later may be recognized as
incidental. This much is certain: today photography and the film are the most
serviceable exemplifications of this new function.

\section{}

In photography, exhibition value begins to displace cult value all along the
line. But cult value does not give way without resistance. It retires into an
ultimate retrenchment: the human countenance. It is no accident that the
portrait was the focal point of early photography. The cult of remembrance of
loved ones, absent or dead, offers a last refuge for the cult value of the
picture. For the last time the aura emanates from the early photographs in the
fleeting expression of a human face. This is what constitutes their melancholy,
incomparable beauty. But as man withdraws from the photographic image, the
exhibition value for the first time shows its superiority to the ritual value.
To have pinpointed this new stage constitutes the incomparable significance of
Atget, who, around 1900, took photographs of deserted Paris streets. It has
quite justly been said of him that he photographed them like scenes of crime.
The scene of a crime, too, is deserted; it is photographed for the purpose of
establishing evidence. With Atget, photographs become standard evidence for
historical occurrences, and acquire a hidden political significance. They
demand a specific kind of approach; free-floating contemplation is not
appropriate to them. They stir the viewer; he feels challenged by them in a new
way. At the same time picture magazines begin to put up signposts for him,
right ones or wrong ones, no matter. For the first time, captions have become
obligatory. And it is clear that they have an altogether different character
than the title of a painting. The directives which the captions give to those
looking at pictures in illustrated magazines soon become even more explicit and
more imperative in the film where the meaning of each single picture appears to
be prescribed by the sequence of all preceding ones.

\section{}

The nineteenth-century dispute as to the artistic value of painting versus
photography today seems devious and confused. This does not diminish its
importance, however; if anything, it underlines it. The dispute was in fact the
symptom of a historical transformation the universal impact of which was not
realized by either of the rivals. When the age of mechanical reproduction
separated art from its basis in cult, the semblance of its autonomy disappeared
forever. The resulting change in the function of art transcended the
perspective of the century; for a long time it even escaped that of the
twentieth century, which experienced the development of the film.

Earlier much futile thought had been devoted to the question of whether
photography is an art. The primary question – whether the very invention of
photography had not transformed the entire nature of art – was not raised. Soon
the film theoreticians asked the same ill-considered question with regard to
the film. But the difficulties which photography caused traditional aesthetics
were mere child’s play as compared to those raised by the film. Whence the
insensitive and forced character of early theories of the film. Abel Gance, for
instance, compares the film with hieroglyphs: “Here, by a remarkable
regression, we have come back to the level of expression of the Egyptians \ldots
Pictorial language has not yet matured because our eyes have not yet adjusted
to it. There is as yet insufficient respect for, insufficient cult of, what it
expresses.” Or, in the words of Séverin-Mars: “What art has been granted a
dream more poetical and more real at the same time! Approached in this fashion
the film might represent an incomparable means of expression. Only the most
high-minded persons, in the most perfect and mysterious moments of their lives,
should be allowed to enter its ambience.” Alexandre Arnoux concludes his
fantasy about the silent film with the question: “Do not all the bold
descriptions we have given amount to the definition of prayer?” It is
instructive to note how their desire to class the film among the “arts” forces
these theoreticians to read ritual elements into it – with a striking lack of
discretion. Yet when these speculations were published, films like
\textit{L’Opinion publique} and \textit{The Gold Rush} had already appeared.
This, however, did not keep Abel Gance from adducing hieroglyphs for purposes
of comparison, nor Séverin-Mars from speaking of the film as one might speak of
paintings by Fra Angelico. Characteristically, even today ultrareactionary
authors give the film a similar contextual significance – if not an outright
sacred one, then at least a supernatural one. Commenting on Max Reinhardt’s
film version of \textit{A Midsummer Night’s Dream}, Werfel states that
undoubtedly it was the sterile copying of the exterior world with its streets,
interiors, railroad stations, restaurants, motorcars, and beaches which until
now had obstructed the elevation of the film to the realm of art. “The film has
not yet realized its true meaning, its real possibilities \ldots these consist in
its unique faculty to express by natural means and with incomparable
persuasiveness all that is fairylike, marvelous, supernatural.”

\section{}

The artistic performance of a stage actor is definitely presented to the public
by the actor in person; that of the screen actor, however, is presented by a
camera, with a twofold consequence. The camera that presents the performance of
the film actor to the public need not respect the performance as an integral
whole. Guided by the cameraman, the camera continually changes its position
with respect to the performance. The sequence of positional views which the
editor composes from the material supplied him constitutes the completed film.
It comprises certain factors of movement which are in reality those of the
camera, not to mention special camera angles, close-ups, etc. Hence, the
performance of the actor is subjected to a series of optical tests. This is the
first consequence of the fact that the actor’s performance is presented by
means of a camera. Also, the film actor lacks the opportunity of the stage
actor to adjust to the audience during his performance, since he does not
present his performance to the audience in person. This permits the audience to
take the position of a critic, without experiencing any personal contact with
the actor. The audience’s identification with the actor is really an
identification with the camera. Consequently the audience takes the position of
the camera; its approach is that of testing. This is not the approach to which
cult values may be exposed.

\section{}

For the film, what matters primarily is that the actor represents himself to
the public before the camera, rather than representing someone else. One of the
first to sense the actor’s metamorphosis by this form of testing was
Pirandello. Though his remarks on the subject in his novel \textit{Si Gira}
were limited to the negative aspects of the question and to the silent film
only, this hardly impairs their validity. For in this respect, the sound film
did not change anything essential. What matters is that the part is acted not
for an audience but for a mechanical contrivance – in the case of the sound
film, for two of them. “The film actor,” wrote Pirandello, “feels as if in
exile – exiled not only from the stage but also from himself. With a vague
sense of discomfort he feels inexplicable emptiness: his body loses its
corporeality, it evaporates, it is deprived of reality, life, voice, and the
noises caused by his moving about, in order to be changed into a mute image,
flickering an instant on the screen, then vanishing into silence \ldots The
projector will play with his shadow before the public, and he himself must be
content to play before the camera.” This situation might also be characterized
as follows: for the first time – and this is the effect of the film – man has
to operate with his whole living person, yet forgoing its aura. For aura is
tied to his presence; there can be no replica of it. The aura which, on the
stage, emanates from Macbeth, cannot be separated for the spectators from that
of the actor. However, the singularity of the shot in the studio is that the
camera is substituted for the public. Consequently, the aura that envelops the
actor vanishes, and with it the aura of the figure he portrays.

It is not surprising that it should be a dramatist such as Pirandello who, in
characterizing the film, inadvertently touches on the very crisis in which we
see the theater. Any thorough study proves that there is indeed no greater
contrast than that of the stage play to a work of art that is completely
subject to or, like the film, founded in, mechanical reproduction. Experts have
long recognized that in the film “the greatest effects are almost always
obtained by ‘acting’ as little as possible \ldots ” In 1932 Rudolf Arnheim saw
“the latest trend \ldots in treating the actor as a stage prop chosen for its
characteristics and\ldots inserted at the proper place.” With this idea something
else is closely connected. The stage actor identifies himself with the
character of his role. The film actor very often is denied this opportunity.
His creation is by no means all of a piece; it is composed of many separate
performances. Besides certain fortuitous considerations, such as cost of
studio, availability of fellow players, décor, etc., there are elementary
necessities of equipment that split the actor’s work into a series of mountable
episodes. In particular, lighting and its installation require the presentation
of an event that, on the screen, unfolds as a rapid and unified scene, in a
sequence of separate shootings which may take hours at the studio; not to
mention more obvious montage. Thus a jump from the window can be shot in the
studio as a jump from a scaffold, and the ensuing flight, if need be, can be
shot weeks later when outdoor scenes are taken. Far more paradoxical cases can
easily be construed. Let us assume that an actor is supposed to be startled by
a knock at the door. If his reaction is not satisfactory, the director can
resort to an expedient: when the actor happens to be at the studio again he has
a shot fired behind him without his being forewarned of it. The frightened
reaction can be shot now and be cut into the screen version. Nothing more
strikingly shows that art has left the realm of the “beautiful semblance”
which, so far, had been taken to be the only sphere where art could thrive.

\section{}

The feeling of strangeness that overcomes the actor before the camera, as
Pirandello describes it, is basically of the same kind as the estrangement felt
before one’s own image in the mirror. But now the reflected image has become
separable, transportable. And where is it transported? Before the public. Never
for a moment does the screen actor cease to be conscious of this fact. While
facing the camera he knows that ultimately he will face the public, the
consumers who constitute the market. This market, where he offers not only his
labor but also his whole self, his heart and soul, is beyond his reach. During
the shooting he has as little contact with it as any article made in a factory.
This may contribute to that oppression, that new anxiety which, according to
Pirandello, grips the actor before the camera. The film responds to the
shriveling of the aura with an artificial build-up of the “personality” outside
the studio. The cult of the movie star, fostered by the money of the film
industry, preserves not the unique aura of the person but the “spell of the
personality,” the phony spell of a commodity. So long as the movie-makers’
capital sets the fashion, as a rule no other revolutionary merit can be
accredited to today’s film than the promotion of a revolutionary criticism of
traditional concepts of art. We do not deny that in some cases today’s films
can also promote revolutionary criticism of social conditions, even of the
distribution of property. However, our present study is no more specifically
concerned with this than is the film production of Western Europe.

It is inherent in the technique of the film as well as that of sports that
everybody who witnesses its accomplishments is somewhat of an expert. This is
obvious to anyone listening to a group of newspaper boys leaning on their
bicycles and discussing the outcome of a bicycle race. It is not for nothing
that newspaper publishers arrange races for their delivery boys. These arouse
great interest among the participants, for the victor has an opportunity to
rise from delivery boy to professional racer. Similarly, the newsreel offers
everyone the opportunity to rise from passer-by to movie extra. In this way any
man might even find himself part of a work of art, as witness Vertov’s
\textit{Three Songs About Lenin} or Ivens’ \textit{Borinage}. Any man today can
lay claim to being filmed. This claim can best be elucidated by a comparative
look at the historical situation of contemporary literature.

For centuries a small number of writers were confronted by many thousands of
readers. This changed toward the end of the last century. With the increasing
extension of the press, which kept placing new political, religious,
scientific, professional, and local organs before the readers, an increasing
number of readers became writers – at first, occasional ones. It began with the
daily press opening to its readers space for “letters to the editor.” And today
there is hardly a gainfully employed European who could not, in principle, find
an opportunity to publish somewhere or other comments on his work, grievances,
documentary reports, or that sort of thing. Thus, the distinction between
author and public is about to lose its basic character. The difference becomes
merely functional; it may vary from case to case. At any moment the reader is
ready to turn into a writer. As expert, which he had to become willy-nilly in
an extremely specialized work process, even if only in some minor respect, the
reader gains access to authorship. In the Soviet Union work itself is given a
voice. To present it verbally is part of a man’s ability to perform the work.
Literary license is now founded on polytechnic rather than specialized training
and thus becomes common property.

All this can easily be applied to the film, where transitions that in
literature took centuries have come about in a decade. In cinematic practice,
particularly in Russia, this change-over has partially become established
reality. Some of the players whom we meet in Russian films are not actors in
our sense but people who portray themselves and primarily in their own work
process. In Western Europe the capitalistic exploitation of the film denies
consideration to modern man’s legitimate claim to being reproduced. Under these
circumstances the film industry is trying hard to spur the interest of the
masses through illusion-promoting spectacles and dubious speculations.

\section{}

The shooting of a film, especially of a sound film, affords a spectacle
unimaginable anywhere at any time before this. It presents a process in which
it is impossible to assign to a spectator a viewpoint which would exclude from
the actual scene such extraneous accessories as camera equipment, lighting
machinery, staff assistants, etc. – unless his eye were on a line parallel with
the lens. This circumstance, more than any other, renders superficial and
insignificant any possible similarity between a scene in the studio and one on
the stage. In the theater one is well aware of the place from which the play
cannot immediately be detected as illusionary. There is no such place for the
movie scene that is being shot. Its illusionary nature is that of the second
degree, the result of cutting. That is to say, in the studio the mechanical
equipment has penetrated so deeply into reality that its pure aspect freed from
the foreign substance of equipment is the result of a special procedure,
namely, the shooting by the specially adjusted camera and the mounting of the
shot together with other similar ones. The equipment-free aspect of reality
here has become the height of artifice; the sight of immediate reality has
become an orchid in the land of technology.

Even more revealing is the comparison of these circumstances, which differ so
much from those of the theater, with the situation in painting. Here the
question is: How does the cameraman compare with the painter? To answer this we
take recourse to an analogy with a surgical operation. The surgeon represents
the polar opposite of the magician. The magician heals a sick person by the
laying on of hands; the surgeon cuts into the patient’s body. The magician
maintains the natural distance between the patient and himself; though he
reduces it very slightly by the laying on of hands, he greatly increases it by
virtue of his authority. The surgeon does exactly the reverse; he greatly
diminishes the distance between himself and the patient by penetrating into the
patient’s body, and increases it but little by the caution with which his hand
moves among the organs. In short, in contrast to the magician --who is still
hidden in the medical practitioner-- the surgeon at the decisive moment
abstains from facing the patient man to man; rather, it is through the
operation that he penetrates into him.

Magician and surgeon compare to painter and cameraman. The painter maintains in
his work a natural distance from reality, the cameraman penetrates deeply into
its web. There is a tremendous difference between the pictures they obtain.
That of the painter is a total one, that of the cameraman consists of multiple
fragments which are assembled under a new law. Thus, for contemporary man the
representation of reality by the film is incomparably more significant than
that of the painter, since it offers, precisely because of the thoroughgoing
permeation of reality with mechanical equipment, an aspect of reality which is
free of all equipment. And that is what one is entitled to ask from a work of
art.

\section{}

Mechanical reproduction of art changes the reaction of the masses toward art.
The reactionary attitude toward a Picasso painting changes into the progressive
reaction toward a Chaplin movie. The progressive reaction is characterized by
the direct, intimate fusion of visual and emotional enjoyment with the
orientation of the expert. Such fusion is of great social significance. The
greater the decrease in the social significance of an art form, the sharper the
distinction between criticism and enjoyment by the public. The conventional is
uncritically enjoyed, and the truly new is criticized with aversion. With
regard to the screen, the critical and the receptive attitudes of the public
coincide. The decisive reason for this is that individual reactions are
predetermined by the mass audience response they are about to produce, and this
is nowhere more pronounced than in the film. The moment these responses become
manifest they control each other. Again, the comparison with painting is
fruitful. A painting has always had an excellent chance to be viewed by one
person or by a few. The simultaneous contemplation of paintings by a large
public, such as developed in the nineteenth century, is an early symptom of the
crisis of painting, a crisis which was by no means occasioned exclusively by
photography but rather in a relatively independent manner by the appeal of art
works to the masses.

Painting simply is in no position to present an object for simultaneous
collective experience, as it was possible for architecture at all times, for
the epic poem in the past, and for the movie today. Although this circumstance
in itself should not lead one to conclusions about the social role of painting,
it does constitute a serious threat as soon as painting, under special
conditions and, as it were, against its nature, is confronted directly by the
masses. In the churches and monasteries of the Middle Ages and at the princely
courts up to the end of the eighteenth century, a collective reception of
paintings did not occur simultaneously, but by graduated and hierarchized
mediation. The change that has come about is an expression of the particular
conflict in which painting was implicated by the mechanical reproducibility of
paintings. Although paintings began to be publicly exhibited in galleries and
salons, there was no way for the masses to organize and control themselves in
their reception. Thus the same public which responds in a progressive manner
toward a grotesque film is bound to respond in a reactionary manner to
surrealism.

\section{}

The characteristics of the film lie not only in the manner in which man
presents himself to mechanical equipment but also in the manner in which, by
means of this apparatus, man can represent his environment. A glance at
occupational psychology illustrates the testing capacity of the equipment.
Psychoanalysis illustrates it in a different perspective. The film has enriched
our field of perception with methods which can be illustrated by those of
Freudian theory. Fifty years ago, a slip of the tongue passed more or less
unnoticed. Only exceptionally may such a slip have revealed dimensions of depth
in a conversation which had seemed to be taking its course on the surface.
Since the \textit{Psychopathology of Everyday Life} things have changed. This
book isolated and made analyzable things which had heretofore floated along
unnoticed in the broad stream of perception. For the entire spectrum of
optical, and now also acoustical, perception the film has brought about a
similar deepening of apperception. It is only an obverse of this fact that
behavior items shown in a movie can be analyzed much more precisely and from
more points of view than those presented on paintings or on the stage. As
compared with painting, filmed behavior lends itself more readily to analysis
because of its incomparably more precise statements of the situation. In
comparison with the stage scene, the filmed behavior item lends itself more
readily to analysis because it can be isolated more easily. This circumstance
derives its chief importance from its tendency to promote the mutual
penetration of art and science. Actually, of a screened behavior item which is
neatly brought out in a certain situation, like a muscle of a body, it is
difficult to say which is more fascinating, its artistic value or its value for
science. To demonstrate the identity of the artistic and scientific uses of
photography which heretofore usually were separated will be one of the
revolutionary functions of the film.

By close-ups of the things around us, by focusing on hidden details of familiar
objects, by exploring common place milieus under the ingenious guidance of the
camera, the film, on the one hand, extends our comprehension of the necessities
which rule our lives; on the other hand, it manages to assure us of an immense
and unexpected field of action. Our taverns and our metropolitan streets, our
offices and furnished rooms, our railroad stations and our factories appeared
to have us locked up hopelessly. Then came the film and burst this prison-world
asunder by the dynamite of the tenth of a second, so that now, in the midst of
its far-flung ruins and debris, we calmly and adventurously go traveling. With
the close-up, space expands; with slow motion, movement is extended. The
enlargement of a snapshot does not simply render more precise what in any case
was visible, though unclear: it reveals entirely new structural formations of
the subject. So, too, slow motion not only presents familiar qualities of
movement but reveals in them entirely unknown ones “which, far from looking
like retarded rapid movements, give the effect of singularly gliding, floating,
supernatural motions.” Evidently a different nature opens itself to the camera
than opens to the naked eye – if only because an unconsciously penetrated space
is substituted for a space consciously explored by man. Even if one has a
general knowledge of the way people walk, one knows nothing of a person’s
posture during the fractional second of a stride. The act of reaching for a
lighter or a spoon is familiar routine, yet we hardly know what really goes on
between hand and metal, not to mention how this fluctuates with our moods. Here
the camera intervenes with the resources of its lowerings and liftings, its
interruptions and isolations, it extensions and accelerations, its enlargements
and reductions. The camera introduces us to unconscious optics as does
psychoanalysis to unconscious impulses.

\section{}

One of the foremost tasks of art has always been the creation of a demand which
could be fully satisfied only later. The history of every art form shows
critical epochs in which a certain art form aspires to effects which could be
fully obtained only with a changed technical standard, that is to say, in a new
art form. The extravagances and crudities of art which thus appear,
particularly in the so-called decadent epochs, actually arise from the nucleus
of its richest historical energies. In recent years, such barbarisms were
abundant in Dadaism. It is only now that its impulse becomes discernible:
Dadaism attempted to create by pictorial – and literary – means the effects
which the public today seeks in the film.

Every fundamentally new, pioneering creation of demands will carry beyond its
goal. Dadaism did so to the extent that it sacrificed the market values which
are so characteristic of the film in favor of higher ambitions – though of
course it was not conscious of such intentions as here described. The Dadaists
attached much less importance to the sales value of their work than to its
uselessness for contemplative immersion. The studied degradation of their
material was not the least of their means to achieve this uselessness. Their
poems are “word salad” containing obscenities and every imaginable waste
product of language. The same is true of their paintings, on which they mounted
buttons and tickets. What they intended and achieved was a relentless
destruction of the aura of their creations, which they branded as reproductions
with the very means of production. Before a painting of Arp’s or a poem by
August Stramm it is impossible to take time for contemplation and evaluation as
one would before a canvas of Derain’s or a poem by Rilke. In the decline of
middle-class society, contemplation became a school for asocial behavior; it
was countered by distraction as a variant of social conduct. Dadaistic
activities actually assured a rather vehement distraction by making works of
art the center of scandal. One requirement was foremost: to outrage the public.

From an alluring appearance or persuasive structure of sound the work of art of
the Dadaists became an instrument of ballistics. It hit the spectator like a
bullet, it happened to him, thus acquiring a tactile quality. It promoted a
demand for the film, the distracting element of which is also primarily
tactile, being based on changes of place and focus which periodically assail
the spectator. Let us compare the screen on which a film unfolds with the
canvas of a painting. The painting invites the spectator to contemplation;
before it the spectator can abandon himself to his associations. Before the
movie frame he cannot do so. No sooner has his eye grasped a scene than it is
already changed. It cannot be arrested. Duhamel, who detests the film and knows
nothing of its significance, though something of its structure, notes this
circumstance as follows: “I can no longer think what I want to think. My
thoughts have been replaced by moving images.” The spectator’s process of
association in view of these images is indeed interrupted by their constant,
sudden change. This constitutes the shock effect of the film, which, like all
shocks, should be cushioned by heightened presence of mind. By means of its
technical structure, the film has taken the physical shock effect out of the
wrappers in which Dadaism had, as it were, kept it inside the moral shock
effect.

\section{}

The mass is a matrix from which all traditional behavior toward works of art
issues today in a new form. Quantity has been transmuted into quality. The
greatly increased mass of participants has produced a change in the mode of
participation. The fact that the new mode of participation first appeared in a
disreputable form must not confuse the spectator. Yet some people have launched
spirited attacks against precisely this superficial aspect. Among these,
Duhamel has expressed himself in the most radical manner. What he objects to
most is the kind of participation which the movie elicits from the masses.
Duhamel calls the movie “a pastime for helots, a diversion for uneducated,
wretched, worn-out creatures who are consumed by their worries a spectacle
which requires no concentration and presupposes no intelligence which kindles
no light in the heart and awakens no hope other than the ridiculous one of
someday becoming a ‘star’ in Los Angeles.” Clearly, this is at bottom the same
ancient lament that the masses seek distraction whereas art demands
concentration from the spectator. That is a commonplace.

The question remains whether it provides a platform for the analysis of the
film. A closer look is needed here. Distraction and concentration form polar
opposites which may be stated as follows: A man who concentrates before a work
of art is absorbed by it. He enters into this work of art the way legend tells
of the Chinese painter when he viewed his finished painting. In contrast, the
distracted mass absorbs the work of art. This is most obvious with regard to
buildings. Architecture has always represented the prototype of a work of art
the reception of which is consummated by a collectivity in a state of
distraction. The laws of its reception are most instructive.

Buildings have been man’s companions since primeval times. Many art forms have
developed and perished. Tragedy begins with the Greeks, is extinguished with
them, and after centuries its “rules” only are revived. The epic poem, which
had its origin in the youth of nations, expires in Europe at the end of the
Renaissance. Panel painting is a creation of the Middle Ages, and nothing
guarantees its uninterrupted existence. But the human need for shelter is
lasting. Architecture has never been idle. Its history is more ancient than
that of any other art, and its claim to being a living force has significance
in every attempt to comprehend the relationship of the masses to art. Buildings
are appropriated in a twofold manner: by use and by perception – or rather, by
touch and sight. Such appropriation cannot be understood in terms of the
attentive concentration of a tourist before a famous building. On the tactile
side there is no counterpart to contemplation on the optical side. Tactile
appropriation is accomplished not so much by attention as by habit. As regards
architecture, habit determines to a large extent even optical reception. The
latter, too, occurs much less through rapt attention than by noticing the
object in incidental fashion. This mode of appropriation, developed with
reference to architecture, in certain circumstances acquires canonical value.
For the tasks which face the human apparatus of perception at the turning
points of history cannot be solved by optical means, that is, by contemplation,
alone. They are mastered gradually by habit, under the guidance of tactile
appropriation.

The distracted person, too, can form habits. More, the ability to master
certain tasks in a state of distraction proves that their solution has become a
matter of habit. Distraction as provided by art presents a covert control of
the extent to which new tasks have become soluble by apperception. Since,
moreover, individuals are tempted to avoid such tasks, art will tackle the most
difficult and most important ones where it is able to mobilize the masses.
Today it does so in the film. Reception in a state of distraction, which is
increasing noticeably in all fields of art and is symptomatic of profound
changes in apperception, finds in the film its true means of exercise. The film
with its shock effect meets this mode of reception halfway. The film makes the
cult value recede into the background not only by putting the public in the
position of the critic, but also by the fact that at the movies this position
requires no attention. The public is an examiner, but an absent-minded one.

% EPILOGUE
\section*{Epilogue}
\addcontentsline{toc}{section}{Epilogue}

The growing proletarianization of modern man and the increasing formation of
masses are two aspects of the same process. Fascism attempts to organize the
newly created proletarian masses without affecting the property structure which
the masses strive to eliminate. Fascism sees its salvation in giving these
masses not their right, but instead a chance to express themselves. The masses
have a right to change property relations; Fascism seeks to give them an
expression while preserving property. The logical result of Fascism is the
introduction of aesthetics into political life. The violation of the masses,
whom Fascism, with its Führer cult, forces to their knees, has its counterpart
in the violation of an apparatus which is pressed into the production of ritual
values.

All efforts to render politics aesthetic culminate in one thing: war. War and
war only can set a goal for mass movements on the largest scale while
respecting the traditional property system. This is the political formula for
the situation. The technological formula may be stated as follows: Only war
makes it possible to mobilize all of today’s technical resources while
maintaining the property system. It goes without saying that the Fascist
apotheosis of war does not employ such arguments. Still, Marinetti says in his
manifesto on the Ethiopian colonial war:
\begin{quote}
	“For twenty-seven years we Futurists have rebelled against the branding of war
	as anti-aesthetic\ldots Accordingly we state:\ldots War is beautiful because it
	establishes man’s dominion over the subjugated machinery by means of gas masks,
	terrifying megaphones, flame throwers, and small tanks. War is beautiful
	because it initiates the dreamt-of metalization of the human body. War is
	beautiful because it enriches a flowering meadow with the fiery orchids of
	machine guns. War is beautiful because it combines the gunfire, the cannonades,
	the cease-fire, the scents, and the stench of putrefaction into a symphony. War
	is beautiful because it creates new architecture, like that of the big tanks,
	the geometrical formation flights, the smoke spirals from burning villages, and
	many others \ldots Poets and artists of Futurism! \ldots remember these principles of
	an aesthetics of war so that your struggle for a new literature and a new
	graphic art \ldots may be illumined by them!”
\end{quote}
This manifesto has the virtue of clarity. Its formulations deserve to be
accepted by dialecticians. To the latter, the aesthetics of today’s war appears
as follows: If the natural utilization of productive forces is impeded by the
property system, the increase in technical devices, in speed, and in the
sources of energy will press for an unnatural utilization, and this is found in
war. The destructiveness of war furnishes proof that society has not been
mature enough to incorporate technology as its organ, that technology has not
been sufficiently developed to cope with the elemental forces of society. The
horrible features of imperialistic warfare are attributable to the discrepancy
between the tremendous means of production and their inadequate utilization in
the process of production – in other words, to unemployment and the lack of
markets. Imperialistic war is a rebellion of technology which collects, in the
form of “human material,” the claims to which society has denied its natural
materrial. Instead of draining rivers, society directs a human stream into a
bed of trenches; instead of dropping seeds from airplanes, it drops incendiary
bombs over cities; and through gas warfare the aura is abolished in a new way.

“\textit{Fiat ars – pereat mundus}”, says Fascism, and, as Marinetti admits,
expects war to supply the artistic gratification of a sense perception that has
been changed by technology. This is evidently the consummation of
“\textit{l’art pour l’art}.” Mankind, which in Homer’s time was an object of
contemplation for the Olympian gods, now is one for itself. Its self-alienation
has reached such a degree that it can experience its own destruction as an
aesthetic pleasure of the first order. This is the situation of politics which
Fascism is rendering aesthetic. Communism responds by politicizing art.

\end{document}
