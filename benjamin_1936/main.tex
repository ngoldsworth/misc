\documentclass{article}

\title{The Work of Art in the Age of Mechanical Reproduction}
\author{Walver Benjamin}
\date{1935}

\begin{document}

\maketitle

\begin{verbatim}
In: Illuminations,
edited by Hannah Arendt,
translated by Harry Zohn, from the 1935 essay
New York: Schocken Books, 1969
\end{verbatim}

\begin{quote}
``Our fine arts were developed, their types and uses were established, in times
very different from the present, by men whose power of action upon things was
insignificant in comparison with ours. But the amazing growth of our
techniques, the adaptability and precision they have attained, the ideas and
habits they are creating, make it a certainty that profound changes are
impending in the ancient craft of the Beautiful. In all the arts there is a
physical component which can no longer be considered or treated as it used to
be, which cannot remain unaffected by our modern knowledge and power. For
the last twenty years neither matter nor space nor time has been what it was
from time immemorial. We must expect great innovations to transform the
entire technique of the arts, thereby affecting artistic invention itself and
perhaps even bringing about an amazing change in our very notion of art.''
\footnote{
	Quoted from Paul Valéry, Aesthetics, “The Conquest of Ubiquity,” translated by Ralph
Manheim, p. 225. Pantheon Books, Bollingen Series, New York, 1964.}
\end{quote}


\section*{Preface}

When Marx undertook his critique of the capitalistic mode of production, this
mode was in its infancy. Marx directed his efforts in such a way as to give
them prognostic value. He went back to the basic conditions underlying
capitalistic production and through his presentation showed what could be
expected of capitalism in the future. The result was that one could expect it
not only to exploit the proletariat with increasing intensity, but ultimately
to create conditions which would make it possible to abolish capitalism itself.

The transformation of the superstructure, which takes place far more slowly
than that of the substructure, has taken more than half a century to manifest
in all areas of culture the change in the conditions of production. Only today
can it be indicated what form this has taken. Certain prognostic requirements
should be met by these statements. However, theses about the art of the
proletariat after its assumption of power or about the art of a classless
society would have less bearing on these demands than theses about the
developmental tendencies of art under present conditions of production. Their
dialectic is no less noticeable in the superstructure than in the economy. It
would therefore be wrong to underestimate the value of such theses as a weapon.
They brush aside a number of outmoded concepts, such as creativity and genius,
eternal value and mystery – concepts whose uncontrolled (and at present almost
uncontrollable) application would lead to a processing of data in the Fascist
sense. The concepts which are introduced into the theory of art in what follows
differ from the more familiar terms in that they are completely useless for the
purposes of Fascism. They are, on the other hand, useful for the formulation of
revolutionary demands in the politics of art.

\section{}

 In principle a work of art has always been reproducible. Man-made artifacts
 could always be imitated by men. Replicas were made by pupils in practice of
 their craft, by masters for diffusing their works, and, finally, by third
 parties in the pursuit of gain. Mechanical reproduction of a work of art,
 however, represents something new. Historically, it advanced intermittently
 and in leaps at long intervals, but with accelerated intensity. The Greeks
 knew only two procedures of technically reproducing works of art: founding and
 stamping. Bronzes, terra cottas, and coins were the only art works which they
 could produce in quantity. All others were unique and could not be
 mechanically reproduced. With the woodcut graphic art became mechanically
 reproducible for the first time, long before script became reproducible by
 print. The enormous changes which printing, the mechanical reproduction of
 writing, has brought about in literature are a familiar story. However, within
 the phenomenon which we are here examining from the perspective of world
 history, print is merely a special, though particularly important, case.
 During the Middle Ages engraving and etching were added to the woodcut; at the
 beginning of the nineteenth century lithography made its appearance. With
 lithography the technique of reproduction reached an essentially new stage.
 This much more direct process was distinguished by the tracing of the design
 on a stone rather than its incision on a block of wood or its etching on a
 copperplate and permitted graphic art for the first time to put its products
 on the market, not only in large numbers as hitherto, but also in daily
 changing forms. Lithography enabled graphic art to illustrate everyday life,
 and it began to keep pace with printing. But only a few decades after its
 invention, lithography was surpassed by photography. For the first time in the
 process of pictorial reproduction, photography freed the hand of the most
 important artistic functions which henceforth devolved only upon the eye
 looking into a lens. Since the eye perceives more swiftly than the hand can
 draw, the process of pictorial reproduction was accelerated so enormously that
 it could keep pace with speech. A film operator shooting a scene in the studio
 captures the images at the speed of an actor’s speech. Just as lithography
 virtually implied the illustrated newspaper, so did photography foreshadow the
 sound film. The technical reproduction of sound was tackled at the end of the
 last century. These convergent endeavors made predictable a situation which
 Paul Valery pointed up in this sentence:

 \begin{quote}
``Just as water, gas, and electricity are brought into our houses from far off
to satisfy our needs in response to a minimal effort, so we shall be supplied
with visual or auditory images, which will appear and disappear at a simple
movement of the hand, hardly more than a sign.''
\end{quote}

Around 1900 technical reproduction had reached a standard that not only
permitted it to reproduce all transmitted works of art and thus to cause the
most profound change in their impact upon the public; it also had captured a
place of its own among the artistic processes. For the study of this standard
nothing is more revealing than the nature of the repercussions that these two
different manifestations – the reproduction of works of art and the art of the
film – have had on art in its traditional form.

\section*{}
 Even the most perfect reproduction of a work of art is lacking in one element:
 its presence in time and space, its unique existence at the place where it
 happens to be. This unique existence of the work of art determined the history
 to which it was subject throughout the time of its existence. This includes
 the changes which it may have suffered in physical condition over the years as
 well as the various changes in its ownership. The traces of the first can be
 revealed only by chemical or physical analyses which it is impossible to
 perform on a reproduction; changes of ownership are subject to a tradition
 which must be traced from the situation of the original.

The presence of the original is the prerequisite to the concept of
authenticity. Chemical analyses of the patina of a bronze can help to establish
this, as does the proof that a given manuscript of the Middle Ages stems from
an archive of the fifteenth century. The whole sphere of authenticity is
outside technical – and, of course, not only technical – reproducibility.
Confronted with its manual reproduction, which was usually branded as a
forgery, the original preserved all its authority; not so vis-à-vis technical
reproduction. The reason is twofold. First, process reproduction is more
independent of the original than manual reproduction. For example, in
photography, process reproduction can bring out those aspects of the original
that are unattainable to the naked eye yet accessible to the lens, which is
adjustable and chooses its angle at will. And photographic reproduction, with
the aid of certain processes, such as enlargement or slow motion, can capture
images which escape natural vision. Secondly, technical reproduction can put
the copy of the original into situations which would be out of reach for the
original itself. Above all, it enables the original to meet the beholder
halfway, be it in the form of a photograph or a phonograph record. The
cathedral leaves its locale to be received in the studio of a lover of art; the
choral production, performed in an auditorium or in the open air, resounds in
the drawing room.

The situations into which the product of mechanical reproduction can be brought
may not touch the actual work of art, yet the quality of its presence is always
depreciated. This holds not only for the art work but also, for instance, for a
landscape which passes in review before the spectator in a movie. In the case
of the art object, a most sensitive nucleus – namely, its authenticity – is
interfered with whereas no natural object is vulnerable on that score. The
authenticity of a thing is the essence of all that is transmissible from its
beginning, ranging from its substantive duration to its testimony to the
history which it has experienced. Since the historical testimony rests on the
authenticity, the former, too, is jeopardized by reproduction when substantive
duration ceases to matter. And what is really jeopardized when the historical
testimony is affected is the authority of the object.

One might subsume the eliminated element in the term “aura” and go on to say:
that which withers in the age of mechanical reproduction is the aura of the
work of art. This is a symptomatic process whose significance points beyond the
realm of art. One might generalize by saying: the technique of reproduction
detaches the reproduced object from the domain of tradition. By making many
reproductions it substitutes a plurality of copies for a unique existence. And
in permitting the reproduction to meet the beholder or listener in his own
particular situation, it reactivates the object reproduced. These two processes
lead to a tremendous shattering of tradition which is the obverse of the
contemporary crisis and renewal of mankind. Both processes are intimately
connected with the contemporary mass movements. Their most powerful agent is
the film. Its social significance, particularly in its most positive form, is
inconceivable without its destructive, cathartic aspect, that is, the
liquidation of the traditional value of the cultural heritage. This phenomenon
is most palpable in the great historical films. It extends to ever new
positions. In 1927 Abel Gance exclaimed enthusiastically:

\begin{quote}
“Shakespeare, Rembrandt, Beethoven will make films\ldots all legends, all mythologies and all myths, all founders of religion, and the very religions\ldots await their exposed resurrection, and the heroes crowd each other at the gate.”
\end{quote}

Presumably without intending it, he issued an invitation to a far-reaching liquidation.

\end{document}
